\documentclass[a4paper,10pt]{article}
\usepackage{ifpdf}

\ifpdf
\usepackage[pdftex]{graphicx}
\usepackage[pdftex]{hyperref}
\else
\usepackage{graphicx}
\usepackage{hyperref}
\fi

\usepackage{listings}
\usepackage[svgnames]{xcolor}
\usepackage{minted}
\usepackage{amsmath}
\usepackage{amssymb}
\usepackage{xspace}
\usepackage{booktabs}
\usepackage{pifont}
\usepackage{longtable}
\usepackage[top=3cm, bottom=3cm, left=2.5cm,right=2.5cm]{geometry}
%\usepackage[left=3cm,right=3cm]{geometry}

\pagestyle{headings}

\author{MS}
\date{\today}
\title{Residual errors in PharmML}

\colorlet{bkgd}{gray!5}
%\usemintedstyle{trac}

%\newminted{xml}{bgcolor=bkgd,fontsize=\footnotesize%
%,fontfamily=courier%
%}

\newminted{xml}{fontsize=\footnotesize,fontfamily=courier}

% \newcommand{\inputxml}[1]{\inputminted[bgcolor=bkgd,fontsize=\scriptsize%
% ,fontfamily=courier%
% ]{xml}{codesnippets/#1}}

\newcommand{\cellml}{CellML\xspace}
\newcommand{\sbml}{SBML\xspace}
\newcommand{\sedml}{SED-ML\xspace}
\newcommand{\mathml}{MathML\xspace}
\newcommand{\uncertml}{UncertML\xspace}
\newcommand{\pharmml}{PharmML\xspace}
\newcommand{\xelem}[1]{\texttt{<#1>}\index{XML Element!\texttt{<#1>}}}
\newcommand{\xatt}[1]{\texttt{#1}\index{XML Attribute!\texttt{#1}}}
\reversemarginpar  % Want "\watchout" to be put on the left, not the right.
\newcommand{\watchout}{\marginpar{\hspace*{34pt}\raisebox{-0.5ex}{\Large\ding{43}}}}

\begin{document}

\maketitle
\tableofcontents

\newpage

\section{Basic model types}


%------------------------------------------------------1------------------------------------------------------------------
\subsection{Constant/additive error model}
\label{model1}
Model:
\begin{eqnarray}
&& y_{ij} = f_{ij} + a \; \epsilon_{ij}; \quad \epsilon_{ij} \sim N(0,1); \quad \mathit{var}(y_{ij}) = a^2 \nonumber
\end{eqnarray}
\\
Alternative model:
\begin{eqnarray}
&& y_{ij} = f_{ij} + \epsilon_{ij}; \quad \epsilon_{ij} \sim N(0,\sigma^2); \quad \mathit{var}(y_{ij}) = \sigma^2 \nonumber
\end{eqnarray}
PharmML - Function definition \& observational model:
\begin{xmlcode}
<!-- type 1 - ADDITIVE MODEL -->
<ct:FunctionDefinition symbId="additiveErrorModel" symbolType="real">
    <ct:FunctionArgument symbId="a" symbolType="real"/>
    <ct:Definition>
        <ct:SymbRef symbIdRef="a"/>
    </ct:Definition>
</ct:FunctionDefinition>

<ObservationModel blkId="om1">
    <SimpleParameter symbId="a"/>
    <RandomVariable symbId="eps">
        <ct:VariabilityReference>
            <ct:SymbRef blkIdRef="obsErr" symbIdRef="residual"/>
        </ct:VariabilityReference>
        <NormalDistribution xmlns="http://www.uncertml.org/3.0" definition="http://www.uncertml.org/distributions/normal">
            <mean>
                <rVal>0</rVal>
            </mean>
            <stddev>
                <prVal>1</prVal>
            </stddev>
        </NormalDistribution>
    </RandomVariable>
    <Standard symbId="C1">
        <Output>
            <ct:SymbRef blkIdRef="sm1" symbIdRef="C1model"/>
        </Output>
        <ErrorModel>
            <ct:Assign>
                <Equation xmlns="http://www.pharmml.org/2013/03/Maths">
                    <FunctionCall>
                        <ct:SymbRef symbIdRef="additiveErrorModel"/>
                        <FunctionArgument symbId="a">
                            <ct:SymbRef symbIdRef="a"/>
                        </FunctionArgument>
                    </FunctionCall>
                </Equation>
            </ct:Assign>
        </ErrorModel>
        <ResidualError>
            <ct:SymbRef symbIdRef="eps"/>
        </ResidualError>
    </Standard>
</ObservationModel>
\end{xmlcode}




%-----------------------------------------------------2-------------------------------------------------------------------
\subsection{Proportional or constant CV (CCV) model error model}
\label{model2}
Model:
\begin{eqnarray}
&& y_{ij} =  f_{ij} + bf_{ij} \; \epsilon_{ij}; \quad \epsilon_{ij} \sim N(0,1); \quad \mathit{var}(y_{ij}) = b^2f_{ij}^2; \quad \mathit{CV} = b \nonumber
\end{eqnarray}
\\
Alternative model:
\begin{eqnarray}
&& y_{ij} =  f_{ij}(1+\epsilon_{1,ij}); \quad \epsilon_{1,ij} \sim N(0,\sigma_1^2); \quad
%&& y_{ij} =  f_{ij}(1+\epsilon_{1,ij}) = f_{ij} +  \sigma_{1} f_{ij} \epsilon_{ij}; \quad
\mathit{var}(y_{ij}) =  \sigma_1^2 f_{ij}^2; \quad \mathit{CV} = \sigma_1 \nonumber
\end{eqnarray}
PharmML - Function definition \& observational model:
\begin{xmlcode}
<!-- type 2 - PROPORTIONAL MODEL -->
<ct:FunctionDefinition symbId="proportionalErrorModel" symbolType="real">
    <ct:FunctionArgument symbId="b" symbolType="real"/>
    <ct:FunctionArgument symbId="f" symbolType="real"/>
    <ct:Definition>
        <Equation xmlns="http://www.pharmml.org/2013/03/Maths">
            <Binop op="times">
                <ct:SymbRef symbIdRef="b"/>
                <ct:SymbRef symbIdRef="f"/>
            </Binop>
        </Equation>
    </ct:Definition>
</ct:FunctionDefinition>

<ObservationModel blkId="om2">
    <SimpleParameter symbId="b"/>
    <RandomVariable symbId="eps">
        <ct:VariabilityReference>
            <ct:SymbRef blkIdRef="obsErr" symbIdRef="residual"/>
        </ct:VariabilityReference>
        <NormalDistribution xmlns="http://www.uncertml.org/3.0" definition="http://www.uncertml.org/distributions/normal">
            <mean>
                <rVal>0</rVal>
            </mean>
            <stddev>
                <prVal>1</prVal>
            </stddev>
        </NormalDistribution>
    </RandomVariable>
    <Standard symbId="C1">
        <Output>
            <ct:SymbRef blkIdRef="sm1" symbIdRef="C1model"/>
        </Output>
        <ErrorModel>
            <ct:Assign>
                <Equation xmlns="http://www.pharmml.org/2013/03/Maths">
                    <FunctionCall>
                        <ct:SymbRef symbIdRef="proportionalErrorModel"/>
                        <FunctionArgument symbId="b">
                            <ct:SymbRef symbIdRef="b"/>
                        </FunctionArgument>
                        <FunctionArgument symbId="f">
                            <ct:SymbRef blkIdRef="sm1" symbIdRef="C1model"/>
                        </FunctionArgument>
                    </FunctionCall>
                </Equation>
            </ct:Assign>
        </ErrorModel>
        <ResidualError>
            <ct:SymbRef symbIdRef="eps"/>
        </ResidualError>
    </Standard>
</ObservationModel>
\end{xmlcode}


%----------------------------------------------------3--------------------------------------------------------------------
\subsection{Combined additive and proportional error model 1}
\label{model3}
Model:
\begin{eqnarray}
&& y_{ij} =  f_{ij} + (a + bf_{ij}) \; \epsilon_{ij}; \quad \epsilon_{ij} \sim N(0,1);\quad \mathit{var}(y_{ij}) = (a + bf_{ij})^2 \nonumber
\end{eqnarray}
PharmML - Function definition \& observational model:
\begin{xmlcode}
<!-- type 3 - COMBINED ADDITIVE and PROPORTIONAL MODEL 1 -->
<ct:FunctionDefinition symbId="combinedAdditiveProportionalModel1" symbolType="real">
    <ct:FunctionArgument symbId="a" symbolType="real"/>
    <ct:FunctionArgument symbId="b" symbolType="real"/>
    <ct:FunctionArgument symbId="f" symbolType="real"/>
    <ct:Definition>
        <Equation xmlns="http://www.pharmml.org/2013/03/Maths">
            <Binop op="plus">
                <ct:SymbRef symbIdRef="a"/>
                <Binop op="times">
                    <ct:SymbRef symbIdRef="b"/>
                    <ct:SymbRef symbIdRef="f"/>
                </Binop>                   
            </Binop>
        </Equation>
    </ct:Definition>
</ct:FunctionDefinition>

<ObservationModel blkId="om3">
    <SimpleParameter symbId="a"/>
    <SimpleParameter symbId="b"/>
    <RandomVariable symbId="eps">
        <ct:VariabilityReference>
            <ct:SymbRef blkIdRef="obsErr" symbIdRef="residual"/>
        </ct:VariabilityReference>
        <NormalDistribution xmlns="http://www.uncertml.org/3.0" definition="http://www.uncertml.org/distributions/normal">
            <mean>
                <rVal>0</rVal>
            </mean>
            <stddev>
                <prVal>1</prVal>
            </stddev>
        </NormalDistribution>
    </RandomVariable>
    <Standard symbId="C1">
        <Output>
            <ct:SymbRef blkIdRef="sm1" symbIdRef="C1model"/>
        </Output>
        <ErrorModel>
            <ct:Assign>
                <Equation xmlns="http://www.pharmml.org/2013/03/Maths">
                    <FunctionCall>
                        <ct:SymbRef symbIdRef="combinedAdditiveProportionalModel1"/>
                        <FunctionArgument symbId="a">
                            <ct:SymbRef symbIdRef="a"/>
                        </FunctionArgument>
                        <FunctionArgument symbId="b">
                            <ct:SymbRef symbIdRef="b"/>
                        </FunctionArgument>
                        <FunctionArgument symbId="f">
                            <ct:SymbRef blkIdRef="sm1" symbIdRef="C1model"/>
                        </FunctionArgument>
                    </FunctionCall>
                </Equation>
            </ct:Assign>
        </ErrorModel>
        <ResidualError>
            <ct:SymbRef symbIdRef="eps"/>
        </ResidualError>
    </Standard>
</ObservationModel>
\end{xmlcode}



%---------------------------------------------------4---------------------------------------------------------------------
\subsection{Combined additive and proportional error model 2}
Following three representations are equivalent assuming uncorrelated $\epsilon_{1,ij}$ and $\epsilon_{2,ij}$.
\label{model4}
Model:
\begin{eqnarray}
&& y_{ij} =  f_{ij} + \sqrt{a^2 + b^2f_{ij}^2} \; \epsilon_{ij}; \quad \epsilon_{ij} \sim N(0,1); \quad \mathit{var}(y_{ij}) = a^2 + b^2f_{ij}^2 \nonumber
\end{eqnarray}
Alternative model 1:
\begin{eqnarray}
&& y_{ij} =  f_{ij} +  a\, \epsilon_{1,ij} + b f_{ij}\, \epsilon_{2,ij}; \quad \epsilon_{1,ij} \sim N(0,1); \quad \epsilon_{2,ij} \sim N(0,1);  \nonumber \\
&& \text{for uncorrelated $\epsilon_{1,ij}$ and $\epsilon_{2,ij}$} \Rightarrow  \mathit{var}(y_{ij}) = a^2 + b^2f_{ij}^2 \nonumber
\end{eqnarray}
Alternative model 2:
\begin{eqnarray}
&& y_{ij} =  f_{ij} (1 + \epsilon_{1,ij}) + \epsilon_{2,ij}; \quad \epsilon_{1,ij} \sim N(0,\sigma_1^2); \quad \epsilon_{2,ij} \sim N(0,\sigma_2^2);  \nonumber \\
&& \text{for uncorrelated $\epsilon_{1,ij}$ and $\epsilon_{2,ij}$} \Rightarrow \mathit{var}(y_{ij}) = \sigma_1^2 f_{ij}^2 + \sigma_2^2 \nonumber
\end{eqnarray}
PharmML - Function definition \& observational model:
\begin{xmlcode}
<!-- type 3 - COMBINED ADDITIVE and PROPORTIONAL MODEL 1 -->
<ct:FunctionDefinition symbId="combinedAdditiveProportionalModel1" symbolType="real">
    <ct:FunctionArgument symbId="a" symbolType="real"/>
    <ct:FunctionArgument symbId="b" symbolType="real"/>
    <ct:FunctionArgument symbId="f" symbolType="real"/>
    <ct:Definition>
        <Equation xmlns="http://www.pharmml.org/2013/03/Maths">
            <Binop op="plus">
                <ct:SymbRef symbIdRef="a"/>
                <Binop op="times">
                    <ct:SymbRef symbIdRef="b"/>
                    <ct:SymbRef symbIdRef="f"/>
                </Binop>                   
            </Binop>
        </Equation>
    </ct:Definition>
</ct:FunctionDefinition>

<ObservationModel blkId="om4">
    <SimpleParameter symbId="a"/>
    <SimpleParameter symbId="b"/>
    <RandomVariable symbId="eps">
        <ct:VariabilityReference>
            <ct:SymbRef blkIdRef="obsErr" symbIdRef="residual"/>
        </ct:VariabilityReference>
        <NormalDistribution xmlns="http://www.uncertml.org/3.0" definition="http://www.uncertml.org/distributions/normal">
            <mean>
                <rVal>0</rVal>
            </mean>
            <stddev>
                <prVal>1</prVal>
            </stddev>
        </NormalDistribution>
    </RandomVariable>
    <Standard symbId="C1">
        <Output>
            <ct:SymbRef blkIdRef="sm1" symbIdRef="C1model"/>
        </Output>
        <ErrorModel>
            <ct:Assign>
                <Equation xmlns="http://www.pharmml.org/2013/03/Maths">
                    <FunctionCall>
                        <ct:SymbRef symbIdRef="combinedAdditiveProportionalModel2"/>
                        <FunctionArgument symbId="a">
                            <ct:SymbRef symbIdRef="a"/>
                        </FunctionArgument>
                        <FunctionArgument symbId="b">
                            <ct:SymbRef symbIdRef="b"/>
                        </FunctionArgument>
                        <FunctionArgument symbId="f">
                            <ct:SymbRef blkIdRef="sm1" symbIdRef="C1model"/>
                        </FunctionArgument>
                    </FunctionCall>
                </Equation>
            </ct:Assign>
        </ErrorModel>
        <ResidualError>
            <ct:SymbRef symbIdRef="eps"/>
        </ResidualError>
    </Standard>
</ObservationModel>
\end{xmlcode}



%----------------------------------------------------5--------------------------------------------------------------------
\subsection{Power error model}
\label{model5}
Model:
\begin{eqnarray}
&& y_{ij} = f_{ij} + b\,f_{ij}^c \; \epsilon_{ij}; \quad \epsilon_{ij} \sim N(0,1); \quad \mathit{var}(y_{ij}) = b^2f_{ij}^{2c} \nonumber
\end{eqnarray}
PharmML - Function definition \& observational model:
\begin{xmlcode}
<!-- type 5 - POWER MODEL -->
<ct:FunctionDefinition symbId="powerModel" symbolType="real">
    <ct:FunctionArgument symbId="b" symbolType="real"/>
    <ct:FunctionArgument symbId="c" symbolType="real"/>
    <ct:FunctionArgument symbId="f" symbolType="real"/>
    <ct:Definition>
        <Equation xmlns="http://www.pharmml.org/2013/03/Maths">
            <Binop op="times">
                <ct:SymbRef symbIdRef="b"/>
                <Binop op="power">
                    <ct:SymbRef symbIdRef="f"/>
                    <ct:SymbRef symbIdRef="c"/>
                </Binop>
            </Binop>
        </Equation>
    </ct:Definition>
</ct:FunctionDefinition>

<ObservationModel blkId="om5">
    <SimpleParameter symbId="a"/>
    <SimpleParameter symbId="b"/>
    <RandomVariable symbId="eps">
        <ct:VariabilityReference>
            <ct:SymbRef blkIdRef="obsErr" symbIdRef="residual"/>
        </ct:VariabilityReference>
        <NormalDistribution xmlns="http://www.uncertml.org/3.0" definition="http://www.uncertml.org/distributions/normal">
            <mean>
                <rVal>0</rVal>
            </mean>
            <stddev>
                <prVal>1</prVal>
            </stddev>
        </NormalDistribution>
    </RandomVariable>
    <Standard symbId="C1">
        <Output>
            <ct:SymbRef blkIdRef="sm1" symbIdRef="C1model"/>
        </Output>
        <ErrorModel>
            <ct:Assign>
                <Equation xmlns="http://www.pharmml.org/2013/03/Maths">
                    <FunctionCall>
                        <ct:SymbRef symbIdRef="powerModel"/>
                        <FunctionArgument symbId="b">
                            <ct:SymbRef symbIdRef="b"/>
                        </FunctionArgument>
                        <FunctionArgument symbId="c">
                            <ct:SymbRef symbIdRef="c"/>
                        </FunctionArgument>
                        <FunctionArgument symbId="f">
                            <ct:SymbRef blkIdRef="sm1" symbIdRef="C1model"/>
                        </FunctionArgument>
                    </FunctionCall>
                </Equation>
            </ct:Assign>
        </ErrorModel>
        <ResidualError>
            <ct:SymbRef symbIdRef="eps"></ct:SymbRef>
        </ResidualError>
    </Standard>
</ObservationModel> 
\end{xmlcode}




%----------------------------------------------------6--------------------------------------------------------------------
\subsection{Combined additive and power error model 1}
\label{model6}
Model:
\begin{eqnarray}
&& y_{ij} =  f_{ij} + (a + b f_{ij}^c) \; \epsilon_{ij}; \quad \epsilon_{ij} \sim N(0,1); \quad \mathit{var}(y_{ij}) = (a + bf_{ij}^c)^2 \nonumber
\end{eqnarray}
PharmML - Function definition \& observational model:
\begin{xmlcode}
<!-- type 6 - COMBINED ADDITIVE and POWER MODEL 1 -->
<ct:FunctionDefinition symbId="combinedAdditivePowerModel1" symbolType="real">
    <ct:FunctionArgument symbId="a" symbolType="real"/>
    <ct:FunctionArgument symbId="b" symbolType="real"/>
    <ct:FunctionArgument symbId="c" symbolType="real"/>
    <ct:FunctionArgument symbId="f" symbolType="real"/>
    <ct:Definition>
        <Equation xmlns="http://www.pharmml.org/2013/03/Maths">
            <Binop op="plus">
                <ct:SymbRef symbIdRef="a"/>
                <Binop op="times">
                    <ct:SymbRef symbIdRef="b"/>
                    <Binop op="power">
                        <ct:SymbRef symbIdRef="f"/>
                        <ct:SymbRef symbIdRef="c"/>
                    </Binop>
                </Binop>
            </Binop>
        </Equation>
    </ct:Definition>
</ct:FunctionDefinition>

<!-- type 6 - COMBINED ADDITIVE and POWER MODEL 1 -->
<ObservationModel blkId="om6">
    <SimpleParameter symbId="a"/>
    <SimpleParameter symbId="b"/>
    <SimpleParameter symbId="c"/>
    <RandomVariable symbId="eps">
        <ct:VariabilityReference>
            <ct:SymbRef blkIdRef="obsErr" symbIdRef="residual"></ct:SymbRef>
        </ct:VariabilityReference>
        <NormalDistribution xmlns="http://www.uncertml.org/3.0" definition="http://www.uncertml.org/distributions/normal">
            <mean>
                <rVal>0</rVal>
            </mean>
            <stddev>
                <prVal>1</prVal>
            </stddev>
        </NormalDistribution>
    </RandomVariable>
    <Standard symbId="C1">
        <Output>
            <ct:SymbRef blkIdRef="sm1" symbIdRef="C1model"></ct:SymbRef>
        </Output>
        <ErrorModel>
            <ct:Assign>
                <Equation xmlns="http://www.pharmml.org/2013/03/Maths">
                    <FunctionCall>
                        <ct:SymbRef symbIdRef="combinedAdditivePowerModel1"/>
                        <FunctionArgument symbId="a">
                            <ct:SymbRef symbIdRef="a"/>
                        </FunctionArgument>
                        <FunctionArgument symbId="b">
                            <ct:SymbRef symbIdRef="b"/>
                        </FunctionArgument>
                        <FunctionArgument symbId="c">
                            <ct:SymbRef symbIdRef="c"/>
                        </FunctionArgument>
                        <FunctionArgument symbId="f">
                            <ct:SymbRef blkIdRef="sm1" symbIdRef="C1model"/>
                        </FunctionArgument>
                    </FunctionCall>
                </Equation>
            </ct:Assign>
        </ErrorModel>
        <ResidualError>
            <ct:SymbRef symbIdRef="eps"></ct:SymbRef>
        </ResidualError>
    </Standard>
</ObservationModel>
\end{xmlcode}



%------------------------------------------------7------------------------------------------------------------------------
\subsection{Combined additive and power error model 2}
\label{model7}
A similar model to the one above but with uncorrelated $\epsilon_{1,ij}$ and $\epsilon_{2,ij}$ leads to a different variance model.
Model:
\begin{eqnarray}
&& y_{ij} = f_{ij} + a\epsilon_{1,ij} + b f_{ij}^c \epsilon_{2,ij}; \quad \epsilon_{1,ij} \sim N(0,1); \quad \epsilon_{2,ij} \sim N(0,1); \quad \mathit{var}(y_{ij}) = a^2 + b^2f_{ij}^{2c} \nonumber
\end{eqnarray}
PharmML - Function definition \& observational model:
\begin{xmlcode}
<!-- type 7 - COMBINED ADDITIVE and POWER MODEL 2 -->
<!-- VERSION 1 with one epsilon-->
<ct:FunctionDefinition symbId="combinedAdditivePowerModel2" symbolType="real">
    <ct:FunctionArgument symbId="a" symbolType="real"/>
    <ct:FunctionArgument symbId="b" symbolType="real"/>
    <ct:FunctionArgument symbId="c" symbolType="real"/>
    <ct:FunctionArgument symbId="f" symbolType="real"/>
    <ct:Definition>
        <Equation xmlns="http://www.pharmml.org/2013/03/Maths">
            <Binop op="root">
                <Binop op="plus">
                    <Binop op="power">
                        <ct:SymbRef symbIdRef="a"/>
                        <ct:Int>2</ct:Int>
                    </Binop>
                    <Binop op="times">
                        <Binop op="power">
                            <ct:SymbRef symbIdRef="b"/>
                            <ct:Real>2</ct:Real>
                        </Binop>
                        <Binop op="power">
                            <ct:SymbRef symbIdRef="f"/>
                            <Binop op="times">
                                <ct:Int>2</ct:Int>
                                <ct:SymbRef symbIdRef="c"/>
                            </Binop>
                        </Binop>
                    </Binop>
                </Binop>
                <Binop op="divide">
                    <ct:Real>1</ct:Real>
                    <ct:Real>2</ct:Real>
                </Binop>
            </Binop>
        </Equation>
    </ct:Definition>
</ct:FunctionDefinition>

<!-- VERSION 1 - using Function Definition - with one epsilon-->
<ObservationModel blkId="om7A">
    <SimpleParameter symbId="a"/>
    <SimpleParameter symbId="b"/>
    <SimpleParameter symbId="c"/>
    <RandomVariable symbId="eps">
        <ct:VariabilityReference>
            <ct:SymbRef blkIdRef="obsErr" symbIdRef="residual"/>
        </ct:VariabilityReference>
        <NormalDistribution xmlns="http://www.uncertml.org/3.0" definition="http://www.uncertml.org/distributions/normal">
            <mean>
                <rVal>0</rVal>
            </mean>
            <stddev>
                <prVal>1</prVal>
            </stddev>
        </NormalDistribution>
    </RandomVariable>
    <Standard symbId="C1">
        <Output>
            <ct:SymbRef blkIdRef="sm1" symbIdRef="C1model"/>
        </Output>
        <ErrorModel>
            <ct:Assign>
                <Equation xmlns="http://www.pharmml.org/2013/03/Maths">
                    <FunctionCall>
                        <ct:SymbRef symbIdRef="combinedAdditivePowerModel2"/>
                        <FunctionArgument symbId="a">
                            <ct:SymbRef symbIdRef="a"/>
                        </FunctionArgument>
                        <FunctionArgument symbId="b">
                            <ct:SymbRef symbIdRef="b"/>
                        </FunctionArgument>
                        <FunctionArgument symbId="c">
                            <ct:SymbRef symbIdRef="c"/>
                        </FunctionArgument>
                        <FunctionArgument symbId="f">
                            <ct:SymbRef blkIdRef="sm1" symbIdRef="C1model"/>
                        </FunctionArgument>
                    </FunctionCall>
                </Equation>
            </ct:Assign>
        </ErrorModel>
        <ResidualError>
            <ct:SymbRef symbIdRef="eps"/>
        </ResidualError>
    </Standard>
</ObservationModel> 


<!-- type 7 - COMBINED ADDITIVE and POWER MODEL 2 -->
<!-- VERSION 2 with two epsilons-->
<ObservationModel blkId="om7B">
    <SimpleParameter symbId="a"/>
    <SimpleParameter symbId="b"/>
    <SimpleParameter symbId="c"/>
    <RandomVariable symbId="eps1">
        <ct:VariabilityReference>
            <ct:SymbRef blkIdRef="obsErr" symbIdRef="residual"/>
        </ct:VariabilityReference>
        <NormalDistribution xmlns="http://www.uncertml.org/3.0" definition="http://www.uncertml.org/distributions/normal">
            <mean>
                <rVal>0</rVal>
            </mean>
            <stddev>
                <prVal>1</prVal>
            </stddev>
        </NormalDistribution>
    </RandomVariable>
    <RandomVariable symbId="eps2">
        <ct:VariabilityReference>
            <ct:SymbRef blkIdRef="obsErr" symbIdRef="residual"/>
        </ct:VariabilityReference>
        <NormalDistribution xmlns="http://www.uncertml.org/3.0" definition="http://www.uncertml.org/distributions/normal">
            <mean>
                <rVal>0</rVal>
            </mean>
            <stddev>
                <prVal>1</prVal>
            </stddev>
        </NormalDistribution>
    </RandomVariable>
    <General symbId="C1">
        <ct:Assign>
            <Equation xmlns="http://www.pharmml.org/2013/03/Maths">
                <Binop op="plus">
                    <Binop op="times">
                        <ct:SymbRef symbIdRef="a"/>
                        <ct:SymbRef symbIdRef="eps1"/>
                    </Binop>
                    <Binop op="times">
                        <ct:SymbRef symbIdRef="b"/>
                        <Binop op="times">
                            <Binop op="power">
                                <ct:SymbRef blkIdRef="sm1" symbIdRef="C1model"/>
                                <ct:SymbRef symbIdRef="c"/>
                            </Binop>
                            <ct:SymbRef symbIdRef="eps2"/>
                        </Binop>
                    </Binop>
                </Binop>
            </Equation>
        </ct:Assign>
    </General>
</ObservationModel> 
\end{xmlcode}





%-------------------------------------------------8-----------------------------------------------------------------------
\subsection{Two (or more) types of measurements error model}
\label{model8}
This model assumes that e.g. the drug concentration, is measured using two different assays. This is expressed by the categorical variable ASY taking values 0 or 1, for the first or second assay respectively.
Model:
\begin{eqnarray}
&& y_{ij} = f_{ij} + \text{ASY}_j\epsilon_{1,ij} + (1-\text{ASY}_j) \epsilon_{2,ij}; \quad \epsilon_{1,ij} \sim N(0,\sigma_1^2); \quad \epsilon_{2,ij} \sim N(0,\sigma_2^2); \nonumber
\end{eqnarray}
Alternative model:
\begin{eqnarray}
&& y_{ij} = \left\{ \begin{array}{rcl}  f_{ij} + \epsilon_{1,ij}  & \mbox{for} & \text{ASY}_j  = 1 \\
f_{ij} + \epsilon_{2,ij}  & \mbox{for} & \text{ASY}_j  = 2 \nonumber
\end{array}\right.
\end{eqnarray}
\begin{eqnarray}
&& \mathit{var}(y_{ij}) = \left\{ \begin{array}{rcl}  \sigma_{1}^2  & \mbox{for} & \text{ASY}_j  = 1 \\
\sigma_{2}^2  & \mbox{for} & \text{ASY}_j  = 2 \nonumber
\end{array}\right.
\end{eqnarray}
PharmML - Function definition \& observational model:
\begin{xmlcode}
<!-- type 8 - TWO TYPES OF MEASUREMENTS MODEL -->
<ObservationModel blkId="om8">
    <SimpleParameter symbId="sigma1"/>
    <SimpleParameter symbId="sigma2"/>
    <SimpleParameter symbId="ASY"/>
    <RandomVariable symbId="eps1">
        <ct:VariabilityReference>
            <ct:SymbRef blkIdRef="obsErr" symbIdRef="residual"/>
        </ct:VariabilityReference>
        <NormalDistribution xmlns="http://www.uncertml.org/3.0" definition="http://www.uncertml.org/distributions/normal">
            <mean>
                <rVal>0</rVal>
            </mean>
            <stddev>
                <var varId="sigma1"/>
            </stddev>
        </NormalDistribution>
    </RandomVariable>
    <RandomVariable symbId="eps2">
        <ct:VariabilityReference>
            <ct:SymbRef blkIdRef="obsErr" symbIdRef="residual"/>
        </ct:VariabilityReference>
        <NormalDistribution xmlns="http://www.uncertml.org/3.0" definition="http://www.uncertml.org/distributions/normal">
            <mean>
                <rVal>0</rVal> 
            </mean>
            <stddev>
                <var varId="sigma2"/>
            </stddev>
        </NormalDistribution>
    </RandomVariable>
    <General symbId="C1">
        <ct:Assign>
            <Equation xmlns="http://www.pharmml.org/2013/03/Maths">
                <Piecewise>
                    <Piece>
                        <Binop op="plus">
                            <ct:SymbRef blkIdRef="sm1" symbIdRef="C1model"/>
                            <ct:SymbRef symbIdRef="eps1"/>
                        </Binop>
                        <Condition>
                            <LogicBinop op="eq">
                                <ct:SymbRef symbIdRef="ASY"/>
                                <ct:Int>1</ct:Int>
                            </LogicBinop>
                        </Condition>
                    </Piece>
                    <Piece>
                        <Binop op="plus">
                            <ct:SymbRef blkIdRef="sm1" symbIdRef="C1model"/>
                            <ct:SymbRef symbIdRef="eps2"/>
                        </Binop>
                        <Condition>
                            <LogicBinop op="eq">
                                <ct:SymbRef symbIdRef="ASY"/>
                                <ct:Int>2</ct:Int>
                            </LogicBinop>
                        </Condition>                                        
                    </Piece>
                </Piecewise>
            </Equation>
        </ct:Assign>
    </General>
</ObservationModel> 
\end{xmlcode}




%---------------------------------------------9---------------------------------------------------------------------------
\subsection{Two (or more) types of observations error model}
\label{model9}
This model assumes that we combine two different types of measurements for modelled e.g. the drug concentration. The first could be the concentration as measured in the blood plasma, the second the concentration measured in urine.
Model:
\begin{eqnarray}
&& y_{ij} = \text{TYP}_{ij} f_{1,ij} + (1-\text{TYP}_{ij}) f_{2,ij} + \text{TYP}_{ij}\epsilon_{1,ij} + (1-\text{TYP}_{ij}) \epsilon_{2,ij}; \quad \epsilon_{1,ij} \sim N(0,\sigma_1^2); \quad \epsilon_{2,ij} \sim N(0,\sigma_2^2);\nonumber
\end{eqnarray}
or
\begin{eqnarray}
&& y_{ij} = \left\{ \begin{array}{rcl}  f_{1,ij} + \epsilon_{1,ij}  & \mbox{for} & \text{TYP}_{ij}  = 1 \\
f_{2,ij} + \epsilon_{2,ij}  & \mbox{for} & \text{TYP}_{ij}  = 2 \nonumber
\end{array}\right\} \quad \text{with} \quad
\mathit{var}(y_{ij}) = \left\{ \begin{array}{rcl}  \sigma_{1}^2  & \mbox{for} & \text{TYP}_j  = 1 \\
\sigma_{2}^2  & \mbox{for} & \text{TYP}_j  = 2 \nonumber
\end{array}\right.
\end{eqnarray}
PharmML - Function definition \& observational model:
\begin{xmlcode}
<!-- type 9 - TWO TYPES OF OBSERVATIONS MODEL -->
<ObservationModel blkId="om9">
    <SimpleParameter symbId="TYP"/>
    <SimpleParameter symbId="sigma1"/>
    <SimpleParameter symbId="sigma2"/>
    <RandomVariable symbId="eps1">
        <ct:VariabilityReference>
            <ct:SymbRef blkIdRef="obsErr" symbIdRef="residual"/>
        </ct:VariabilityReference>
        <NormalDistribution xmlns="http://www.uncertml.org/3.0" definition="http://www.uncertml.org/distributions/normal">
            <mean>
                <rVal>0</rVal>
            </mean>
            <stddev>
                <var varId="sigma1"/>
            </stddev>
        </NormalDistribution>
    </RandomVariable>
    <RandomVariable symbId="eps2">
        <ct:VariabilityReference>
            <ct:SymbRef blkIdRef="obsErr" symbIdRef="residual"/>
        </ct:VariabilityReference>
        <NormalDistribution xmlns="http://www.uncertml.org/3.0" definition="http://www.uncertml.org/distributions/normal">
            <mean>
                <rVal>0</rVal>
            </mean>
            <stddev>
                <var varId="sigma2"/>
            </stddev>
        </NormalDistribution>
    </RandomVariable>
    <General symbId="C1">
        <ct:Assign>
            <Equation xmlns="http://www.pharmml.org/2013/03/Maths">
                <Piecewise>
                    <Piece>
                        <Binop op="plus">
                            <ct:SymbRef blkIdRef="sm1" symbIdRef="C1model"/>
                            <ct:SymbRef symbIdRef="eps1"/>
                        </Binop>
                        <Condition>
                            <LogicBinop op="eq">
                                <ct:SymbRef symbIdRef="TYP"/>
                                <ct:Int>1</ct:Int>
                            </LogicBinop>
                        </Condition>
                    </Piece>
                    <Piece>
                        <Binop op="plus">
                            <ct:SymbRef blkIdRef="sm1" symbIdRef="C2model"/>
                            <ct:SymbRef symbIdRef="eps2"/>
                        </Binop>
                        <Condition>
                            <LogicBinop op="eq">
                                <ct:SymbRef symbIdRef="TYP"/>
                                <ct:Int>2</ct:Int>
                            </LogicBinop>
                        </Condition>
                    </Piece> 
                </Piecewise> 
            </Equation> 
        </ct:Assign> 
    </General> 
</ObservationModel> 
\end{xmlcode}


\newpage
\section{TBS models}

%%%%%%%%%%%%%%%%%%%%%%%%%%%%%%%-1-%%%%%%%%%%%%%%%%%%%%%%%%%%%%%%%%%%%%%%%%%

\subsection{Example 1} 
The following variance model was proposed specifically to get the same error structure for log-transformed data as in case of the combined additive and proportional model for the untransformed data (see
\href{http://www.cognigencorp.com/nonmem/nm/99apr232002.html}{[NMusers] forum} discussion):	
\begin{eqnarray}
&& \log(y_{ij}) = \log(f_{ij}) + \sqrt{a^2 + b^2/f_{ij}^2}\,\epsilon_{ij}, \quad \epsilon_{ij} \sim N(0,1) \quad \text{with} \quad \mathit{var}(y_{ij}) = a^2 + b^2/f_{ij}^2. \nonumber
\end{eqnarray}
PharmML - Function definition \& observational model:
\begin{xmlcode}
<!-- 2.2.1	Exponential error model - proposed by Mats Karlsson -->
<ct:FunctionDefinition symbId="extendedLogG" symbolType="real">
    <ct:FunctionArgument symbId="a" symbolType="real"/>
    <ct:FunctionArgument symbId="b" symbolType="real"/>
    <ct:FunctionArgument symbId="f" symbolType="real"/>
    <ct:Definition>
        <Equation xmlns="http://www.pharmml.org/2013/03/Maths">
            <Binop op="root">
                <ct:Real>2</ct:Real>
                <Binop op="plus">
                    <Binop op="power">
                        <ct:SymbRef symbIdRef="a"/>
                        <ct:Real>2</ct:Real>
                    </Binop>
                    <Binop op="divide">
                        <Binop op="power">
                            <ct:SymbRef symbIdRef="b"/>
                            <ct:Real>2</ct:Real>
                        </Binop>
                        <Binop op="power">
                            <ct:SymbRef symbIdRef="f"/>
                            <ct:Real>2</ct:Real>
                        </Binop>
                    </Binop>
                </Binop>
            </Binop>
        </Equation>
    </ct:Definition>
</ct:FunctionDefinition>

<ObservationModel blkId="om1">
    <SimpleParameter symbId="a"/>
    <SimpleParameter symbId="b"/>
    <RandomVariable symbId="eps">
        <ct:VariabilityReference>
            <ct:SymbRef blkIdRef="obsErr" symbIdRef="residual"/>
        </ct:VariabilityReference>
        <NormalDistribution xmlns="http://www.uncertml.org/3.0" definition="http://www.uncertml.org/distributions/normal">
            <mean>
                <rVal>0</rVal>
            </mean>
            <stddev>
                <prVal>1</prVal>
            </stddev>
        </NormalDistribution>
    </RandomVariable>
    <Standard symbId="C">
        <Transformation>log</Transformation>
        <Output>
            <ct:SymbRef blkIdRef="sm1" symbIdRef="C1model"/>
        </Output>
        <ErrorModel>
            <ct:Assign>
                <Equation xmlns="http://www.pharmml.org/2013/03/Maths">
                    <FunctionCall>
                        <ct:SymbRef symbIdRef="extendedLogG"/>
                        <FunctionArgument symbId="a">
                            <ct:SymbRef symbIdRef="a"/>
                        </FunctionArgument>
                        <FunctionArgument symbId="b">
                            <ct:SymbRef symbIdRef="b"/>
                        </FunctionArgument>
                        <FunctionArgument symbId="f">
                            <ct:SymbRef symbIdRef="C1model"/>
                        </FunctionArgument>
                    </FunctionCall>
                </Equation>
            </ct:Assign>
        </ErrorModel>
        <ResidualError>
            <ct:SymbRef symbIdRef="eps"/>
        </ResidualError>
    </Standard>
</ObservationModel>
\end{xmlcode}


\subsection{Example 2} 
The model:
\begin{eqnarray}
\log(y_{ij}) =  \log(f_{ij}+\theta) + \frac{f_{ij}}{f_{ij}+\theta} \epsilon_{1,ij} + \frac{\theta}{f_{ij}+\theta} \epsilon_{2,ij} 
\quad \text{with} \quad \mathit{var}(y_{ij}) = \frac{f_{ij}^2 + \theta^2}{(f_{ij}+\theta)^2}\nonumber
\end{eqnarray}
for uncorrelated $\epsilon_{1,ij}\sim N(0,1)$ and $\epsilon_{2,ij}\sim N(0,1)$ cannot be encoded in the current version.


\newpage
\section{Models from the Keizer/Karlsson list}

%%%%%%%%%%%%%%%%%%%%%%%%%%%%%%%-1-%%%%%%%%%%%%%%%%%%%%%%%%%%%%%%%%%%%%%%%%%
\subsection{Inter-individual and inter-occasion variability in residual error magnitude}
\label{modelKK_RM1}


%---------------------------------------------------------------------------1/1---------------------------------------------------------------------------------------
\subsubsection{case 1: IIV of the residual error magnitude}
With 'typical' value
\begin{eqnarray}
&& W_{typ,ij} = \sqrt{\theta_1^2 + \theta_2^2 f_{ij}^2} \nonumber
\end{eqnarray}
$W_{ij}$ is the individual log-normally distributed standard deviation of the residual error
\begin{eqnarray}
&& W_{ij} = W_{typ,ij} \exp(\eta_i) \nonumber
\end{eqnarray}
The observation model reads then
\begin{eqnarray}
&& y_{ij} \sim \mathcal{N}(f_{ij},W_{ij}^2): \quad y_{ij} = f_{ij} + W_{ij} \epsilon_{ij}, \quad \epsilon_{ij} \sim \mathcal{N}(0,1)	 \nonumber
\end{eqnarray}
PharmML - Function definition \& observational model:
\begin{xmlcode}
<!-- 3.1.1  case 1: IIV of the residual error magnitude -->
<ObservationModel blkId="om1.1">
    <SimpleParameter symbId="theta1"/>
    <SimpleParameter symbId="theta2"/>
    <RandomVariable symbId="eps">
        <ct:VariabilityReference>
            <ct:SymbRef blkIdRef="obsErr" symbIdRef="residual"/>
        </ct:VariabilityReference>
        <NormalDistribution xmlns="http://www.uncertml.org/3.0" definition="http://www.uncertml.org/distributions/normal">
            <mean>
                <rVal>0</rVal>
            </mean>
            <stddev>
                <prVal>1</prVal>
            </stddev>
        </NormalDistribution>
    </RandomVariable>
    <RandomVariable symbId="eta">
        <ct:VariabilityReference>
            <ct:SymbRef blkIdRef="varModel" symbIdRef="indiv"/>
        </ct:VariabilityReference>
        <NormalDistribution xmlns="http://www.uncertml.org/3.0" definition="http://www.uncertml.org/distributions/normal">
            <mean>
                <rVal>0</rVal>
            </mean>
            <stddev>
                <prVal>1</prVal>
            </stddev>
        </NormalDistribution>
    </RandomVariable>
    <General symbId="C">
        <ct:Assign>
            <Equation xmlns="http://www.pharmml.org/2013/03/Maths">
                <Binop op="plus">
                    <ct:SymbRef blkIdRef="sm1" symbIdRef="C1model"/>
                    <Binop op="times">
                        <Binop op="times">
                            <FunctionCall>
                                <ct:SymbRef symbIdRef="typicalW"/>
                                <FunctionArgument symbId="a">
                                    <ct:SymbRef symbIdRef="theta1"/>
                                </FunctionArgument>
                                <FunctionArgument symbId="b">
                                    <ct:SymbRef symbIdRef="theta2"/>
                                </FunctionArgument>
                                <FunctionArgument symbId="f">
                                    <ct:SymbRef blkIdRef="sm1" symbIdRef="C1model"/>
                                </FunctionArgument>
                            </FunctionCall>
                            <Uniop op="exp">
                                <ct:SymbRef symbIdRef="eta"/>
                            </Uniop>
                        </Binop>
                        <ct:SymbRef symbIdRef="eps"/>
                    </Binop>
                </Binop>
            </Equation>
        </ct:Assign>
    </General>
</ObservationModel>
\end{xmlcode}

%---------------------------------------------------------------------------1/2---------------------------------------------------------------------------------------\
\subsubsection{case 2: IOV of the residual error magnitude}
Here the magnitude of the residual error varies with occasions.
\begin{eqnarray}
&&W_{typ,ij} = \sqrt{\theta_1^2 + \theta_2^2 f_{ij}^2} \nonumber \\
&&W_{ij} = \left\{ \begin{array}{rcl} W_{typ,ij} \exp(\eta_{1,i})  & \mbox{for}  & FLAG  = 1 \\
W_{typ,ij} \exp(\eta_{2,i})    & \mbox{for} & FLAG  = 0 \nonumber
\end{array}\right.
\end{eqnarray}
with $\eta_{1,i} \sim \mathcal{N}(0,\omega^2)$ and $\eta_{2,i} \sim \mathcal{N}(0,\omega^2)$.
The observation model reads then
\begin{eqnarray}
&& y_{ij} \sim \mathcal{N}(f_{ij},W_{ij}^2): \quad y_{ij} = f_{ij} + W_{ij} \epsilon_{ij}, \quad \epsilon_{ij} \sim \mathcal{N}(0,1).	 \nonumber
\end{eqnarray}
PharmML - Function definition \& observational model:
\begin{xmlcode}
<!-- 3.1.2  case 2: IOV of the residual error magnitude -->
<ObservationModel blkId="om1.2">
    <SimpleParameter symbId="omega"/>
    <SimpleParameter symbId="FLAG"/>
    <SimpleParameter symbId="theta1"/>
    <SimpleParameter symbId="theta2"/>
    <RandomVariable symbId="eps">
        <ct:VariabilityReference>
            <ct:SymbRef blkIdRef="obsErr" symbIdRef="residual"/>
        </ct:VariabilityReference>
        <NormalDistribution xmlns="http://www.uncertml.org/3.0" definition="http://www.uncertml.org/distributions/normal">
            <mean>
                <rVal>0</rVal>
            </mean>
            <stddev>
                <prVal>1</prVal>
            </stddev>
        </NormalDistribution>
    </RandomVariable>
    <RandomVariable symbId="eta1">
        <ct:VariabilityReference>
            <ct:SymbRef blkIdRef="varModel" symbIdRef="occasion"/>
        </ct:VariabilityReference>
        <NormalDistribution xmlns="http://www.uncertml.org/3.0" definition="http://www.uncertml.org/distributions/normal">
            <mean>
                <rVal>0</rVal>
            </mean>
            <stddev>
                <var varId="omega"/> 
            </stddev>
        </NormalDistribution>
    </RandomVariable>
    <RandomVariable symbId="eta2">
        <ct:VariabilityReference>
            <ct:SymbRef blkIdRef="varModel" symbIdRef="occasion"/>
        </ct:VariabilityReference>
        <NormalDistribution xmlns="http://www.uncertml.org/3.0" definition="http://www.uncertml.org/distributions/normal">
            <mean>
                <rVal>0</rVal>
            </mean>
            <stddev>
                <var varId="omega"/>
            </stddev>
        </NormalDistribution>
    </RandomVariable>
    <General symbId="C">
        <ct:Assign>
            <Equation xmlns="http://www.pharmml.org/2013/03/Maths">
                <Piecewise>
                    <Piece>                                <!-- FLAG = 1 -->
                        <Binop op="plus">
                            <ct:SymbRef blkIdRef="sm1" symbIdRef="C1model"/>
                            <Binop op="times">
                                <Binop op="times">
                                    <FunctionCall>
                                        <ct:SymbRef symbIdRef="typicalW"/>
                                        <FunctionArgument symbId="a">
                                            <ct:SymbRef symbIdRef="theta1"/>
                                        </FunctionArgument>
                                        <FunctionArgument symbId="b">
                                            <ct:SymbRef symbIdRef="theta2"/>
                                        </FunctionArgument>
                                        <FunctionArgument symbId="f">
                                            <ct:SymbRef blkIdRef="sm1" symbIdRef="C1model"/>
                                        </FunctionArgument>
                                    </FunctionCall>
                                    <Uniop op="exp">
                                        <ct:SymbRef symbIdRef="eta1"/>
                                    </Uniop>
                                </Binop>
                                <ct:SymbRef symbIdRef="eps"/>
                            </Binop>
                        </Binop>
                        <Condition>
                            <LogicBinop op="eq">
                                <ct:SymbRef symbIdRef="FLAG"/>
                                <ct:Int>1</ct:Int>
                            </LogicBinop>
                        </Condition>
                    </Piece>                                <!-- FLAG = 0 -->
                    <Piece>
                        <Binop op="plus">
                            <ct:SymbRef blkIdRef="sm1" symbIdRef="C1model"/>
                            <Binop op="times">
                                <Binop op="times">
                                    <FunctionCall>
                                        <ct:SymbRef symbIdRef="typicalW"/>
                                        <FunctionArgument symbId="a">
                                            <ct:SymbRef symbIdRef="theta1"/>
                                        </FunctionArgument>
                                        <FunctionArgument symbId="b">
                                            <ct:SymbRef symbIdRef="theta2"/>
                                        </FunctionArgument>
                                        <FunctionArgument symbId="f">
                                            <ct:SymbRef blkIdRef="sm1" symbIdRef="C1model"/>
                                        </FunctionArgument>
                                    </FunctionCall>
                                    <Uniop op="exp">
                                        <ct:SymbRef symbIdRef="eta2"/>
                                    </Uniop>
                                </Binop>
                                <ct:SymbRef symbIdRef="eps"/>
                            </Binop>
                        </Binop>
                        <Condition>
                            <LogicBinop op="eq">
                                <ct:SymbRef symbIdRef="FLAG"/>
                                <ct:Int>0</ct:Int>
                            </LogicBinop>
                        </Condition>
                    </Piece>
                </Piecewise>
            </Equation>
        </ct:Assign>
    </General>
</ObservationModel>
\end{xmlcode}


%---------------------------------------------------------------------------1-3---------------------------------------------------------------------------------------
\subsubsection{Alternative representation (not in the K/K list)}
The models above illustrate IIV or IOV of the residual error as applied to the standard deviation function $g$, here $W$.
An alternative would be to assign IIV/IOV to the parameters of the residual error, e.g. in case 1:
\begin{eqnarray}
&& a_i = a_{pop}\, \exp(\eta_{a,ij}); \quad \eta_{a,ij} \sim \mathcal{N}(0,\omega_a^2) \nonumber \\
&& b_i = b_{pop}\, \exp(\eta_{b,ij}); \quad \eta_{b,ij} \sim \mathcal{N}(0,\omega_b^2) \nonumber \\
&& W_{ij} = \sqrt{a_1^2 + b_2^2 f_{ij}^2} \nonumber \\
&& y_{ij} = f_{ij} + W_{ij} \epsilon_{ij}, \quad \epsilon_{ij} \sim \mathcal{N}(0,1)	\nonumber
\end{eqnarray}
PharmML - Function definition \& observational model:
\begin{xmlcode}
<!-- 3.1.3	Alternative representation (not on the K/K list) -->
<ObservationModel blkId="om1.3">
    <!-- a - error parameter with IIV -->
    <SimpleParameter symbId="omega_a"/>
    <SimpleParameter symbId="pop_a"/>
    <RandomVariable symbId="eta_a">
        <ct:VariabilityReference>
            <ct:SymbRef blkIdRef="sm1" symbIdRef="indiv"/>
        </ct:VariabilityReference>
        <NormalDistribution xmlns="http://www.uncertml.org/3.0" definition="http://www.uncertml.org/distributions/normal">
            <mean>
                <rVal>0</rVal>
            </mean>
            <stddev>
                <var varId="omega_a"/>
            </stddev>
        </NormalDistribution>
    </RandomVariable>
    <IndividualParameter symbId="a">
        <GaussianModel>
            <Transformation>log</Transformation>
            <LinearCovariate>
                <PopulationParameter>
                    <ct:Assign>
                        <Equation xmlns="http://www.pharmml.org/2013/03/Maths">
                            <ct:SymbRef symbIdRef="pop_a"/>
                        </Equation>
                    </ct:Assign>
                </PopulationParameter>
            </LinearCovariate>
            <RandomEffects>
                <ct:SymbRef symbIdRef="eta_a"/>
            </RandomEffects>
        </GaussianModel>
    </IndividualParameter>
    <!-- b - error parameter with IIV -->
    <SimpleParameter symbId="omega_b"/>
    <SimpleParameter symbId="pop_b"/>
    <RandomVariable symbId="eta_b">
        <ct:VariabilityReference>
            <ct:SymbRef blkIdRef="sm1" symbIdRef="indiv"/>
        </ct:VariabilityReference>
        <NormalDistribution xmlns="http://www.uncertml.org/3.0" definition="http://www.uncertml.org/distributions/normal">
            <mean>
                <rVal>0</rVal>
            </mean>
            <stddev>
                <var varId="omega_b"/>
            </stddev>
        </NormalDistribution>
    </RandomVariable>
    <IndividualParameter symbId="b">
        <GaussianModel>
            <Transformation>log</Transformation>
            <LinearCovariate>
                <PopulationParameter>
                    <ct:Assign>
                        <Equation xmlns="http://www.pharmml.org/2013/03/Maths">
                            <ct:SymbRef symbIdRef="pop_b"/>
                        </Equation>
                    </ct:Assign>
                </PopulationParameter>
            </LinearCovariate>
            <RandomEffects>
                <ct:SymbRef symbIdRef="eta_b"/>
            </RandomEffects>
        </GaussianModel>
    </IndividualParameter>
    <RandomVariable symbId="eps">
        <ct:VariabilityReference>
            <ct:SymbRef blkIdRef="obsErr" symbIdRef="residual"/>
        </ct:VariabilityReference>
        <NormalDistribution xmlns="http://www.uncertml.org/3.0" definition="http://www.uncertml.org/distributions/normal">
            <mean>
                <rVal>0</rVal> 
            </mean>
            <stddev>
                <prVal>1</prVal> 
            </stddev>
        </NormalDistribution>
    </RandomVariable>
    <Standard symbId="C">
        <Output>
            <ct:SymbRef blkIdRef="sm1" symbIdRef="C1model"/>
        </Output>
        <ErrorModel>
            <ct:Assign>
                <Equation xmlns="http://www.pharmml.org/2013/03/Maths">
                    <FunctionCall>
                        <ct:SymbRef symbIdRef="typicalW"/>
                        <FunctionArgument symbId="a">
                            <ct:SymbRef symbIdRef="a"/>
                        </FunctionArgument>
                        <FunctionArgument symbId="b">
                            <ct:SymbRef symbIdRef="b"/>
                        </FunctionArgument>
                        <FunctionArgument symbId="f">
                            <ct:SymbRef symbIdRef="C1model"/>
                        </FunctionArgument>
                    </FunctionCall>
                </Equation>
            </ct:Assign>
        </ErrorModel>
        <ResidualError>
            <ct:SymbRef symbIdRef="eps"/>
        </ResidualError>
    </Standard>
</ObservationModel>
\end{xmlcode}

%%%%%%%%%%%%%%%%%%%%%%%%%%%%%%%%%-2-%%%%%%%%%%%%%%%%%%%%%%%%%%%%%%%%%%%%%%%
\subsection{Joint residual error between multiple observations}
\label{modelKK_RM2}

%------------------------------------------------------------------------------------------------------------------------
\subsubsection{case 1: Joint  parent/metabolite measurements}
\begin{eqnarray}
&& y_{ij} = \left\{ \begin{array}{rcl}  f_{1,ij} + \epsilon_{1,ij} & \mbox{for}  & \mbox{TYPE}  = 1 \\
f_{2,ij} + \epsilon_{2,ij}    & \mbox{for} & \mbox{TYPE}  = 0  
\end{array}\right. \quad \mbox{with} \quad 
\bigg[ \begin{array}{l}  \epsilon_{1,ij} \\
\epsilon_{2,ij}   \nonumber
\end{array} \bigg]
\in \mathcal{N} 
\Bigg( 0, \bigg[ \begin{array}{ll} \sigma_1^2 & \sigma_{12} \\ \sigma_{12} & \sigma_2^2 \end{array}  \bigg] \Bigg)
\end{eqnarray}
\textbf{Note 1} The original code assumes $\text{IPRED1} = \text{IPRED2} = \text{IPRED}$, which probably was not meant to be as we have two distinct measurements, from the parent drug and its metabolite.
\textbf{Note 2} This model is equivalent to the 'Two (or more) types of observations error model' type, \ref{model9}, with additional correlation between the residual errors.\\
PharmML - Function definition \& observational model:
\begin{xmlcode}
<!-- 3.2.1	case 1: Joint parent/metabolite measurements -->
<ObservationModel blkId="om1.4">
    <SimpleParameter symbId="TYPE"/>
    <SimpleParameter symbId="sigma1"/>
    <SimpleParameter symbId="sigma2"/>
    <SimpleParameter symbId="rho_eps1_eps2"/>
    <RandomVariable symbId="eps1">
        <ct:VariabilityReference>
            <ct:SymbRef blkIdRef="obsErr" symbIdRef="residual"/>
        </ct:VariabilityReference>
        <NormalDistribution xmlns="http://www.uncertml.org/3.0" definition="http://www.uncertml.org/distributions/normal">
            <mean>
                <rVal>0</rVal>
            </mean>
            <stddev>
                <var varId="sigma1"/> 
            </stddev>
        </NormalDistribution>
    </RandomVariable>
    <RandomVariable symbId="eps2">
        <ct:VariabilityReference>
            <ct:SymbRef blkIdRef="obsErr" symbIdRef="residual"/>
        </ct:VariabilityReference>
        <NormalDistribution xmlns="http://www.uncertml.org/3.0" definition="http://www.uncertml.org/distributions/normal">
            <mean>
                <rVal>0</rVal>
            </mean>
            <stddev>
                <var varId="sigma2"/> 
            </stddev>
        </NormalDistribution>
    </RandomVariable>
    <!-- CORRELATION OF EPSILONS -->
    <Correlation>
        <ct:VariabilityReference>
            <ct:SymbRef blkIdRef="obsErr" symbIdRef="residual"/>
        </ct:VariabilityReference>
        <RandomVariable1>
            <ct:SymbRef symbIdRef="eps1"/>
        </RandomVariable1>
        <RandomVariable2>
            <ct:SymbRef symbIdRef="eps2"/>
        </RandomVariable2>
        <CorrelationCoefficient>
            <ct:SymbRef symbIdRef="rho_eps1_eps2"/>
        </CorrelationCoefficient>
    </Correlation>
    <General symbId="C">
        <ct:Assign>
            <Equation xmlns="http://www.pharmml.org/2013/03/Maths">
                <Piecewise>
                    <Piece>
                        <Binop op="plus">
                            <ct:SymbRef blkIdRef="sm1" symbIdRef="C1model"/>
                            <ct:SymbRef symbIdRef="eps1"/>
                        </Binop>
                        <Condition>
                            <LogicBinop op="eq">
                                <ct:SymbRef symbIdRef="TYPE"/>
                                <ct:Int>1</ct:Int>
                            </LogicBinop>
                        </Condition>
                    </Piece>
                    <Piece>
                        <Binop op="plus">
                            <ct:SymbRef blkIdRef="sm1" symbIdRef="C2model"/>
                            <ct:SymbRef symbIdRef="eps2"/>
                        </Binop>
                        <Condition>
                            <LogicBinop op="eq">
                                <ct:SymbRef symbIdRef="TYPE"/>
                                <ct:Int>0</ct:Int>
                            </LogicBinop>
                        </Condition>
                    </Piece>
                </Piecewise>
            </Equation>
        </ct:Assign>
    </General>
</ObservationModel>
\end{xmlcode}


%------------------------------------------------------------------------------------------------------------------------
\subsubsection{case 2: Replicate observations}
\begin{eqnarray}
&& y_{ij} = \left\{ \begin{array}{rcl}  f_{ij} + \epsilon_{1,ij} + \epsilon_{2,ij} & \mbox{for}  & \mbox{TYPE}  = 2 \\
f_{ij} + \epsilon_{1,ij}  + \epsilon_{3,ij}    & \mbox{for} & else  
\end{array}\right. \quad \mbox{with} \quad \epsilon_{1,ij} \sim \mathcal{N}(0,\sigma_1^2), \quad \epsilon_{2,ij},\;\epsilon_{3,ij} \sim \mathcal{N}(0,\sigma^2) \nonumber \\
&& \mbox{and} \quad g = \sqrt{\sigma_1^2 + \sigma^2} \nonumber
\end{eqnarray}
PharmML - Function definition \& observational model:
\begin{xmlcode}
<!-- 3.2.2	case 2: Replicate observations -->
<ObservationModel blkId="om1.5">
    <SimpleParameter symbId="TYPE"/>
    <SimpleParameter symbId="sigma1"/>
    <SimpleParameter symbId="sigma2"/>
    <SimpleParameter symbId="sigma3"/>
    <RandomVariable symbId="eps1">
        <ct:VariabilityReference>
            <ct:SymbRef blkIdRef="obsErr" symbIdRef="residual"/>
        </ct:VariabilityReference>
        <NormalDistribution xmlns="http://www.uncertml.org/3.0" definition="http://www.uncertml.org/distributions/normal">
            <mean>
                <rVal>0</rVal>
            </mean>
            <stddev>
                <var varId="sigma1"/> 
            </stddev>
        </NormalDistribution>
    </RandomVariable>
    <RandomVariable symbId="eps2">
        <ct:VariabilityReference>
            <ct:SymbRef blkIdRef="obsErr" symbIdRef="residual"/>
        </ct:VariabilityReference>
        <NormalDistribution xmlns="http://www.uncertml.org/3.0" definition="http://www.uncertml.org/distributions/normal">
            <mean>
                <rVal>0</rVal>
            </mean>
            <stddev>
                <var varId="sigma2"/> 
            </stddev>
        </NormalDistribution>
    </RandomVariable>
    <RandomVariable symbId="eps3">
        <ct:VariabilityReference>
            <ct:SymbRef blkIdRef="obsErr" symbIdRef="residual"/>
        </ct:VariabilityReference>
        <NormalDistribution xmlns="http://www.uncertml.org/3.0" definition="http://www.uncertml.org/distributions/normal">
            <mean>
                <rVal>0</rVal>
            </mean>
            <stddev>
                <var varId="sigma3"/> 
            </stddev>
        </NormalDistribution>
    </RandomVariable>
    <General symbId="C">
        <ct:Assign>
            <Equation xmlns="http://www.pharmml.org/2013/03/Maths">
                <Piecewise>
                    <Piece>
                        <Binop op="plus">
                            <ct:SymbRef symbIdRef="C1model"/>
                            <Binop op="plus">
                                <ct:SymbRef symbIdRef="eps1"/>
                                <ct:SymbRef symbIdRef="eps2"/>
                            </Binop>
                        </Binop>
                        <Condition>
                            <LogicBinop op="eq">
                                <ct:SymbRef symbIdRef="TYPE"/>
                                <ct:Int>2</ct:Int>
                            </LogicBinop>
                        </Condition>
                    </Piece>
                    <Piece>
                        <Binop op="plus">
                            <ct:SymbRef symbIdRef="C1model"/>
                            <Binop op="plus">
                                <ct:SymbRef symbIdRef="eps1"/>
                                <ct:SymbRef symbIdRef="eps3"/>
                            </Binop>
                        </Binop>
                        <Condition>
                            <Otherwise/>
                        </Condition>
                    </Piece>
                </Piecewise>
            </Equation>
        </ct:Assign>
    </General>
</ObservationModel>
\end{xmlcode}


%%%%%%%%%%%%%%%%%%%%%%%%%%%%%%%%%%%-3-%%%%%%%%%%%%%%%%%%%%%%%%%%%%%%%%%%%%%
\subsection{Combination of L2 data items with likelihood based inclusion of BLQ data}
\label{modelKK_RM3}
Explicit likelihood implementation will be supported in a future PharmML release.

%%%%%%%%%%%%%%%%%%%%%%%%%%%%%%%%%%%-4-%%%%%%%%%%%%%%%%%%%%%%%%%%%%%%%%%%%%%
\subsection{Serial correlation}
\label{modelKK_RM4}
Residual error autocorrelation will be supported in a future PharmML release.

%%%%%%%%%%%%%%%%%%%%%%%%%%%%%%%%%%%-5-%%%%%%%%%%%%%%%%%%%%%%%%%%%%%%%%%%%%%
\subsection{Residual error magnitude varying with covariates and time}
\label{modelKK_RM5}
%------------------------------------------------------------------------------------------------------------------------
\subsubsection{case 1}
\begin{eqnarray}
&&KA = \theta_4 \nonumber \\
&&MAT = 1/KA \nonumber \\
&&FACT = \left\{ \begin{array}{rcl}  1 & \mbox{for}  & time > MAT \\
\theta_3  & \mbox{for} & else  \nonumber
\end{array}\right.
\end{eqnarray}
\begin{eqnarray}
&&g = \sqrt{\theta_2^2 + \theta_1^2 f^2} \times FACT \nonumber
\end{eqnarray}
PharmML - Function definition \& observational model:
VERSION 1
\begin{xmlcode}
<!-- FUNCTION FOR 'TYPICAL' W = sqrt(a^2 + b^2 f^2) -->
<ct:FunctionDefinition symbId="typicalW" symbolType="real">
    <ct:FunctionArgument symbId="a" symbolType="real"/>
    <ct:FunctionArgument symbId="b" symbolType="real"/>
    <ct:FunctionArgument symbId="f" symbolType="real"/>
    <ct:Definition> 
        <Equation xmlns="http://www.pharmml.org/2013/03/Maths">
            <Binop op="root">
                <Binop op="plus">
                    <Binop op="power">
                        <ct:SymbRef symbIdRef="a"/>
                        <ct:Int>2</ct:Int>
                    </Binop>
                    <Binop op="times">
                        <Binop op="power">
                            <ct:SymbRef symbIdRef="b"/>
                            <ct:Int>2</ct:Int>
                        </Binop>
                        <Binop op="power">
                            <ct:SymbRef symbIdRef="f"/>
                            <ct:Int>2</ct:Int>
                        </Binop>
                    </Binop>
                </Binop>
                <ct:Real>2</ct:Real>
            </Binop>
        </Equation>
    </ct:Definition>
</ct:FunctionDefinition>


<!-- 3.5 Residual error magnitude varying with covariates and time - 3.5.1 - case 1 -->
<!-- VERSION 1 -->
<ObservationModel blkId="om1.6">
    <!-- theta1,2,3,4 -->
    <SimpleParameter symbId="theta1"/>
    <SimpleParameter symbId="theta2"/>
    <SimpleParameter symbId="theta3"/>
    <SimpleParameter symbId="theta4"/>
    <!-- KA -->
    <SimpleParameter symbId="KA">
        <ct:Assign>
            <Equation xmlns="http://www.pharmml.org/2013/03/Maths">
                <ct:SymbRef symbIdRef="theta4"/>
            </Equation>
        </ct:Assign>
    </SimpleParameter>
    <!-- MAT -->
    <SimpleParameter symbId="MAT">
        <ct:Assign>
            <Equation xmlns="http://www.pharmml.org/2013/03/Maths">
                <Binop op="divide">
                    <ct:Real>1</ct:Real>
                    <ct:SymbRef symbIdRef="KA"/>
                </Binop>
            </Equation>
        </ct:Assign>
    </SimpleParameter>
    <RandomVariable symbId="eps">
        <ct:VariabilityReference>
            <ct:SymbRef blkIdRef="obsErr" symbIdRef="residual"/>
        </ct:VariabilityReference>
        <NormalDistribution xmlns="http://www.uncertml.org/3.0" definition="http://www.uncertml.org/distributions/normal">
            <mean>
                <rVal>0</rVal>
            </mean>
            <stddev>
                <prVal>1</prVal>
            </stddev>
        </NormalDistribution>
    </RandomVariable>
    <General symbId="C">
        <ct:Assign>
            <Equation xmlns="http://www.pharmml.org/2013/03/Maths">
                <Piecewise>
                    <Piece>
                        <Binop op="plus">
                            <ct:SymbRef symbIdRef="C1model"/>
                            <Binop op="times">
                                <FunctionCall>			<!-- typicalW function call -->
                                    <ct:SymbRef symbIdRef="typicalW"/>
                                    <FunctionArgument symbId="a">
                                        <ct:SymbRef symbIdRef="theta1"/>
                                    </FunctionArgument>
                                    <FunctionArgument symbId="b">
                                        <ct:SymbRef symbIdRef="theta2"/>
                                    </FunctionArgument>
                                    <FunctionArgument symbId="f">
                                        <ct:SymbRef symbIdRef="C1model"/>
                                    </FunctionArgument>
                                </FunctionCall>
                                <ct:SymbRef symbIdRef="eps"/>
                            </Binop>
                        </Binop>
                        <Condition>
                            <LogicBinop op="gt">
                                <ct:SymbRef symbIdRef="time"/>
                                <ct:SymbRef symbIdRef="MAT"/> 
                            </LogicBinop>
                        </Condition>
                    </Piece>
                    <Piece>
                        <Binop op="plus">
                            <ct:SymbRef symbIdRef="C1model"/>
                            <Binop op="times">
                                <Binop op="times">
                                    <FunctionCall>			<!-- typicalW function call -->
                                        <ct:SymbRef symbIdRef="typicalW"/>
                                        <FunctionArgument symbId="a">
                                            <ct:SymbRef symbIdRef="theta1"/>
                                        </FunctionArgument>
                                        <FunctionArgument symbId="b">
                                            <ct:SymbRef symbIdRef="theta2"/>
                                        </FunctionArgument>
                                        <FunctionArgument symbId="f">
                                            <ct:SymbRef symbIdRef="C1model"/>
                                        </FunctionArgument>
                                    </FunctionCall>
                                    <ct:SymbRef symbIdRef="theta3"/>
                                </Binop>                                            
                                <ct:SymbRef symbIdRef="eps"/>
                            </Binop>
                        </Binop>
                        <Condition>
                            <Otherwise/>
                        </Condition>
                    </Piece>
                </Piecewise>
            </Equation>
        </ct:Assign>
    </General>
</ObservationModel>
\end{xmlcode}

VERSION 2
\begin{xmlcode}
<!-- FUNCTION FACT for 3.5.1. case 2 -->
<ct:FunctionDefinition symbId="FACT" symbolType="real">
    <ct:FunctionArgument symbId="firstValue" symbolType="real"/>
    <ct:FunctionArgument symbId="secondValue" symbolType="real"/>
    <ct:FunctionArgument symbId="firstCond" symbolType="real"/>
    <ct:FunctionArgument symbId="IDV" symbolType="real"/>
    <ct:Definition>
        <Equation xmlns="http://www.pharmml.org/2013/03/Maths">
            <Piecewise>
                <Piece>
                    <ct:Real>1</ct:Real>
                    <Condition>
                        <LogicBinop op="gt">
                            <ct:SymbRef symbIdRef="IDV"/>
                            <ct:SymbRef symbIdRef="firstCond"/>
                        </LogicBinop>
                    </Condition>
                </Piece>
                <Piece>
                    <ct:SymbRef symbIdRef="secondValue"/>
                    <Condition>
                        <Otherwise/>
                    </Condition>
                </Piece>
            </Piecewise>
        </Equation>
    </ct:Definition>
</ct:FunctionDefinition>

<!-- FUNCTION FOR 'TYPICAL' W = sqrt(a^2 + b^2 f^2) -->
<!-- see definition above -->

<!-- VERSION 2 -->
<ObservationModel blkId="om1.7">
    <!-- theta1,2,3,4 -->
    <SimpleParameter symbId="theta1"/>
    <SimpleParameter symbId="theta2"/>
    <SimpleParameter symbId="theta3"/>
    <SimpleParameter symbId="theta4"/>
    <!-- KA -->
    <SimpleParameter symbId="KA">
        <ct:Assign>
            <Equation xmlns="http://www.pharmml.org/2013/03/Maths">
                <ct:SymbRef symbIdRef="theta4"/>
            </Equation>
        </ct:Assign>
    </SimpleParameter>
    <!-- MAT -->
    <SimpleParameter symbId="MAT">
        <ct:Assign>
            <Equation xmlns="http://www.pharmml.org/2013/03/Maths">
                <Binop op="divide">
                    <ct:Real>1</ct:Real>
                    <ct:SymbRef symbIdRef="KA"/>
                </Binop>
            </Equation>
        </ct:Assign>
    </SimpleParameter>
    <RandomVariable symbId="eps">
        <ct:VariabilityReference>
            <ct:SymbRef blkIdRef="obsErr" symbIdRef="residual"/>
        </ct:VariabilityReference>
        <NormalDistribution xmlns="http://www.uncertml.org/3.0" definition="">
            <mean>
                <rVal>0</rVal>
            </mean>
            <stddev>
                <prVal>1</prVal>
            </stddev>
        </NormalDistribution>
    </RandomVariable>
    <Standard symbId="C">
        <Output>
            <ct:SymbRef blkIdRef="sm1" symbIdRef="C1model"/>
        </Output>
        <ErrorModel>
            <ct:Assign>
                <Equation xmlns="http://www.pharmml.org/2013/03/Maths">
                    <Binop op="times">
                        <FunctionCall>				<!-- typicalW function call -->
                            <ct:SymbRef symbIdRef="typicalW"/>
                            <FunctionArgument symbId="a">
                                <ct:SymbRef symbIdRef="theta1"/>
                            </FunctionArgument>
                            <FunctionArgument symbId="b">
                                <ct:SymbRef symbIdRef="theta2"/>
                            </FunctionArgument>
                            <FunctionArgument symbId="f">
                                <ct:SymbRef symbIdRef="C1model"/>
                            </FunctionArgument>
                        </FunctionCall>
                        <FunctionCall>				<!-- FACT function call -->
                            <ct:SymbRef symbIdRef="FACT"/>
                            <FunctionArgument symbId="firstValue">
                                <ct:Real>1</ct:Real>
                            </FunctionArgument>
                            <FunctionArgument symbId="secondValue">
                                <ct:SymbRef symbIdRef="theta3"/>
                            </FunctionArgument>
                            <FunctionArgument symbId="firstCond">
                                <ct:SymbRef symbIdRef="MAT"/>
                            </FunctionArgument>
                        </FunctionCall>
                    </Binop>
                </Equation>
            </ct:Assign>
        </ErrorModel>
        <ResidualError>
            <ct:SymbRef symbIdRef="eps"/>
        </ResidualError>
    </Standard>
</ObservationModel>
\end{xmlcode}



%------------------------------------------------------------------------------------------------------------------------
\subsubsection{case 2}
\begin{eqnarray}
&&TT = \theta_5 \nonumber \\
&&FACT = \left\{ \begin{array}{rcl}  1 & \mbox{for}  & time > TT \\
\theta_3  & \mbox{for} & else  \nonumber
\end{array}\right.
\end{eqnarray}
\begin{eqnarray}
&&g = \sqrt{\theta_2^2 + \theta_1^2 f^2} \times FACT \nonumber
\end{eqnarray}
PharmML - Function definition \& observational model:
\begin{xmlcode}
<!-- FUNCTION FACT for 3.5.1. case 2 -->
<!-- see definition above  -->

<!-- FUNCTION FOR 'TYPICAL' W = sqrt(a^2 + b^2 f^2) -->
<!-- see definition above  -->


<!-- 3.5 Residual error magnitude varying with covariates and time - 3.5.1 - case 2 -->
<ObservationModel blkId="om1.8">
    <!-- theta1,2,3 & 5 -->
    <SimpleParameter symbId="theta1"/>
    <SimpleParameter symbId="theta2"/>
    <SimpleParameter symbId="theta3"/>
    <SimpleParameter symbId="theta5"/>
    <!-- TT -->
    <SimpleParameter symbId="TT">
        <ct:Assign>
            <Equation xmlns="http://www.pharmml.org/2013/03/Maths">
                <ct:SymbRef symbIdRef="theta5"/>
            </Equation>
        </ct:Assign>
    </SimpleParameter>
    <RandomVariable symbId="eps">
        <ct:VariabilityReference>
            <ct:SymbRef blkIdRef="obsErr" symbIdRef="residual"/>
        </ct:VariabilityReference>
        <NormalDistribution xmlns="http://www.uncertml.org/3.0" definition="http://www.uncertml.org/distributions/normal">
            <mean>
                <rVal>0</rVal>
            </mean>
            <stddev>
                <prVal>1</prVal>
            </stddev>
        </NormalDistribution>
    </RandomVariable>
    <Standard symbId="C">
        <Output>
            <ct:SymbRef symbIdRef="C1model"/>
        </Output>
        <ErrorModel>
            <ct:Assign>
                <Equation xmlns="http://www.pharmml.org/2013/03/Maths">
                    <Binop op="times">
                        <FunctionCall>
                            <ct:SymbRef symbIdRef="typicalW"/>
                            <FunctionArgument symbId="a">
                                <ct:SymbRef symbIdRef="theta1"/>
                            </FunctionArgument>
                            <FunctionArgument symbId="b">
                                <ct:SymbRef symbIdRef="theta2"/>
                            </FunctionArgument>
                            <FunctionArgument symbId="f">
                                <ct:SymbRef symbIdRef="C1model"/>
                            </FunctionArgument>
                        </FunctionCall>
                        <FunctionCall>
                            <ct:SymbRef symbIdRef="FACT"/>
                            <FunctionArgument symbId="firstValue">
                                <ct:Real>1</ct:Real>
                            </FunctionArgument>
                            <FunctionArgument symbId="secondValue">
                                <ct:SymbRef symbIdRef="theta3"/>
                            </FunctionArgument>
                            <FunctionArgument symbId="firstCond">
                                <ct:SymbRef symbIdRef="TT"/>
                            </FunctionArgument>
                        </FunctionCall>
                    </Binop>
                </Equation>
            </ct:Assign>
        </ErrorModel>
        <ResidualError>
            <ct:SymbRef symbIdRef="dsds"/>
        </ResidualError>
    </Standard>
</ObservationModel>
\end{xmlcode}


%---------------------------------------------10---------------------------------------------------------------------------
\subsubsection{case 3 (not in the K/K list)}
This model example, provided by Roberto, assumes a dependence between $g$ and both the regression variable, TIME, and a covariate, here \textit{subject} denoted as ID.
Model: 
\begin{eqnarray}
y_{ij} = \left\{ \begin{array}{rcl}  f_{ij} + \epsilon_{1,ij}  & \mbox{for} & \text{TIME $\le$ 10 \&\& ID == 1} \\
f_{ij} + \epsilon_{2,ij}  & \mbox{for} & \text{TIME $>$ 10 \&\& ID == 1} \nonumber \\
\cdots \nonumber
\end{array}\right\} \quad \text{with} \quad
\mathit{var}(y_{ij}) = \left\{ \begin{array}{rcl}  \sigma_{1}^2  & \mbox{for} & \text{TIME $\le$ 10 \&\& ID == 1} \\
\sigma_{2}^2  & \mbox{for} & \text{TIME $>$ 10 \&\& ID == 1}  \nonumber \\
\dots && \nonumber
\end{array}\right.
\end{eqnarray}
for $\epsilon_{1,ij} \sim N(0,\sigma_1^2),\, \epsilon_{2,ij} \sim N(0,\sigma_2^2), \cdots$
PharmML - Function definition \& observational model:
\begin{xmlcode}
<!-- 3.5 Residual error magnitude varying with covariates and time - 3.5.1 - case 3  (not in the K/K list) -->
<ObservationModel blkId="om1.9">
    <SimpleParameter symbId="TIME"/>
    <SimpleParameter symbId="ID"/>
    <SimpleParameter symbId="sigma1"/>
    <SimpleParameter symbId="sigma2"/>
    <RandomVariable symbId="eps1">
        <ct:VariabilityReference>
            <ct:SymbRef blkIdRef="obsErr" symbIdRef="residual"/>
        </ct:VariabilityReference>
        <NormalDistribution xmlns="http://www.uncertml.org/3.0" definition="http://www.uncertml.org/distributions/normal">
            <mean>
                <rVal>0</rVal>
            </mean>
            <stddev>
                <var varId="sigma1"/>
            </stddev>
        </NormalDistribution>
    </RandomVariable>
    <RandomVariable symbId="eps2">
        <ct:VariabilityReference>
            <ct:SymbRef blkIdRef="obsErr" symbIdRef="residual"/>
        </ct:VariabilityReference>
        <NormalDistribution xmlns="http://www.uncertml.org/3.0" definition="http://www.uncertml.org/distributions/normal">
            <mean>
                <rVal>0</rVal>
            </mean>
            <stddev>
                <var varId="sigma2"/>
            </stddev>
        </NormalDistribution>
    </RandomVariable>
    <General symbId="C">
        <ct:Assign>
            <Equation xmlns="http://www.pharmml.org/2013/03/Maths">
                <Piecewise>
                    <Piece>
                        <Binop op="plus">
                            <ct:SymbRef symbIdRef="C1model"/>
                            <ct:SymbRef symbIdRef="eps1"/>
                        </Binop>
                        <Condition>
                            <LogicBinop op="and">
                                <LogicBinop op="leq">
                                    <ct:SymbRef symbIdRef="TIME"/>
                                    <ct:Real>10</ct:Real>
                                </LogicBinop>
                                <LogicBinop op="eq">
                                    <ct:SymbRef symbIdRef="ID"/>
                                    <ct:Real>1</ct:Real>
                                </LogicBinop>
                            </LogicBinop>
                        </Condition>
                    </Piece>
                    <Piece>
                        <Binop op="plus">
                            <ct:SymbRef symbIdRef="C1model"/>
                            <ct:SymbRef symbIdRef="eps2"/>
                        </Binop>
                        <Condition>
                            <LogicBinop op="and">
                                <LogicBinop op="gt">
                                    <ct:SymbRef symbIdRef="TIME"/>
                                    <ct:Real>10</ct:Real>
                                </LogicBinop>
                                <LogicBinop op="eq">
                                    <ct:SymbRef symbIdRef="ID"/>
                                    <ct:Real>1</ct:Real>
                                </LogicBinop>
                            </LogicBinop>
                        </Condition>
                    </Piece>
                </Piecewise>
            </Equation>
        </ct:Assign>
    </General>
</ObservationModel>
\end{xmlcode}


%%%%%%%%%%%%%%%%%%%%%%%%%%%%%%%%%%-6-%%%%%%%%%%%%%%%%%%%%%%%%%%%%%%%%%%%%%%
\subsection{Residual error magnitude varying with derivatives of functions w.r.t. to time and parameters}
\label{modelKK_RM6}

Still to do -- for a future release.



%%%%%%%%%%%%%%%%%%%%%%%%%%%%%%%%%%-7-%%%%%%%%%%%%%%%%%%%%%%%%%%%%%%%%%%%%%%
\subsection{Mixture model for residual error}
\label{modelKK_RM7}

Explicit likelihood implementation will be supported in a future PharmML release.



%%%%%%%%%%%%%%%%%%%%%%%%%%%%%%%%%%%-8-%%%%%%%%%%%%%%%%%%%%%%%%%%%%%%%%%%%%%
\subsection{Prior on residual error magnitude}
\label{modelKK_RM8}

Priors will be supported in a future PharmML release.


%%%%%%%%%%%%%%%%%%%%%%%%%%%%%%%%%%-9-%%%%%%%%%%%%%%%%%%%%%%%%%%%%%%%%%%%%%%
\subsection{Flexible errors-in-variables models}
\label{modelKK_RM9}

This is not part of the residual error up to our knowledge. It is a specific type of a variability model for the measurement time points.


%%%%%%%%%%%%%%%%%%%%%%%%%%%%%%%%%%-10-%%%%%%%%%%%%%%%%%%%%%%%%%%%%%%%%%%%%%%
\subsection{Transformation of residual error variables}
\label{modelKK_RM10}

\textbf{Transform-both-sides} models are covered in the standard model types.




\newpage
\appendix

\section{Model template}
... with placeholders for 'FunctionDef', 'ObservationalModel', 'TrialDesign' and 'ModellingSteps'

\begin{xmlcode}
<?xml version="1.0" encoding="UTF-8"?>
<PharmML xmlns="http://www.pharmml.org/2013/03/PharmML"
    xmlns:xsi="http://www.w3.org/2001/XMLSchema-instance"
    xsi:schemaLocation="http://www.pharmml.org/2013/03/PharmML http://www.pharmml.org/2013/03/PharmML"
    xmlns:ct="http://www.pharmml.org/2013/03/CommonTypes"
    writtenVersion="0.1">
    <ct:Name>Residual error models - multiple types</ct:Name>
    <ct:Description>based on document 'Residual error models and Keizer/Karlsson list'</ct:Description>

    <!-- intependent variable -->
    <IndependentVariable symbId="time"/>

    <!-- FUNCTION DEFINITION -->
    <!-- placeholder for residual error definition, e.g. additive error model:
        <ct:FunctionDefinition symbId="additiveErrorModel" symbolType="real">
            ...
        </ct:FunctionDefinition>
    -->

    <!-- MODEL DEFINITION -->
    <ModelDefinition xmlns="http://www.pharmml.org/2013/03/ModelDefinition">
        
        <!-- VARIABILITY MODEL -->
        <VariabilityModel blkId="model">
            <Level symbId="indiv">
                <ct:Name>Subject Level</ct:Name>
            </Level>
        </VariabilityModel>
        
        <VariabilityModel blkId="obsErr">
            <Level symbId="residual">
                <ct:Name>Residual Error</ct:Name>
                <ParentLevel>
                    <ct:SymbRef blkIdRef="model" symbIdRef="indiv"/>
                </ParentLevel>
            </Level>
        </VariabilityModel>
        
        <!-- COVARIATE MODEL -->
        <CovariateModel blkId="cm1">
            <Covariate symbId="W">
                <Continuous>
                    <Transformation>
                        <Equation xmlns="http://www.pharmml.org/2013/03/Maths">
                            <Binop op="divide">
                                <ct:SymbRef symbIdRef="W"/>
                                <ct:Real>70</ct:Real>
                            </Binop>
                        </Equation>
                    </Transformation>
                </Continuous>
            </Covariate>
        </CovariateModel>
        
        <!-- PARAMETER MODEL -->
        <ParameterModel blkId="pm1">
            <SimpleParameter symbId="omega_V"/>
            <SimpleParameter symbId="omega_V2"/>
            <SimpleParameter symbId="omega_k"/>
            <SimpleParameter symbId="omega_k2"/>
            <SimpleParameter symbId="pop_V"/>
            <SimpleParameter symbId="pop_V2"/>
            <SimpleParameter symbId="pop_k"/>
            <SimpleParameter symbId="pop_k2"/>
            <!-- V -->
            <RandomVariable symbId="eta_V">
                <ct:VariabilityReference>
                    <ct:SymbRef symbIdRef="indiv"/>
                </ct:VariabilityReference>
                <NormalDistribution xmlns="http://www.uncertml.org/3.0" definition="http://www.uncertml.org/distributions/normal">
                    <mean>
                        <rVal>0</rVal>
                    </mean>
                    <stddev>
                        <var varId="omega_V"/>
                    </stddev>
                </NormalDistribution>
            </RandomVariable>
            <IndividualParameter symbId="V">
                <GaussianModel>
                    <Transformation>log</Transformation>
                    <LinearCovariate>
                        <PopulationParameter>
                            <ct:Assign>
                                <ct:SymbRef symbIdRef="pop_V"/>
                            </ct:Assign>
                        </PopulationParameter>
                        <Covariate>
                            <ct:SymbRef blkIdRef="cm1" symbIdRef="W"/>
                            <FixedEffect>
                                <ct:SymbRef symbIdRef="beta_V"/>
                            </FixedEffect>
                        </Covariate>
                    </LinearCovariate>
                    <RandomEffects>
                        <ct:SymbRef symbIdRef="eta_V"/>
                    </RandomEffects>
                </GaussianModel>
            </IndividualParameter>
            <!-- V2 -->
	    <!-- SNIP -->
            <!-- k -->
            <RandomVariable symbId="eta_k">
                <ct:VariabilityReference>
                    <ct:SymbRef symbIdRef="indiv"/>
                </ct:VariabilityReference>
                <NormalDistribution xmlns="http://www.uncertml.org/3.0" definition="http://www.uncertml.org/distributions/normal">
                    <mean>
                        <rVal>0</rVal>
                    </mean>
                    <stddev>
                        <var varId="omega_k"/>
                    </stddev>
                </NormalDistribution>
            </RandomVariable>
            <IndividualParameter symbId="k">
                <GaussianModel>
                    <Transformation>log</Transformation>
                    <LinearCovariate>
                        <PopulationParameter>
                            <ct:Assign>
                                <ct:SymbRef symbIdRef="pop_k"/>
                            </ct:Assign>
                        </PopulationParameter>
                    </LinearCovariate>
                    <RandomEffects>
                        <ct:SymbRef symbIdRef="eta_k"/>
                    </RandomEffects>
                </GaussianModel>
            </IndividualParameter>
            <!-- k2 -->
	    <!-- SNIP -->
        </ParameterModel>
        
        <!-- STRUCTURAL MODEL -->
        <StructuralModel blkId="sm1">
            <ct:Variable symbolType="real" symbId="tD"/>
            <ct:Variable symbolType="real" symbId="D"/>
            <ct:Variable symbolType="real" symbId="tD2"/>
            <ct:Variable symbolType="real" symbId="D2"/>
            
            <!-- C1model variable -->
            <ct:Variable symbolType="real" symbId="C1model">
                <ct:Assign>
                    <Equation xmlns="http://www.pharmml.org/2013/03/Maths">
                        <Binop op="times">
                            <Binop op="divide">
                                <ct:SymbRef symbIdRef="D"/>
                                <ct:SymbRef blkIdRef="pm1" symbIdRef="V"/>
                            </Binop>
                            <Uniop op="exp">
                                <Binop op="times">
                                    <Uniop op="minus">
                                        <ct:SymbRef blkIdRef="pm1" symbIdRef="k"/>
                                    </Uniop>
                                    <Binop op="minus">
                                        <ct:SymbRef symbIdRef="time"/>
                                        <ct:SymbRef symbIdRef="tD"/>
                                    </Binop>
                                </Binop>
                            </Uniop>
                        </Binop>
                    </Equation>
                </ct:Assign>
            </ct:Variable>
            
            <!-- C2model variable -->
	    <!-- SNIP -->
	    
            <!-- C1plusTheta stands for the sum f_ij + theta FOR the TBS example -->
            <ct:Variable symbolType="real" symbId="C1plusTheta">
                <ct:Assign>
                    <Equation xmlns="http://www.pharmml.org/2013/03/Maths">
                        <Binop op="plus">
                            <ct:SymbRef symbIdRef="C1model"/>
                            <ct:SymbRef blkIdRef="pm1" symbIdRef="theta"/>
                        </Binop>
                    </Equation>
                </ct:Assign>
            </ct:Variable>
        </StructuralModel>
        
        
        <!-- OBSERVATION MODELS -->
        <!-- placeholder for observation error definition, e.g. 
            <ObservationModel blkId="om1">' 
                ...
            </ObservationModel> 
        -->
        
    </ModelDefinition>
    
    <!-- TRAIL DESIGN -->
    <!-- placeholder for trial design model:
        <TrialDesign xmlns="http://www.pharmml.org/2013/03/TrialDesign">
            ...
        </TrialDesign>
    -->
    
    <!-- MODELLING STEPS -->
    <!-- placeholder for modelling steps:
        <ModellingSteps xmlns="http://www.pharmml.org/2013/03/ModellingSteps">
            ...
        </ModellingSteps>
    -->

</PharmML>
\end{xmlcode}



%%%%%%%%%%%%%%%%%%%%%%%%%%%%%%%%%%%%%%%%%%%%%%%%%%%%%%%%%%%%%
%\begin{figure}[htbp]
%\centering
%\includegraphics[width=\linewidth]{../pics/trialDesign_caseF}
%\caption{Design overview}
%\label{fig:data-conversion}
%\end{figure}



%\begin{align*}
%u(x, \lambda) &=
%\begin{cases}
%\frac{x^\lambda -1}{\lambda} & \text{ for } \lambda \neq 0\\
%\log(x) & \text{ for } \lambda = 0
%\end{cases}
%\end{align*}


%\chapter{Bibliography}
\bibliography{pharmml-specification}

\end{document}

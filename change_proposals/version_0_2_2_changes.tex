
\documentclass[a4paper,10pt]{article}

\usepackage{graphicx}
\usepackage{hyperref}
\usepackage{amsmath}
\usepackage{amssymb}
\usepackage{xspace}
\pagestyle{headings}
%\usepackage[margin=1.2in]{geometry}
\usepackage[total={6in,9.5in}, top=1.5in, left=.8in, includefoot]{geometry}
\usepackage{float}
\restylefloat{table}
\usepackage{listings}
\usepackage{color}
\definecolor{gray}{rgb}{0.4,0.4,0.4}
\definecolor{darkblue}{rgb}{0.0,0.0,0.58}
\definecolor{attributeColor}{rgb}{0.96,0.517,0.29}
\definecolor{darkgreen}{rgb}{0,.392,0}
\definecolor{stringColor}{rgb}{0.6,0.2,0}

\lstset{
  basicstyle=\ttfamily,
  columns=fullflexible,
  showstringspaces=false,
  commentstyle=\color{gray}\upshape
%  numbers=left,
%  stepnumber=5
}

\lstdefinelanguage{XML}
{
  basicstyle=\ttfamily\footnotesize,
  morestring=[b]",
  morestring=[s]{>}{<},
  moredelim=[s][\bfseries\color{darkblue}]{<}{\ },
  moredelim=[s][\bfseries\color{darkblue}]{</}{>},
  moredelim=[l][\bfseries\color{darkblue}]{/>},
  moredelim=[l][\bfseries\color{darkblue}]{>},
  morecomment=[s]{<?}{?>},
  morecomment=[s]{<!--}{-->},
  morecomment=[s]{<![CDATA[}{]]>},
  commentstyle=\color{darkgreen},
  stringstyle=\color{stringColor},
  identifierstyle=\color{red},
  keywordstyle=\color{attributeColor},
  morekeywords={oid,columnId,columnIdRef,symbId,symbolType,op,columnNum,columnType,valueType,inputTarget,blkId,blkIdRef,symbIdRef,xmlns,version,type,VariableMapping,IndividualMapping, NONMEMdataSet} % list your attributes here
}

\lstdefinelanguage{NONMEMdataSet}
{
  basicstyle=\ttfamily\footnotesize,
  morestring=[b]",
  morestring=[s]{>}{<},
  moredelim=[s][\bfseries\color{darkblue}]{<}{\ },
  moredelim=[s][\bfseries\color{darkblue}]{</}{>},
  moredelim=[l][\bfseries\color{darkblue}]{/>},
  moredelim=[l][\bfseries\color{darkblue}]{>},
  morecomment=[s]{<?}{?>},
  morecomment=[s]{<!--}{-->},
  morecomment=[s]{<![CDATA[}{]]>},
  commentstyle=\color{darkgreen},
  stringstyle=\color{stringColor},
  identifierstyle=\color{black},
  keywordstyle=\color{attributeColor},
  morekeywords={kjkj} % list your attributes here
}

\author{MJS, SM}
\date{\today}
%\title{Extension for PharmML 0.2.2}
\title{NONMEM format dataset support for PharmML 0.2.2}


\newcommand{\cellml}{CellML\xspace}
\newcommand{\sbml}{SBML\xspace}
\newcommand{\sedml}{SED-ML\xspace}
\newcommand{\mathml}{MathML\xspace}
\newcommand{\uncertml}{UncertML\xspace}
\newcommand{\pharmml}{PharmML\xspace}
\newcommand{\xelem}[1]{\texttt{<#1>}\index{XML Element!\texttt{<#1>}}}
\newcommand{\xatt}[1]{\texttt{#1}\index{XML Attribute!\texttt{#1}}}

\begin{document}

\maketitle

\tableofcontents

\section{External data file support}
So far, only inline storage of data was supported, for example covariates were stored in the following 
table in the \xelem{Population} block:

\lstset{language=XML}
\begin{lstlisting}
    <ds:DataSet>
        <ds:Definition>
            <ds:Column columnId="ID" valueType="id" columnNum="1"/>
            <ds:Column columnId="ARM" valueType="id" columnNum="2"/>
            <ds:Column columnId="SEX" valueType="id" columnNum="3"/>
            <ds:Column columnId="EPOCH" valueType="id" columnNum="4"/>
        </ds:Definition>
        <ds:Table>
            <ds:Row><ct:Id>i1</ct:Id><ct:Id>a1</ct:Id><ct:Id>M</ct:Id><ct:Id>ep1</ct:Id></ds:Row>
            <ds:Row><ct:Id>i1</ct:Id><ct:Id>a1</ct:Id><ct:Id>M</ct:Id><ct:Id>ep3</ct:Id></ds:Row>
            <ds:Row><ct:Id>i2</ct:Id><ct:Id>a1</ct:Id><ct:Id>M</ct:Id><ct:Id>ep1</ct:Id></ds:Row>
            <ds:Row><ct:Id>i2</ct:Id><ct:Id>a1</ct:Id><ct:Id>M</ct:Id><ct:Id>ep3</ct:Id></ds:Row>
            <!-- SKIP -->
        </ds:Table>
    </ds:DataSet>
    \end{lstlisting}
                    
This has number of disadvantages, therefore an externalisation of datasets is supported in the 
upcoming release. As shown in the following listing, the external data set can be specified 
alternatively in the new \xelem{ImportData} element, instead of \xelem{Table}, by providing the file name, URL, file format 
and delimiter. Only one file format is allowed so far, the \emph{character-separated value}, 
\emph{CSV}, with three delimiters: \emph{COMMA}, \emph{SPACE} and \emph{TAB}.
The template for this reads:
\lstset{language=XML}
\begin{lstlisting}
    <ds:ImportData oid="id1">
        <ds:name>FILENAME</ds:name>
        <ds:url>URL_TO_FILE</ds:url>
        <ds:format>FORMAT</ds:format>
        <ds:delimiter>DELIMITER_TYPE</ds:delimiter>
    </ds:ImportData>
    \end{lstlisting}
For example:
\lstset{language=XML}
\begin{lstlisting}
    <ds:DataSet>
        <ds:Definition>
            <ds:Column columnId="ID" columnType="id" valueType="id" columnNum="1"/>
            <ds:Column columnId="ARM" columnType="arm" valueType="id" columnNum="2"/>
            <ds:Column columnId="SEX" columnType="covariate" valueType="id" columnNum="3"/>
            <ds:Column columnId="EPOCH" columnType="epoch" valueType="id" columnNum="4"/>
        </ds:Definition>
    <ds:ImportData oid="id1">
        <ds:name>warfarin_conc_pca</ds:name>
        <ds:url>./../examples_0_2_2/datasets/</ds:url>
        <ds:format>CSV</ds:format>
        <ds:delimiter>COMMA</ds:delimiter>
    </ds:ImportData>
    \end{lstlisting}

Note, that an additional attribute \xatt{columnType} is used to annotate the column, see section \ref{subsec:refNONMEM} 
for detailed description.
                    
\section{NONMEM-format dataset support}
%The reference and mapping to the external NONMEM-format dataset is discussed in connection with an estimation task i.e. one located in the \xelem{EstimationStep}, within \xelem{ModellingSteps}, but it can be defined in \xelem{SimulationStep} as well if necessary. 

At first the mapping of the columns of the data file to the according elements in the 
\xelem{ModelDefinition} will be described, then the dataset reference will be discussed 
and at the end how to handle NMTRAN code injection. The warfarin dataset is used for the first 
two sections, \emph{warfarin\_conc\_pca.csv}:

\begin{table}[ht]
\begin{center}
\begin{tabular}{lllllllll}
  \hline
\#ID	& time	& wt	& age	& sex	& amt	& dvid	& dv	& mdv \\
1	& 0	& 66.7	& 50	& 1	& 100	& 0	& .	& 1 \\
1	& 0	& 66.7	& 50	& 1	& .	& 2	& .	& 1 \\
1	& 0.5	& 66.7	& 50	& 1	& .	& 1	& 0	& 0 \\
...\\
  \hline
\end{tabular}
\end{center}
\end{table}
The Chan et al. dataset will be used in the last section to explain the code injection.

\subsection{Structure in nutshell}
The support for NONMEM datasets is an extension of the \xelem{EstimationStep} blocks and 
is located in the \xelem{NONMEMdataSet} element. It consists of three parts:
\begin{itemize}
\item
Part I -- Mapping of the variables defined in the dataset
\item
Part II -- NONMEM-format dataset (as external files or inline data)
\item
Part III -- Code injection (with symbol mapping)
\end{itemize}

\subsection{Part I -- Mapping of the columns in dataset}
A mapping type proposed in the form 
 \lstset{language=XML}
\begin{lstlisting}
                <ColumnMapping>
                    <ds:ColumnRef columnIdRef="COLUMN_NAME"/>
                    <ct:SymbRef symbIdRef="SYMBOL_NAME_IN_THE_MODEL"/>
                </ColumnMapping>
\end{lstlisting}
The following code (warfarin\_PK\_PRED.xml) illustrates the mapping of dataset columns 
to the variables or covariates as used in one of the elements of \xelem{ModelDefinition}, 
such as \xelem{CovariateModel  blkId="cm1"} or \xelem{ObservationModel blkId="om1"}
\lstset{language=XML}
\begin{lstlisting}
<!-- MODELLING STEPS -->
<ModellingSteps xmlns="http://www.pharmml.org/2013/03/ModellingSteps">
    
    <!-- ESTIMATION -->
    <EstimationStep oid="estimStep1">
        
            <NONMEMdataSet>
            
                <!-- Mapping -->
                <ColumnMapping>
                    <ds:ColumnRef columnIdRef="TIME"/>
                    <ct:SymbRef symbIdRef="t"/>
                </ColumnMapping>
                <ColumnMapping>
                    <ds:ColumnRef columnIdRef="DV"/>
                    <ct:SymbRef blkIdRef="om1" symbIdRef="C_obs"/>
                </ColumnMapping>
                <ColumnMapping>
                    <ds:ColumnRef columnIdRef="WT"/>
                    <ct:SymbRef blkIdRef="cm1" symbIdRef="W"/>
                </ColumnMapping>
                <ColumnMapping>
                    <ds:ColumnRef columnIdRef="AGE"/>
                    <ct:SymbRef blkIdRef="cm1" symbIdRef="AGE"/>
                </ColumnMapping>
                <ColumnMapping>
                    <ds:ColumnRef columnIdRef="SEX"/>
                    <ct:SymbRef blkIdRef="cm1" symbIdRef="SEX"/>
                </ColumnMapping>
                \end{lstlisting}
All other symbols/flags and variables, such as \emph{dvid}, \emph{mdv} etc. not used in the 
model will not be mapped. Instead, to assign meaning to these variables, the attribute 
\xatt{columnType} is used in the column definition in the \xelem{DataSet}, similar to the attribute 
\xatt{use} in MDL. See next section for more details.

%\subsubsection{Reserved headers in Monolix}
%The following list collects main headers identified by Monolix (User Guide, version 4.2.2) 
%%but as discussed below in the Chan/Holford model there can be many other column names:
%\begin{itemize}
%\item
%ID (or \#ID, I) -- subject identifiers
%\item
%TIME (or T) -- time
%\item
%AMT (or DOSE, D) -- dose
%\item
%X (or REG, XX) -- any regression variable
%\item
%Y (or DV, CONC) -- observations
%\item
%YTYPE (or ITYPE, TYPE, DVID) -- for the type of observations when there are several types of observations (1 for the first type, 2 for the second type, ...). YTYPE is not necessary in the case of a single output.
%\item
%COV -- continuos covariates 
%\item
%CAT -- categorical covariates
%\item
%OCCASION (or OCC) -- occasions
%\item
%ADM -- type of administration
%\item
%CENS -- left-censored data. Can be -1 to represent right-censored data.
%\item
%LIMIT for interval-censored data. It gives the lower limit while Y gives the upper limit.
%\item
%RATE (or R) -- infusion rate
%\item
%TINF -- infusion duration
%\item
%SS -- steady-state (requires column II)
%\item
%ADDL -- number of additional doses  (requires column II)
%\item
%II (or TAU) -- inter-dose interval
%\item
%MDV (Missing Dependent Variable). MDV= 0 if the row contains an observation and MDV=1 otherwise. MDV is not necessary if a dot is used to say when a row does not contain any measurement (Y=���). You can use MDV=2 to include times for regression variables updates and for prediction evaluation
%\item
%EVID -- Dose events. EVID is not necessary if DOSE=��� is used when a row does not contain any dose information. EVID=4 resets the system to the initial state.
%\end{itemize}
%
%\begin{description}
%\item[Question 1] NONMEM allows any string to be used in the header, correct? 
%This has consequences for the translation from MDL/NMTRAN to PharmML because, if the answer is \emph{yes}, the translation tool will not know what arbitrary names stand for.
%\end{description}

\subsection{Part II -- Referencing NONMEM dataset}
\label{subsec:refNONMEM}
This part is fairly straightforward. The structure is that of an externalised dataset discussed above. The columns 
are defined by providing the attributes for the column identifier, column type, value type and 
column number. 

\begin{lstlisting}
                <ds:DataSet>
                    <ds:Definition>
                        <ds:Column columnId="ID" columnType="id" valueType="id" columnNum="1"/>
                        <ds:Column columnId="TIME" columnType="time" valueType="real" columnNum="2"/>
                        <ds:Column columnId="WT" columnType="covariate" valueType="real" columnNum="3"/>
                        <ds:Column columnId="AGE" columnType="covariate" valueType="real" columnNum="4"/>
                        <ds:Column columnId="SEX" columnType="covariate" valueType="int" columnNum="5"/>
                        <ds:Column columnId="AMT" columnType="dose" valueType="real" columnNum="6"/>
                        <ds:Column columnId="DVID" columnType="dvid" valueType="real" columnNum="7"/>
                        <ds:Column columnId="DV" columnType="dv" valueType="real" columnNum="8"/>
                        <ds:Column columnId="MDV" columnType="mdv" valueType="real" columnNum="9"/>
                    </ds:Definition>
                    <ds:ImportData oid="id1">
                        <ds:name>warfarin_conc_pca</ds:name>
                        <ds:url>file:///../examples_0_2_2/datasets/</ds:url>
                        <ds:format>CSV</ds:format>
                        <ds:delimiter>COMMA</ds:delimiter>
                    </ds:ImportData>
                </ds:DataSet>            
            </NONMEMdataSet>
\end{lstlisting}

The following Tabel \ref{tab:MDLPharmML_columnTypes} collects all possible values for the \xatt{columnType} attribute which is identical with one in MDL. 
In contrast to MDL, PharmML allows only one value for each type.

\begin{table}[ht]
\begin{center}
\begin{tabular}{lll}
  \hline
\xatt{use}/MDL & \xatt{columnType}/PharmML & Meaning \\  
  \hline
addl &  addl  & number of additional doses     \\
adm &  adm  & type of administration     \\
cens &  censoring  & left-censored data     \\
covariate &  covariate  & covariates     \\
dose &  dose  & dose      \\
dv &  dv  & dependent variable     \\
dvid, type, itype &  dvid  & mixed observations     \\
evid &  evid  &  dose events     \\
id & id   & subject identifiers      \\
idv & idv   & independent variable     \\
ii, tau & ii   & inter-dose interval     \\
tinf & duration & infusion duration \\
limit &  limit  & lower limit for interval-censored data     \\
mdv &  mdv  & missing dependent variable     \\
occasion & occasion   & occasions     \\
rate & rate   & infusion rate     \\
reg & reg   & regression variable     \\
ss & ss   & steady state     \\
time &  time  & time     \\
   \hline
\end{tabular}
\caption{The table summarises allowed values for the attribute \xatt{use} in MDL and their counterpart \xatt{columnType} in PharmML.}
\label{tab:MDLPharmML_columnTypes}
\end{center}
\end{table}

Additionally there are few PharmML specific types the user can chose from to assign 
to \xatt{columnType}, see Table \ref{tab:PharmML_columnTypes}. 
They correspond to one of the symbols/elements in the \xelem{TrialDesign}, i.e. when the 
study design is defined explicitly and not completely by the NONMEM dataset. Accordingly 
these attribute values should be used in tables used in blocks \xelem{Population}, 
\xelem{ObjectiveDataSet} and \xelem{IndividualDosing}.

\begin{table}[ht]
\begin{center}
\begin{tabular}{lll}
  \hline
\xatt{columnType}/PharmML & Meaning \\  
  \hline
arm &     study arm \\
demographic &     demographic type \\
doseAmount &     dose amount \\
dosingTime &     time of dosing \\
epoch &     study epoch \\
replicate &     in cases when subjects are assumed identical \\
ssEndTime &     steady-state administration end time \\
ssPeriod &     steady-state administration interval \\
  \hline
\end{tabular}
\caption{The table lists additional values allowed for the attribute \xatt{columnType} in PharmML.}
\label{tab:PharmML_columnTypes}
\end{center}
\end{table}

                 
\subsection{Part III -- NMTRAN code injection (Chan et al.)}
This section goes beyond the standard models, such as the previous one, and deals with cases 
when the model and dataset are strongly interconnected. Not only does the dosing come from 
the datafile but also certain model relevant information is stored in the data. For example, additionally 
to the usual MDL encoding, NMTRAN code needs to be passed 'as is' to the target tool via PharmML, 
i.e. in this case the following code
\lstset{language=NONMEMdataSet}
\begin{lstlisting}
      if (PREV==0) {
         D1=TTK0
      } else {
         R1=CL*CSS
      }
\end{lstlisting}
which contains links between certain \emph{flags}, e.g. '1' in PREV or '-2' RATE, encoded in the dataset and the 
model definition and parameters.

This NMTRAN code is right now part of MDL but we think that this needs to be marked as target 
code and handled by MDL so that the translator knows which part is NONMEM-specific and 
needs to be passed without any interpretation.

The above example exemplifies so called \emph{events} which appear in the Chan/Holford model. 
Below the first few rows of the according dataset are listed:

\begin{table}[ht]
\begin{center}
\scriptsize
\begin{tabular}{llllllllllllllll}
  \hline
\#ID& MONTH& TRIAL& OCC& WTKG& PREV& TIME& AMT& RATE& SS& II& CMT& DV& MDV \\
552& 0& 1& 1& 73& 1& 0& 0& 0& 1& 0& 1& .& 1 \\
552& 0& 1& 1& 73& 0& 0& 0& 0& 0& 0& 1& 0.46& 0 \\
552& 0& 1& 1& 73& 0& 0& 740.37& -2& 0& 0& 1& .& 1\\
552& 0& 1& 1& 73& 0& 0.25& 0& 0& 0& 0& 1& .& 1\\
552& 0& 1& 1& 73& 0& 0.5& 0& 0& 0& 0& 1& .& 1 \\
... & ... & ... & ... & ... & ... & ... & ... & ... & ... & ... & ... & ... & ... \\
  \hline
\end{tabular}
\end{center}
\end{table}


Part I and II of the code are analog to the previous example. Only columns appearing in the \xelem{ModelDefinition} 
will be mapped using the generic \xelem{ColumnMapping}. All other have an appropriate attribute \xatt{columnType}
which identifies each column.
 
\lstset{language=XML}
\begin{lstlisting}
    <ModellingSteps xmlns="http://www.pharmml.org/2013/03/ModellingSteps">
        <EstimationStep oid="estStep">
           
           <!-- NONMEM OBJECTIVE DATA -->
            <NONMEMdataSet>
                
                <!-- NM dataset mapping -->
                <ColumnMapping>
                    <ds:ColumnRef columnIdRef="WTKG"/>
                    <ct:SymbRef blkIdRef="cm1" symbIdRef="W"/>
                </ColumnMapping>
                <ColumnMapping>
                    <ds:ColumnRef columnIdRef="TIME"/>
                    <ct:SymbRef symbIdRef="t"/>
                </ColumnMapping>
                <ColumnMapping>
                    <ds:ColumnRef columnIdRef="DV"/>
                    <ct:SymbRef blkIdRef="om1" symbIdRef="C_obs"/>
                </ColumnMapping>
                
                <!-- Dataset definition -->
                <ds:DataSet>
                    <ds:Definition>
                        <ds:Column columnId="ID" columnType="id" valueType="id" columnNum="1"/>
                        <ds:Column columnId="MONTH" columnType="covariate" valueType="real" columnNum="2"/>
                        <ds:Column columnId="TRIAL" columnType="covariate" valueType="real" columnNum="3"/>
                        <ds:Column columnId="OCC" columnType="occasion" valueType="real" columnNum="4"/>
                        <ds:Column columnId="WTKG" columnType="covariate" valueType="real" columnNum="5"/>
                        <ds:Column columnId="PREV" columnType="covariate" valueType="real" columnNum="6"/>
                        <ds:Column columnId="TIME" columnType="time" valueType="real" columnNum="7"/>
                        <ds:Column columnId="AMT" columnType="dose" valueType="real" columnNum="8"/>
                        <ds:Column columnId="RATE" columnType="rate" valueType="real" columnNum="9"/>
                        <ds:Column columnId="SS" columnType="ss" valueType="real" columnNum="10"/>
                        <ds:Column columnId="II" columnType="ii" valueType="real" columnNum="11"/>
                        <ds:Column columnId="CMT" valueType="real" columnNum="12"/>
                        <ds:Column columnId="DV" columnType="dv" valueType="real" columnNum="13"/>
                        <ds:Column columnId="MDV" columnType="mdv" valueType="real" columnNum="14"/>
                    </ds:Definition>
                    <ds:ImportData oid="id1">
                        <ds:name>warfarin_conc_pca</ds:name>
                        <ds:url>./../examples_0_2_2/datasets/</ds:url>
                        <ds:format>CSV</ds:format>
                        <ds:delimiter>COMMA</ds:delimiter>
                    </ds:ImportData>
                </ds:DataSet>
\end{lstlisting}

The following is new and specific to Part III.
The first section of the following listing shows again a mapping, this time it is about the variables/parameters 
used in the NMTRAN code to be injected. Some of them are model related, others are just indicators for NONMEM 
to perform a particular operation. (For now, the general rule is that we don't distinguish between them -- unless 
a translator from from NMTRAN/MDL to PharmML will provide this information.)

The mapping of the parameters \emph{TTK0}, \emph{CSS, CL} and the flag \emph{PRED} are shown. 
Note that the three parameters are mapped to the according symbols in the 
\xelem{ParameterModel}, but the last one, \emph{PRED}, to the column in the above dataset. 
As before we use one general \xelem{SymbolMapping} element. 

The final \xelem{NMTRANcode} section starts with a list of all symbols appearing in the NMTRAN 
code which provides the ground for the above \xelem{SymbolMapping} mapping. 
\lstset{language=XML}
\begin{lstlisting}
                <!-- Code injection -->
                <CodeInjection>
                    <!-- TTK0, CSS and CL are mapped to the ParameterModel 'pm1' -->
                    <SymbolMapping>
                        <ct:SymbRef symbIdRef="TTK0"/>
                        <ct:SymbRef blkIdRef="pm1" symbIdRef="TTK0"/>
                    </SymbolMapping>
                    <SymbolMapping>
                        <ct:SymbRef symbIdRef="CSS"/>
                        <ct:SymbRef blkIdRef="pm1" symbIdRef="CSS"/>
                    </SymbolMapping>
                    <SymbolMapping>
                        <ct:SymbRef symbIdRef="CL"/>
                        <ct:SymbRef blkIdRef="pm1" symbIdRef="CL"/>
                    </SymbolMapping>
                    <!-- PRED is mapped to the dataset column define above -->
                    <SymbolMapping>
                        <ct:SymbRef symbIdRef="PRED"/>
                        <ds:ColumnRef columnIdRef="PRED"/>
                    </SymbolMapping>
                    <!-- The following symbols are extracted from the injected code -->
                    <NMTRANcode>
                        <Symbol symbId="CSS"/>
                        <Symbol symbId="TTK0"/>
                        <Symbol symbId="CL"/>
                        <Symbol symbId="PRED"/>
                        <Symbol symbId="D1"/>
                        <Symbol symbId="R1"/>
                        <Code>
                            <![CDATA[
                                if (PRED==0) {
                                D1=TTK0
                                } else {
                                R1=CL*CSS
                                }                    
                            ]]>
                        </Code>
                    </NMTRANcode>
                </CodeInjection>
            </NONMEMdataSet>
\end{lstlisting}

Then, the code is stored in the \xelem{![CDATA[]]}, a structure which makes sure that everything inside it is 
ignored by the language parser and passed 'as is'.

The last question to discuss is:
\begin{description}
\item[Question] Are mappings of e.g. R1, D1 required?
\item[Answer] The easiest solution, for now, would be to extract all symbols out of the target code. 
\item[Alternatively] MDL should deal with it by making sure that the target code passed to PharmML 
follows well defined rules and grammar.
\end{description}

\section{Lookup table as new input type}
Table \ref{tab:C1C2C2} gives an overview of new options provided. In the first two cases experimental 
data is known and provided in form of a lookup table. For example, the PK profile has been measured 
but not model has been estimated so the data has to be connected to a subsequent PD model using
an interpolation method. In case \emph{C1} the user can chose from a list of known algorithms, e.g.
\{nearest, linear, spline, chip, cubic \}\footnote{The choice is based on the interpolation algorithms 
provided by MATLAB, section \emph{interp1:1-D data interpolation (table lookup)}' 
\url{http://www.mathworks.co.uk/help/matlab/ref/interp1.html}}. In case \emph{C2} the interpolation 
is an arbitrary user-defined piece-wise function. In case \emph{C3} not data is available but the 
parameters of an user-defined function are estimated based on the model only.


\begin{table}[ht!]
\begin{center}

\begin{tabular}{ll | ll | l}
  \hline
Type	 & Setup/Input  & TrialDesign & Target/ModelDefinition & Comments \\ 
  \hline
  \hline
  & \multicolumn{4}{ c }{\textbf{General time-varying input}} \\
  \hline
C1 	& Experimental data \&  	& -- Data as lookup table 		& -- Structural model 			& No \emph{Input} parameters  \\ [-.25ex]
	& Interpolation type 		& -- Reference to \emph{Input}	& with \emph{Input}				& to be estimated \\ [-.25ex]	
	& from list	 			&						& -- Interpolation type, e.g.:		& \\ [-.25ex]
	& 					& 						&  \emph{nearest}, \emph{linear}, \emph{pchip} etc. & \\ [1ex]
C2	& Experimental data \& 	& -- Data as lookup table 		& -- Structural model  			& No \emph{Input} parameters  \\ [-.25ex] 
	& User-defined input		& -- Reference to \emph{Input}	& with \emph{Input} 				& to be estimated \\ [-.25ex]
	& function			 	& 						& -- \emph{Input} as piece-wise	& \\ [-.25ex] 
	&   					&  						& function  					& \\ [-.25ex]
	& 					&  						&  							& \\  [1ex]
C3	& Data unavailable 		& 						& -- Structural model  			& Estimation of \emph{Input}  \\  [-.25ex] 
	& User-defined input 	& 						& with \emph{Input} 				& parameters is \\ [-.25ex] 
	& function			  	& 						& -- \emph{Input} as piece-wise  	& of interest \\  [-.25ex]
	&					&						& function  					& \\  [-.25ex]
	& 					& 						& -- \emph{Input} parameters in  	& \\  [-.25ex] 
	&					& 						& Parameter model 				& \\  [.5ex]
   \hline
\end{tabular}
\end{center}
\caption{The table summarises possible configurations of what is given as input, column \emph{Setup/Input} and how it can be structured and implemented, columns \emph{TrialDesign} and \emph{Target/ModelDefinition}.}
\label{tab:C1C2C2}
\end{table}

\subsection{Case C1}
The main ingredients in this case are 
\begin{itemize}
\item
\emph{Cc} -- measured drug concentration encoded as a variable in the model AND interpolation type 
chosen from a list: \emph{\{constant, linear, spline, pchip\}}

\lstset{language=XML}
\begin{lstlisting}
            <!-- TARGET FOR LOOKUP DATA REFERENCE -->
            <ct:Variable symbolType="real" symbId="Cc"/>
                <ct:Assign>
                    <ct:Interpolation>
                        <ct:Algorithm>linear</ct:Algorithm>
                        <ct:InterpIndepVar>
                            <ct:SymbRef symbIdRef="t"/>
                        </ct:InterpIndepVar>
                    </ct:Interpolation>
                </ct:Assign>
            </ct:Variable>

\end{lstlisting}

\emph{Cc} is given as a lookup table defined in the \emph{TrialDesign}
\begin{table}[H]
\begin{center}
\scriptsize
\begin{tabular}{lllll}
  \hline
ID & TIME & EPOCH & ARM & Cc \\
  \hline
1& 10& 1& 1 & 10 \\
1& 20& 1& 1 & 12 \\
... & ... & ... & ... & ...\\
1& 10& 2& 1 & 6 \\
1& 20& 2& 1 & 12 \\
... & ... & ... & ... & ...\\
2& 10& 1& 1 & 2 \\
2& 20& 1& 1 & 4 \\
2& 40& 1& 1 & 7 \\
... & ... & ... & ... & ...\\
  \hline
\end{tabular}
\caption{Lookup table, first few data records from \emph{Cc\_lookupTable.csv} dataset.}
%\label{tab:lookupData}
\end{center}
\end{table}
\item
e.g. an effect model containing the target variable \emph{Cc}
\begin{align}
\frac{dE}{dt}=Rin \times \Big(1- \frac{Imax \times Cc}{Cc+IC_{50}}\Big) - kout \times E		\nonumber
\end{align}
\lstset{language=XML}
\begin{lstlisting}
            <!-- EFFECT MODEL with TARGET 'Cc' -->
            <ct:DerivativeVariable symbId="E" symbolType="real">
                <ct:Description>PCA</ct:Description>
                <ct:Assign>
                    <Equation xmlns="http://www.pharmml.org/2013/03/Maths">
                        ...
                           <Binop op="times">
                                 <ct:SymbRef blkIdRef="p1" symbIdRef="Imax"/>
                                 <ct:SymbRef symbIdRef="Cc"/>
                           </Binop>
                        ...
\end{lstlisting}
\end{itemize}
In order to make the connection between the column names in the dataset and the according variables in the \xelem{TrialDesign} and \xelem{ModelDefinition}, following mappings are defined using again only one type of mapping:

\lstset{language=XML}
\begin{lstlisting}
                <LookupTable inputTarget="variable">
                    <!-- MAPPING -->
                    <ColumnMapping> 
                        <ds:ColumnRef columnIdRef="TIME"/>
                        <ct:SymbRef symbIdRef="t"/>
                    </ColumnMapping>
                    
                    <ColumnMapping> 
                        <ds:ColumnRef columnIdRef="C"/>
                        <ct:SymbRef blkIdRef="sm1" symbIdRef="Cc"/>
                    </ColumnMapping>
                    
                    <!-- DATASET -->
                    <ds:DataSet>
                        <ds:Definition>
                            <ds:Column columnId="ID" columnType="id" valueType="id" columnNum="1"/>
                            <ds:Column columnId="TIME" columnType="time" valueType="real" columnNum="2"/>
                            <ds:Column columnId="EPOCH" columnType="epoch" valueType="real" columnNum="3"/>
                            <ds:Column columnId="ARM" columnType="arm" valueType="real" columnNum="4"/>
                            <ds:Column columnId="Cc" columnType="dv" valueType="real" columnNum="5"/>
                        </ds:Definition>
                        <ds:ImportData oid="importData">
                            <ds:name>Cc_lookupTable</ds:name>
                            <ds:url>file:///../../datasets/</ds:url>
                            <ds:format>CSV</ds:format>
                            <ds:delimiter>COMMA</ds:delimiter>
                        </ds:ImportData>
                    </ds:DataSet>
                </LookupTable>
\end{lstlisting}
Note, that the \xatt{inputTarget="variable"} attribute in the \xelem{LookupTable} tag points that the following input target is of the type \emph{variable},
which stands for a general input target, beside variables defined by an ODE or a dose variable, e.g. D.
 
The last part of the above listing, \xelem{DataSet}, follows the same rules as described 
in the section about the externalised datasets.

\subsection{Case C2}
In this case we model is defined as 

\begin{itemize}
\item
Cc -- measured drug concentration encoded as a variable in the model which is defined by 
the user-defined interpolation function, here \emph{MyInterpolationFunction(...)}
\lstset{language=XML}
\begin{lstlisting}
            <!-- USER-DEFINED TARGET FOR LOOKUP DATA REFERENCE -->
            <ct:Variable symbolType="real" symbId="Cc">
                <ct:Assign>
                    <math:Equation>
                        <math:FunctionCall>
                            <ct:SymbRef symbIdRef="MyInterpolationFunction"/>
                            <!-- SKIP -->
                        </math:FunctionCall>
                    </math:Equation>
                </ct:Assign>
            </ct:Variable>
\end{lstlisting}

is given as a lookup table defined in the \emph{TrialDesign}
\begin{table}[H]
\begin{center}
\scriptsize
\begin{tabular}{lllll}
  \hline
ID & TIME & EPOCH & ARM & Cc \\
  \hline
1& 10& 1& 1 & 10 \\
1& 20& 1& 1 & 12 \\
... & ... & ... & ... & ...\\
1& 10& 2& 1 & 6 \\
1& 20& 2& 1 & 12 \\
... & ... & ... & ... & ...\\
2& 10& 1& 1 & 2 \\
2& 20& 1& 1 & 4 \\
2& 40& 1& 1 & 7 \\
... & ... & ... & ... & ...\\
  \hline
\end{tabular}
\caption{Lookup table, first few data records from \emph{Cc\_lookupTable.csv} dataset.}
%\label{tab:lookupData}
\end{center}
\end{table}

\item
Interpolation function for \emph{Cc}

\begin{align}
Cc(t) =     \left\{ \begin{array}{rcl}
         k_{i-1} + \frac{k_i - k_{i-1}}{t_i - t_{i-1}} (t - t_{i-1}) & \mbox{for} & t_{i-1} <= t < t_i, i = 1\ldots7 \nonumber \\ 
         0 & \mbox{for} & else \nonumber
             \end{array}\right.
\end{align}

is defined in the \emph{StructuralModel}, e.g.

\lstset{language=XML}
\begin{lstlisting}
    <!-- FUNCTION DEFINITION -->
    <FunctionDefinition xmlns="http://www.pharmml.org/2013/03/CommonTypes"
        symbId="MyInterpolationFunction" symbolType="real">
        <!-- SKIP -->
    </FunctionDefinition>
\end{lstlisting}

\item
e.g. an effect model containing the target variable \emph{Cc}
\begin{align}
\frac{dE}{dt}=Rin \times \Big(1- \frac{Imax \times Cc}{Cc+IC_{50}}\Big) - kout \times E		\nonumber
\end{align}
\lstset{language=XML}
\begin{lstlisting}
            <!-- EFFECT MODEL with TARGET 'Cc' -->
            <ct:DerivativeVariable symbId="E" symbolType="real">
                <ct:Description>PCA</ct:Description>
                <ct:Assign>
                    <Equation xmlns="http://www.pharmml.org/2013/03/Maths">
                        ...
                           <Binop op="times">
                                 <ct:SymbRef blkIdRef="p1" symbIdRef="Imax"/>
                                 <ct:SymbRef symbIdRef="Cc"/>
                           </Binop>
                        ...
\end{lstlisting}
\end{itemize}
The complete code for the user-defined function, \emph{MyInterpolationFunction}, without the repetitive 
middle section for time points \emph{t2 ... t5}, is given in the following listing
\lstset{language=XML}
\begin{lstlisting}
    <!-- FUNCTION DEFINITION -->
    <FunctionDefinition xmlns="http://www.pharmml.org/2013/03/CommonTypes"
        symbId="MyInterpolationFunction" symbolType="real">
        <FunctionArgument symbId="time" symbolType="real"/>
        <FunctionArgument symbId="k0" symbolType="real"/>
        <FunctionArgument symbId="k1" symbolType="real"/>
        <!-- SKIP -->    
        <FunctionArgument symbId="k7" symbolType="real"/>
        <FunctionArgument symbId="t0" symbolType="real"/>
        <FunctionArgument symbId="t1" symbolType="real"/>
        <!-- SKIP -->    
        <FunctionArgument symbId="t7" symbolType="real"/>

        <Definition>
            <Equation xmlns="http://www.pharmml.org/2013/03/Maths">
                <Piecewise>
                    <Piece>
                        <Binop op="plus">
                            <ct:SymbRef symbIdRef="k0"/>
                            <Binop op="times">
                                <Binop op="divide">
                                    <Binop op="minus">
                                        <ct:SymbRef symbIdRef="k1"/>
                                        <ct:SymbRef symbIdRef="k0"/>
                                    </Binop>
                                    <Binop op="minus">
                                        <ct:SymbRef symbIdRef="t1"/>
                                        <ct:SymbRef symbIdRef="t0"/>
                                    </Binop>
                                </Binop>
                                <Binop op="minus">
                                    <ct:SymbRef symbIdRef="time"/>
                                    <ct:SymbRef symbIdRef="t0"/>
                                </Binop>
                            </Binop>
                        </Binop>
                        <Condition>
                            <LogicBinop op="and">
                                <LogicBinop op="geq">
                                    <ct:SymbRef symbIdRef="time"/>
                                    <ct:SymbRef symbIdRef="t0"/>
                                </LogicBinop>
                                <LogicBinop op="lt">
                                    <ct:SymbRef symbIdRef="time"/>
                                    <ct:SymbRef symbIdRef="t1"/>
                                </LogicBinop>
                            </LogicBinop>
                        </Condition>
                    </Piece>
                    <!-- SKIP -->                    
                    <Piece>
                        <Binop op="plus">
                            <ct:SymbRef symbIdRef="k6"/>
                            <Binop op="times">
                                <Binop op="divide">
                                    <Binop op="minus">
                                        <ct:SymbRef symbIdRef="k7"/>
                                        <ct:SymbRef symbIdRef="k6"/>
                                    </Binop>
                                    <Binop op="minus">
                                        <ct:SymbRef symbIdRef="t7"/>
                                        <ct:SymbRef symbIdRef="t6"/>
                                    </Binop>
                                </Binop>
                                <Binop op="minus">
                                    <ct:SymbRef symbIdRef="time"/>
                                    <ct:SymbRef symbIdRef="t6"/>
                                </Binop>
                            </Binop>
                        </Binop>
                        <Condition>
                            <LogicBinop op="and">
                                <LogicBinop op="geq">
                                    <ct:SymbRef symbIdRef="time"/>
                                    <ct:SymbRef symbIdRef="t6"/>
                                </LogicBinop>
                                <LogicBinop op="lt">
                                    <ct:SymbRef symbIdRef="time"/>
                                    <ct:SymbRef symbIdRef="t7"/>
                                </LogicBinop>
                            </LogicBinop>
                        </Condition>
                    </Piece>                
                </Piecewise>
            </Equation>
        </Definition>
    </FunctionDefinition>
\end{lstlisting}
The following listing shows how the just defined interpolation function is then referenced in the model
\lstset{language=XML}
\begin{lstlisting}
        <StructuralModel blkId="sm1">

            <!-- TARGET FOR LOOKUP DATA REFERENCE -->
            <ct:Variable symbolType="real" symbId="Cc">
                <ct:Assign>
                    <math:Equation>
                        <math:FunctionCall>
                            <ct:SymbRef symbIdRef="MyInterpolationFunction"/>
                            <math:FunctionArgument symbId="time">
                                <ct:SymbRef symbIdRef="t"/>
                            </math:FunctionArgument>
                            <math:FunctionArgument symbId="k0">
                                <ct:SymbRef blkIdRef="pm1" symbIdRef="k0"/>
                            </math:FunctionArgument>
                            <!-- SKIP -->     
                            <math:FunctionArgument symbId="t7">
                                <ct:SymbRef blkIdRef="pm1" symbIdRef="t7"/>
                            </math:FunctionArgument>
                        </math:FunctionCall>
                    </math:Equation>
                </ct:Assign>
            </ct:Variable>
\end{lstlisting}

The last part is the \emph{Lookup table} definition and mapping, identical to the previous example with the exception that no interpolation type is define for the \emph{Target} variable:

\lstset{language=XML}
\begin{lstlisting}
            <Activity oid="d1">
               <LookupTable>
               
                    <!-- MAPPING -->
                    <!-- SKIP --> 
			<!-- this part is identical to the code in C1 case -->
                    <!-- SKIP --> 

                    <!-- TARGET mapping -->
                    <LookupTableTargetMapping inputTarget="variable">                       		
                        <ds:ColumnRef columnIdRef="Cc"/>
                        <ct:SymbRef blkIdRef="sm1" symbIdRef="Cc"/>
                    </LookupTableTargetMapping>

                    <!-- DATASET -->
                    <!-- SKIP --> 
			<!-- this part is identical to the code in C1 case -->
                    <!-- SKIP --> 

                </LookupTable>                
            </Activity>
\end{lstlisting}



\bibliography{pharmml-specification}
\end{document}






%TEMPLATES 
%% 1. Template for table with figures
%\begin{figure}[htbp]
%\centering
%\begin{tabular}{cc}
% \includegraphics[width=80mm]{pics/pic1} & 
% \includegraphics[width=80mm]{pics/pic2} \\
% \includegraphics[width=80mm]{pics/pic3} &
% \includegraphics[width=80mm]{pics/pic4}
%\end{tabular}
%\caption{about the figure}
%\label{figTable:labelText}
%\end{figure}

%\begin{table}[ht]
%\begin{center}
%\begin{tabular}{rrrrrrrrrrr}
%  \hline
% & 1 & 2 & 3 & 4 & 5 & 6 & 7 & 8 & 9 & 10 \\ 
%  \hline
%1 & 0.24 & -1.47 & -0.56 & 0.24 & 0.71 & 1.23 & 0.44 & 0.40 & 1.10 & 1.84 \\ 
%   \hline
%\end{tabular}
%\end{center}
%\end{table}
 

%\begin{figure}[htb!]
%\centering
%  \includegraphics[width=105mm]{}
% \caption{}
% \label{fig:myplot}
%\end{figure}

%PIECE-WISE
%f(z) =     \left\{ \begin{array}{rcl}
%         value1 & \mbox{for} & condition1 \\ 
%         value1 & \mbox{for} & condition1
%             \end{array}\right.

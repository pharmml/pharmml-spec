
\documentclass[a4paper,10pt]{article}

\usepackage{graphicx}
\usepackage{hyperref}
\usepackage{amsmath}
\usepackage{amssymb}
\usepackage{xspace}
\pagestyle{headings}
%\usepackage[margin=1.2in]{geometry}
\usepackage[total={6in,9.5in}, top=1.5in, left=.8in, includefoot]{geometry}
\usepackage{float}
\restylefloat{table}
\usepackage{listings}
\usepackage{color}
\definecolor{gray}{rgb}{0.4,0.4,0.4}
\definecolor{darkblue}{rgb}{0.0,0.0,0.58}
\definecolor{attributeColor}{rgb}{0.96,0.517,0.29}
\definecolor{darkgreen}{rgb}{0,.392,0}
\definecolor{stringColor}{rgb}{0.6,0.2,0}

\lstset{
  basicstyle=\ttfamily,
  columns=fullflexible,
  showstringspaces=false,
  commentstyle=\color{gray}\upshape
%  numbers=left,
%  stepnumber=5
}

\lstdefinelanguage{XML}
{
  basicstyle=\ttfamily\footnotesize,
  morestring=[b]",
  morestring=[s]{>}{<},
  moredelim=[s][\bfseries\color{darkblue}]{<}{\ },
  moredelim=[s][\bfseries\color{darkblue}]{</}{>},
  moredelim=[l][\bfseries\color{darkblue}]{/>},
  moredelim=[l][\bfseries\color{darkblue}]{>},
  morecomment=[s]{<?}{?>},
  morecomment=[s]{<!--}{-->},
  morecomment=[s]{<![CDATA[}{]]>},
  commentstyle=\color{darkgreen},
  stringstyle=\color{stringColor},
  identifierstyle=\color{red},
  keywordstyle=\color{attributeColor},
  morekeywords={oid,columnId,columnIdRef,symbId,symbolType,op,columnNum,valueType,inputTarget,blkId,blkIdRef,symbIdRef,xmlns,version,type,VariableMapping,IndividualMapping, NONMEMdataSet} % list your attributes here
}

\lstdefinelanguage{NONMEMdataSet}
{
  basicstyle=\ttfamily\footnotesize,
  morestring=[b]",
  morestring=[s]{>}{<},
  moredelim=[s][\bfseries\color{darkblue}]{<}{\ },
  moredelim=[s][\bfseries\color{darkblue}]{</}{>},
  moredelim=[l][\bfseries\color{darkblue}]{/>},
  moredelim=[l][\bfseries\color{darkblue}]{>},
  morecomment=[s]{<?}{?>},
  morecomment=[s]{<!--}{-->},
  morecomment=[s]{<![CDATA[}{]]>},
  commentstyle=\color{darkgreen},
  stringstyle=\color{stringColor},
  identifierstyle=\color{black},
  keywordstyle=\color{attributeColor},
  morekeywords={kjkj} % list your attributes here
}

\author{MJS, SM}
\date{\today}
%\title{Extension for PharmML 0.2.2}
\title{NONMEM format dataset support for PharmML 0.2.2}


\newcommand{\cellml}{CellML\xspace}
\newcommand{\sbml}{SBML\xspace}
\newcommand{\sedml}{SED-ML\xspace}
\newcommand{\mathml}{MathML\xspace}
\newcommand{\uncertml}{UncertML\xspace}
\newcommand{\pharmml}{PharmML\xspace}
\newcommand{\xelem}[1]{\texttt{<#1>}\index{XML Element!\texttt{<#1>}}}
\newcommand{\xatt}[1]{\texttt{#1}\index{XML Attribute!\texttt{#1}}}

\begin{document}

\maketitle

\tableofcontents

\section{External data files support}
to do ....

\section{NONMEM-format dataset support}
%The reference and mapping to the external NONMEM-format dataset is discussed in connection with an estimation task i.e. one located in the \xelem{EstimationStep}, within \xelem{ModellingSteps}, but it can be defined in \xelem{SimulationStep} as well if necessary. 

At first the mapping of the columns of the data file to the according elements in the \xelem{ModelDefinition} 
will be described, then the dataset reference will be discussed and how to handle NMTRAN code injection. 
The warfarin dataset is used for the first two sections, \emph{warfarin\_conc\_pca.csv}:

\begin{table}[ht]
\begin{center}
\begin{tabular}{lllllllll}
  \hline
\#ID	& time	& wt	& age	& sex	& amt	& dvid	& dv	& mdv \\
1	& 0	& 66.7	& 50	& 1	& 100	& 0	& .	& 1 \\
1	& 0	& 66.7	& 50	& 1	& .	& 2	& .	& 1 \\
1	& 0.5	& 66.7	& 50	& 1	& .	& 1	& 0	& 0 \\
...\\
  \hline
\end{tabular}
\end{center}
\end{table}
The Chan et al. dataset will be used in the last section to explain the code injection.

\subsection{Structure in nutshell}
The support for NONMEM datasets is an extension of the \xelem{EstimationStep} blocks and is located 
in the \xelem{NONMEMdataSet} element. It consists of three parts:
\begin{itemize}
\item
Part I -- Mapping of the variables defined in the dataset (with three types - individual, covariate, general variable)
\item
Part II -- Dataset (as inline data or external files)
\item
Part III -- Code injection (with symbol mapping)
\end{itemize}

\subsection{Part I -- Mapping of the columns in dataset}
Three types of mapping types are proposed:
\begin{itemize}
\item
subject -- \xelem{NMIndividualMapping}
\item
covariates -- \xelem{NMCovariateMapping}
\item 
variables -- \xelem{NMVariableMapping}
\end{itemize}
which assumes that the translator from \emph{NN} tool $\rightarrow$ PharmML will be able to recognise 
these general categories. 
 
 
% 
%\subsubsection{Mapping of not model related variables, flags or parameters}
%The updated proposal is to NOT use additional type for mapping, e.g.
%\begin{lstlisting}     
%                <MDVMapping>
%                    <ds:ColumnRef columnIdRef="MDV"/>
%                </MDVMapping>
%                <DOSEMapping>
%                    <ds:ColumnRef columnIdRef="DOSE"/>
%                </DosingMapping>
%                <EVIDMapping>
%                    <ds:ColumnRef columnIdRef="EVID"/>
%                </EVIDMapping>
%\end{lstlisting}
%but to use only those three types described at the beginning of this section. i.e. \xelem{NMIndividualMapping} for \emph{subject}, \xelem{NMCovariateMapping} for \emph{covariates} and \xelem{NMVariableMapping} for all other \emph{flags, parameters} and \emph{variables}.
%

 
The following code illustrates the mapping of dataset columns to the variables and covariates as used 
in the model definition and other NONMEM format typical symbols/flags and variables, such as \emph{dvid}, 
\emph{mdv} etc. The identifier \emph{ID} is mapped to itself as there are no \emph{ID}'s in the model, 
similar holds for \emph{dvid}, \emph{mdv} etc.

\lstset{language=XML}
\begin{lstlisting}
<!-- MODELLING STEPS -->
<ModellingSteps xmlns="http://www.pharmml.org/2013/03/ModellingSteps">
    
    <!-- ESTIMATION -->
    <EstimationStep oid="estimStep1">
        
        <!-- NONMEM OBJECTIVE DATA -->
            <NONMEMdataSet>
                <!-- subject ID -->
                <NMIndividualMapping>
                    <ds:ColumnRef columnIdRef="ID"/>
                </NMIndividualMapping>
                <!-- time, wt, age, sex and dv can be mapped to variables/parameters in the model -->
                <NMVariableMapping>
                    <ds:ColumnRef columnIdRef="time"/>
                    <ct:SymbRef symbIdRef="t"/>
                </NMVariableMapping>
                <NMCovariateMapping>
                    <ds:ColumnRef columnIdRef="wt"/>
                    <ct:SymbRef blkIdRef="cm1" symbIdRef="W"/>
                </NMCovariateMapping>
                <NMCovariateMapping>
                    <ds:ColumnRef columnIdRef="age"/>
                    <ct:SymbRef blkIdRef="cm1" symbIdRef="AGE"/>
                </NMCovariateMapping>
                <NMCovariateMapping>
                    <ds:ColumnRef columnIdRef="sex"/>
                    <ct:SymbRef blkIdRef="cm1" symbIdRef="SEX"/>
                </NMCovariateMapping>
                <NMVariableMapping>
                    <ds:ColumnRef columnIdRef="dv"/>
                    <ct:SymbRef blkIdRef="om1" symbIdRef="C_obs"/>
                </NMVariableMapping>
                <!-- the following variables cannot be mapped to any model element and are optional-->
                <NMVariableMapping>
                    <ds:ColumnRef columnIdRef="amt"/>
                </NMVariableMapping>
                <NMVariableMapping>
                    <ds:ColumnRef columnIdRef="dvid"/>
                </NMVariableMapping>
                <NMVariableMapping>
                    <ds:ColumnRef columnIdRef="mdv"/>
                </NMVariableMapping>
                \end{lstlisting}

\begin{description}
\item[Note 1] The three types could be simplified to just one, dependent on the capability of the translator. 
\item[Note 2] For variables not appearing in the ModelDefinition the mapping is not mandatory.
\end{description}

\subsubsection{Reserved headers in Monolix}
The following list collects main headers identified by Monolix (User Guide, version 4.2.2) 
%but as discussed below in the Chan/Holford model there can be many other column names:
\begin{itemize}
\item
ID (or \#ID, I) -- subject identifiers
\item
TIME (or T) -- time
\item
AMT (or DOSE, D) -- dose
\item
X (or REG, XX) -- any regression variable
\item
Y (or DV, CONC) -- observations
\item
YTYPE (or ITYPE, TYPE, DVID) -- for the type of observations when there are several types of observations (1 for the first type, 2 for the second type, ...). YTYPE is not necessary in the case of a single output.
\item
COV -- continuos covariates 
\item
CAT -- categorical covariates
\item
OCCASION (or OCC) -- occasions
\item
ADM -- type of administration
\item
CENS -- left-censored data. Can be -1 to represent right-censored data.
\item
LIMIT for interval-censored data. It gives the lower limit while Y gives the upper limit.
\item
RATE (or R) -- infusion rate
\item
TINF -- infusion duration
\item
SS -- steady-state (requires column II)
\item
ADDL -- number of additional doses  (requires column II)
\item
II (or TAU) -- inter-dose interval
\item
MDV (Missing Dependent Variable). MDV= 0 if the row contains an observation and MDV=1 otherwise. MDV is not necessary if a dot is used to say when a row does not contain any measurement (Y=���). You can use MDV=2 to include times for regression variables updates and for prediction evaluation
\item
EVID -- Dose events. EVID is not necessary if DOSE=��� is used when a row does not contain any dose information. EVID=4 resets the system to the initial state.
\end{itemize}
%
%\begin{description}
%\item[Question 1] NONMEM allows any string to be used in the header, correct? 
%This has consequences for the translation from MDL/NMTRAN to PharmML because, if the answer is \emph{yes}, the translation tool will not know what arbitrary names stand for.
%\end{description}

\subsection{Part II -- Referencing NONMEM dataset}
This part is fairly straightforward. The structure is that of an externalised dataset. The columns are defined 
by providing the attributes for the column identifier, value type and column number. Then the file name, 
URL, format and delimiter are provided.

\begin{lstlisting}
                <ds:DataSet>
                    <ds:Definition>
                        <ds:Column columnId="ID" valueType="id" columnNum="1"/>
                        <ds:Column columnId="time" valueType="real" columnNum="2"/>
                        <ds:Column columnId="wt" valueType="real" columnNum="3"/>
                        <ds:Column columnId="age" valueType="real" columnNum="4"/>
                        <ds:Column columnId="sex" valueType="int" columnNum="5"/>
                        <ds:Column columnId="amt" valueType="real" columnNum="6"/>
                        <ds:Column columnId="dvid" valueType="real" columnNum="7"/>
                        <ds:Column columnId="dv" valueType="real" columnNum="8"/>
                        <ds:Column columnId="mdv" valueType="real" columnNum="9"/>
                    </ds:Definition>
                    <ds:ImportData oid="id1">
                        <ds:name>warfarin_conc_pca</ds:name>
                        <ds:url>file:///../examples_0_2_2/datasets/</ds:url>
                        <ds:format>CSV</ds:format>
                        <ds:delimiter>COMMA</ds:delimiter>
                    </ds:ImportData>
                </ds:DataSet>            
            </NONMEMdataSet>
\end{lstlisting}
Only one file format is allowed so far, the \emph{CSV}, with four different \emph{delimiters}: \emph{COMMA}, \emph{SPACE} and \emph{TAB}.

                 
\subsection{Part III -- NMTRAN code injection (Chan et al.)}
This section goes beyond the standard models, such as the previous one, and deals with cases 
when the model and dataset are strongly interconnected. Not only does the dosing come from 
the datafile but also certain model relevant information is stored in the data. For example, additionally 
to the usual MDL encoding, NMTRAN code needs to be passed 'as is' to the target tool via PharmML, 
i.e. in this case the following code
\lstset{language=NONMEMdataSet}
\begin{lstlisting}
      if (PREV==0) {
         D1=TTK0
      } else {
         R1=CL*CSS
      }
\end{lstlisting}
which contains links between certain \emph{flags}, e.g. PREV, encoded in the dataset and the 
model definition and parameters.

This NMTRAN code is right now part of MDL but we think that this needs to be marked as target 
code and handled by MDL so that the translator knows which part is NONMEM-specific and 
needs to be passed without any interpretation.

The above example exemplifies so called \emph{events} which appear in the Chan/Holford model. 
Below the first few rows of the according dataset are listed:

\begin{table}[ht]
\begin{center}
\scriptsize
\begin{tabular}{llllllllllllllll}
  \hline
\#ID& MONTH& TRIAL& OCC& WTKG& PREV& TIME& AMT& RATE& SS& II& CMT& DV& MDV \\
552& 0& 1& 1& 73& 1& 0& 0& 0& 1& 0& 1& .& 1 \\
552& 0& 1& 1& 73& 0& 0& 0& 0& 0& 0& 1& 0.46& 0 \\
552& 0& 1& 1& 73& 0& 0& 740.37& -2& 0& 0& 1& .& 1\\
552& 0& 1& 1& 73& 0& 0.25& 0& 0& 0& 0& 1& .& 1\\
552& 0& 1& 1& 73& 0& 0.5& 0& 0& 0& 0& 1& .& 1 \\
... & ... & ... & ... & ... & ... & ... & ... & ... & ... & ... & ... & ... & ... \\
  \hline
\end{tabular}
\end{center}
\end{table}


Part I and II of the code are analog to the previous example. Here, as explained before, three 
mapping types are considered. All columns other then \emph{subject} and \emph{covariate} related, 
such as \emph{OCC, PREV, SS} will be mapped using the \xelem{NMVariableMapping}. 
To support other types, one would require a very smart translator able to recognise of these 
names and their meaning.
 
\lstset{language=XML}
\begin{lstlisting}
    <ModellingSteps xmlns="http://www.pharmml.org/2013/03/ModellingSteps">
        <EstimationStep oid="estStep">
           
           <!-- NONMEM OBJECTIVE DATA -->
            <NONMEMdataSet>
                <!-- NM dataset mapping -->
                <NMIndividualMapping>
                    <ds:ColumnRef columnIdRef="ID"/>
                </NMIndividualMapping>
                <NMVariableMapping>
                    <ds:ColumnRef columnIdRef="MONTH"/>
                </NMVariableMapping>
                <NMVariableMapping>
                    <ds:ColumnRef columnIdRef="TRIAL"/>
                </NMVariableMapping>
                <NMVariableMapping>
                    <ds:ColumnRef columnIdRef="OCC"/>
                </NMVariableMapping>
                <NMCovariateMapping>
                    <ds:ColumnRef columnIdRef="WTKG"/>
                    <ct:SymbRef blkIdRef="cm1" symbIdRef="W"/>
                </NMCovariateMapping>
                <NMVariableMapping>
                    <ds:ColumnRef columnIdRef="PREV"/>
                </NMVariableMapping>                
                <NMVariableMapping>
                    <ds:ColumnRef columnIdRef="TIME"/>
                    <ct:SymbRef symbIdRef="t"/>
                </NMVariableMapping>
                <NMVariableMapping>
                    <ds:ColumnRef columnIdRef="AMT"/>
                </NMVariableMapping>
                <NMVariableMapping>
                    <ds:ColumnRef columnIdRef="RATE"/>
                </NMVariableMapping>
                <NMVariableMapping>
                    <ds:ColumnRef columnIdRef="SS"/>
                </NMVariableMapping>
                <NMVariableMapping>
                    <ds:ColumnRef columnIdRef="II"/>
                </NMVariableMapping>
                <NMVariableMapping>
                    <ds:ColumnRef columnIdRef="CMT"/>
                </NMVariableMapping>
                <NMVariableMapping>
                    <ds:ColumnRef columnIdRef="DV"/>
                    <ct:SymbRef blkIdRef="om1" symbIdRef="C_obs"/>
                </NMVariableMapping>
                <NMVariableMapping>
                    <ds:ColumnRef columnIdRef="MDV"/>
                </NMVariableMapping>
                <!-- Dataset definition -->
                <ds:DataSet>
                    <ds:Definition>
                        <ds:Column columnId="ID" valueType="id" columnNum="1"/>
                        <ds:Column columnId="MONTH" valueType="real" columnNum="2"/>
                        <ds:Column columnId="TRIAL" valueType="real" columnNum="3"/>
                        <ds:Column columnId="OCC" valueType="real" columnNum="4"/>
                        <ds:Column columnId="WTKG" valueType="real" columnNum="5"/>
                        <ds:Column columnId="PREV" valueType="real" columnNum="6"/>
                        <ds:Column columnId="TIME" valueType="real" columnNum="7"/>
                        <ds:Column columnId="AMT" valueType="real" columnNum="8"/>
                        <ds:Column columnId="RATE" valueType="real" columnNum="9"/>
                        <ds:Column columnId="SS" valueType="real" columnNum="10"/>
                        <ds:Column columnId="II" valueType="real" columnNum="11"/>
                        <ds:Column columnId="CMT" valueType="real" columnNum="12"/>
                        <ds:Column columnId="DV" valueType="real" columnNum="13"/>
                        <ds:Column columnId="MDV" valueType="real" columnNum="14"/>
                    </ds:Definition>
                    <ds:ImportData oid="id1">
                        <ds:name>dataFileName</ds:name>
                        <ds:url>./../examples_0_2_2/datasets/</ds:url>
                        <ds:format>CSV</ds:format>
                        <ds:delimiter>COMMA</ds:delimiter>
                    </ds:ImportData>
                </ds:DataSet>
                \end{lstlisting}

The following is new and specific to Part III.
The first section of the following listing shows again a mapping, this time it is about the variables/parameters 
used in the NMTRAN code to be injected. Some of them are model related, others are just indicators for NONMEM 
to perform a particular operation. 

\lstset{language=XML}
\begin{lstlisting}
                <!-- Code injection -->
                <CodeInjection>
                    <!-- TTK0, CSS and CL are mapped to the ParameterModel 'pm1' -->
                    <NMsymbolMapping>
                        <ct:SymbRef symbIdRef="TTK0"/>	
                        <ct:SymbRef blkIdRef="pm1" symbIdRef="TTK0"/>
                    </NMsymbolMapping>
                    <NMsymbolMapping>
                        <ct:SymbRef symbIdRef="CSS"/>
                        <ct:SymbRef blkIdRef="pm1" symbIdRef="CSS"/>
                    </NMsymbolMapping>
                    <NMsymbolMapping>
                        <ct:SymbRef symbIdRef="CL"/>
                        <ct:SymbRef blkIdRef="pm1" symbIdRef="CL"/>
                    </NMsymbolMapping>
                    <!-- PRED is mapped to the dataset column define above -->
                    <NMsymbolMapping>
                        <ct:SymbRef symbIdRef="PRED"/>
                        <ds:ColumnRef columnIdRef="PRED"/>
                    </NMsymbolMapping>
                    <NMsymbolMapping> 
                        <ct:SymbRef symbIdRef="D1"/>
                    </NMsymbolMapping>
                    <NMsymbolMapping>
                        <ct:SymbRef symbIdRef="R1"/>
                    </NMsymbolMapping>
                    <NMTRANcode>
                        <!-- The following symbols are extracted from the injected code below -->
                        <NMsymbol symbId="PRED"/>
                        <NMsymbol symbId="EVID"/>
                        <NMsymbol symbId="TTK0"/>
                        <NMsymbol symbId="D1"/>
                        <NMsymbol symbId="R1"/>
                        <Code>
                            <![CDATA[
                                if (PRED==0) {
                                D1=TTK0
                                } else {
                                R1=CL*CSS
                                }                    
                                ]]>
                        </Code>
                    </NMTRANcode>
                </CodeInjection>
            </NONMEMdataSet>
\end{lstlisting}

The code is stored in the \xelem{![CDATA[]]}, a structure which makes sure that everything inside it is 
ignored by the language parser and passed 'as is'.

The mapping of the parameters \emph{TTK0}, \emph{CSS, CL} and the flag \emph{PRED} are shown 
among others. Note that the first three are mapped to the according parameters in the 
\xelem{ParameterModel}, the last one, \emph{PRED}, to the column in the above dataset/table. 
So far only one general \xelem{NMSymbolMapping} element is defined based on the assumption 
that automatic distinction between parameters, flags and other symbols is not doable in general. 
The \xelem{NMTRANcode} section at the end starts with a list of all symbols appearing in the 
NMTRAN code which provides the ground for the \xelem{NMSymbolMapping} mapping. The last 
question to discuss is:
\begin{description}
\item[Question] Do mappings e.g. for R1, D1 make sense?
\item[Answer] The easiest solution, for now, would be to extract all symbols out of the target code, 
which however might result in a high number of non-sensical mappings. 
\item[Alternatively] MDL should deal with it by making sure that the target code passed to PharmML 
follows well defined rules and grammar.
\end{description}

\section{Lookup table with data}

\begin{table}[ht!]
\begin{center}

\begin{tabular}{ll | ll | l}
  \hline
Type	 & Setup/Input  & TrialDesign & Target/ModelDefinition & Comments \\ 
  \hline
  \hline
  & \multicolumn{4}{ c }{\textbf{General time-varying input}} \\
  \hline
C1 	& Experimental data \&  	& -- Data as lookup table 		& -- Structural model 			& No \emph{Input} parameters  \\ [-.25ex]
	& Interpolation type 		& -- Interpolation type, e.g.:	&with \emph{Input}				& to be estimated \\ [-.25ex]	
	& from list	 			& \emph{constant}, \emph{linear}, \emph{cubic} etc.  & 			& \\ [-.25ex]
	& 					& -- Reference to \emph{Input}&  							& \\ [-.25ex]
	& 					& 						& 							& \\ [+1ex]
C2	& Experimental data \& 	& -- Data as lookup table 		& -- Structural model  			& No \emph{Input} parameters  \\ [-.25ex] 
	& User-defined input		& -- Reference to \emph{Input}	& with \emph{Input} 				& to be estimated \\ [-.25ex]
	& function			 	& 						& -- \emph{Input} as piece-wise	& \\ [-.25ex] 
	&   					&  						& function  					& \\ [-.25ex]
	& 					&  						&  							& \\  [1ex]
C3	& Data unavailable 		& 						& -- Structural model  			& Estimation of \emph{Input}  \\  [-.25ex] 
	& User-defined input 	& 						& with \emph{Input} 				& parameters is \\ [-.25ex] 
	& function			  	& 						& -- \emph{Input} as piece-wise  	& of interest \\  [-.25ex]
	&					&						& function  					& \\  [-.25ex]
	& 					& 						& -- \emph{Input} parameters in  	& \\  [-.25ex] 
	&					& 						& Parameter model 				& \\  [.5ex]
   \hline
\end{tabular}
\end{center}
\caption{The table summarises possible configurations of what is given as input, column \emph{Setup/Input} and how it can be structured and implemented, columns \emph{TrialDesign} and \emph{Target/ModelDefinition}.}
\end{table}

\subsection{Case C1}
The main ingredients in this case are 
\begin{itemize}
\item
Cc -- measured drug concentration encoded as a variable in the model
\lstset{language=XML}
\begin{lstlisting}
            <!-- TARGET FOR LOOKUP DATA REFERENCE -->
            <ct:Variable symbolType="real" symbId="Cc"/>
                <ct:Assign>
                    <ct:Interpolation>
                        <ct:Algorithm>constant</ct:Algorithm>
                        <ct:InterpIndepVar>
                            <ct:SymbRef symbIdRef="t"/>
                        </ct:InterpIndepVar>
                    </ct:Interpolation>
                </ct:Assign>
            </ct:Variable>

\end{lstlisting}

is given as a lookup table defined in the \emph{TrialDesign}
\begin{table}[H]
\begin{center}
\scriptsize
\begin{tabular}{lllll}
  \hline
ID & TIME & EPOCH & ARM & Cc \\
  \hline
1& 10& 1& 1 & 10 \\
1& 20& 1& 1 & 12 \\
... & ... & ... & ... & ...\\
1& 10& 2& 1 & 6 \\
1& 20& 2& 1 & 12 \\
... & ... & ... & ... & ...\\
2& 10& 1& 1 & 2 \\
2& 20& 1& 1 & 4 \\
2& 40& 1& 1 & 7 \\
... & ... & ... & ... & ...\\
  \hline
\end{tabular}
\caption{Lookup table, first few data records from \emph{Cc\_lookupTable.csv} dataset.}
%\label{tab:lookupData}
\end{center}
\end{table}

\item
Interpolation type chosen from a list: \emph{\{constant, linear, spline, pchip\}}
\item
E -- effect model given by an ODE
\begin{align}
\frac{dE}{dt}=Rin \times \Big(1- \frac{Imax \times Cc}{Cc+IC_{50}}\Big) - kout \times E		\nonumber
\end{align}
\lstset{language=XML}
\begin{lstlisting}
            <!-- EFFECT MODEL with TARGET 'Cc' -->
            <ct:DerivativeVariable symbId="E" symbolType="real">
                <ct:Description>PCA</ct:Description>
                <ct:Assign>
                    <Equation xmlns="http://www.pharmml.org/2013/03/Maths">
                        ...
                           <Binop op="times">
                                 <ct:SymbRef blkIdRef="p1" symbIdRef="Imax"/>
                                 <ct:SymbRef symbIdRef="Cc"/>
                           </Binop>
                        ...
\end{lstlisting}
\end{itemize}
In order to make the connection between the column names in the dataset and the according variables in the \xelem{TrialDesign} and \xelem{ModelDefinition}, following mappings are defined
\begin{itemize}
\item
Individual mapping -- to identify the subject in the dataset.
\item
Arm mapping -- to assign subjects to the appropriate arm.
\item
Epoch mapping -- to assign subjects to the appropriate epoch.
\item
Variable mapping -- to assign the variable with respect to which the interpolation is performed.
\item
Target mapping -- to assign the interpolated variable.
\end{itemize}
as shown in the first part of the following code:

CHENGE-TO-NEW-TYPE CHENGE-TO-NEW-TYPE CHENGE-TO-NEW-TYPE CHENGE-TO-NEW-TYPE CHENGE-TO-NEW-TYPE CHENGE-TO-NEW-TYPE CHENGE-TO-NEW-TYPE CHENGE-TO-NEW-TYPE CHENGE-TO-NEW-TYPE CHENGE-TO-NEW-TYPE

\lstset{language=XML}
\begin{lstlisting}
            <Activity oid="d1">
               <LookupTable>
               
                    <!-- MAPPING -->
                    <!-- INDIVIDUAL mapping -->
                    <LookupTableIndividualMapping> 
                        <ds:ColumnRef columnIdRef="ID"/>
                    </LookupTableIndividualMapping>
		    <!-- ARM mapping -->
                    <LookupTableArmMapping>
                        <ds:ColumnRef columnIdRef="ARM"/>
                    </LookupTableArmMapping>
                    <!-- EPOCH mapping -->
                    <LookupTableEpochMapping>
                    <ds:ColumnRef columnIdRef="EPOCH"/>
                    </LookupTableEpochMapping>
                    <!-- VARIABLE  mapping -->
                    <LookupTableVariableMapping>
                        <ds:ColumnRef columnIdRef="TIME"/>
                        <ct:SymbRef symbIdRef="t"/>
                    </LookupTableVariableMapping>
                    <!-- TARGET mapping -->
                    <LookupTableTargetMapping inputTarget="variable">                       		
                        <ds:ColumnRef columnIdRef="Cc"/>
                        <ct:SymbRef blkIdRef="sm1" symbIdRef="Cc"/>
                        <ds:Interpolation>
                            <ds:Algorithm>linear</ds:Algorithm>
                        </ds:Interpolation>
                    </LookupTableTargetMapping>

                    <!-- DATASET -->
                    <ds:DataSet>
                        <ds:Definition>
                            <ds:Column columnId="ID" valueType="id" columnNum="1"/>
                            <ds:Column columnId="TIME" valueType="real" columnNum="2"/>
                            <ds:Column columnId="ARM" valueType="id" columnNum="3"/>
                            <ds:Column columnId="EPOCH" valueType="real" columnNum="4"/>
                            <ds:Column columnId="Cc" valueType="real" columnNum="5"/>
                        </ds:Definition>
                        <ds:ImportData oid="importData">
                            <ds:name>Cc_lookupTable</ds:name>
                            <ds:url>file:///../../datasets/</ds:url>
                            <ds:format>CSV</ds:format>
                            <ds:delimiter>COMMA</ds:delimiter>
                        </ds:ImportData>
                    </ds:DataSet>
                </LookupTable>                
            </Activity>
\end{lstlisting}
The second part of the above listing, \xelem{DataSet}, follows the same rules as described in the section about the externalised datasets.

\subsection{Case C2}
In this case we model is defined as 

\begin{itemize}
\item
Cc -- measured drug concentration encoded as a variable in the model which is defined by the user-defined interpolation function, here \emph{MyInterpolationFunction(...)}
\lstset{language=XML}
\begin{lstlisting}
            <!-- USER-DEFINED TARGET FOR LOOKUP DATA REFERENCE -->
            <ct:Variable symbolType="real" symbId="Cc">
                <ct:Assign>
                    <math:Equation>
                        <math:FunctionCall>
                            <ct:SymbRef symbIdRef="MyInterpolationFunction"/>
                            <!-- SKIP -->
                        </math:FunctionCall>
                    </math:Equation>
                </ct:Assign>
            </ct:Variable>
\end{lstlisting}

is given as a lookup table defined in the \emph{TrialDesign}
\begin{table}[H]
\begin{center}
\scriptsize
\begin{tabular}{lllll}
  \hline
ID & TIME & EPOCH & ARM & Cc \\
  \hline
1& 10& 1& 1 & 10 \\
1& 20& 1& 1 & 12 \\
... & ... & ... & ... & ...\\
1& 10& 2& 1 & 6 \\
1& 20& 2& 1 & 12 \\
... & ... & ... & ... & ...\\
2& 10& 1& 1 & 2 \\
2& 20& 1& 1 & 4 \\
2& 40& 1& 1 & 7 \\
... & ... & ... & ... & ...\\
  \hline
\end{tabular}
\caption{Lookup table, first few data records from \emph{Cc\_lookupTable.csv} dataset.}
%\label{tab:lookupData}
\end{center}
\end{table}

\item
Interpolation function for \emph{Cc}

\begin{align}
Cc(t) =     \left\{ \begin{array}{rcl}
         k_{i-1} + \frac{k_i - k_{i-1}}{t_i - t_{i-1}} (t - t_{i-1}) & \mbox{for} & t_{i-1} <= t < t_i, i = 1\ldots7 \nonumber \\ 
         0 & \mbox{for} & else \nonumber
             \end{array}\right.
\end{align}

is defined in the \emph{StructuralModel}, e.g.

\lstset{language=XML}
\begin{lstlisting}
    <!-- FUNCTION DEFINITION -->
    <FunctionDefinition xmlns="http://www.pharmml.org/2013/03/CommonTypes"
        symbId="MyInterpolationFunction" symbolType="real">
        <!-- SKIP -->
    </FunctionDefinition>
\end{lstlisting}

\item
E -- effect model given by an ODE
\begin{align}
\frac{dE}{dt}=Rin \times \Big(1- \frac{Imax \times Cc}{Cc+IC_{50}}\Big) - kout \times E		\nonumber
\end{align}
\lstset{language=XML}
\begin{lstlisting}
            <!-- EFFECT MODEL with TARGET 'Cc' -->
            <ct:DerivativeVariable symbId="E" symbolType="real">
                <ct:Description>PCA</ct:Description>
                <ct:Assign>
                    <Equation xmlns="http://www.pharmml.org/2013/03/Maths">
                        ...
                           <Binop op="times">
                                 <ct:SymbRef blkIdRef="p1" symbIdRef="Imax"/>
                                 <ct:SymbRef symbIdRef="Cc"/>
                           </Binop>
                        ...
\end{lstlisting}
\end{itemize}
The complete code for the user-defined function, without the repetitive middle section for time points \emph{t2 ... t5}, is given in the following listing
\lstset{language=XML}
\begin{lstlisting}
    <!-- FUNCTION DEFINITION -->
    <FunctionDefinition xmlns="http://www.pharmml.org/2013/03/CommonTypes"
        symbId="MyInterpolationFunction" symbolType="real">
        <FunctionArgument symbId="time" symbolType="real"/>
        <FunctionArgument symbId="k0" symbolType="real"/>
        <FunctionArgument symbId="k1" symbolType="real"/>
        <!-- SKIP -->    
        <FunctionArgument symbId="k7" symbolType="real"/>
        <FunctionArgument symbId="t0" symbolType="real"/>
        <FunctionArgument symbId="t1" symbolType="real"/>
        <!-- SKIP -->    
        <FunctionArgument symbId="t7" symbolType="real"/>

        <Definition>
            <Equation xmlns="http://www.pharmml.org/2013/03/Maths">
                <Piecewise>
                    <Piece>
                        <Binop op="plus">
                            <ct:SymbRef symbIdRef="k0"/>
                            <Binop op="times">
                                <Binop op="divide">
                                    <Binop op="minus">
                                        <ct:SymbRef symbIdRef="k1"/>
                                        <ct:SymbRef symbIdRef="k0"/>
                                    </Binop>
                                    <Binop op="minus">
                                        <ct:SymbRef symbIdRef="t1"/>
                                        <ct:SymbRef symbIdRef="t0"/>
                                    </Binop>
                                </Binop>
                                <Binop op="minus">
                                    <ct:SymbRef symbIdRef="time"/>
                                    <ct:SymbRef symbIdRef="t0"/>
                                </Binop>
                            </Binop>
                        </Binop>
                        <Condition>
                            <LogicBinop op="and">
                                <LogicBinop op="geq">
                                    <ct:SymbRef symbIdRef="time"/>
                                    <ct:SymbRef symbIdRef="t0"/>
                                </LogicBinop>
                                <LogicBinop op="lt">
                                    <ct:SymbRef symbIdRef="time"/>
                                    <ct:SymbRef symbIdRef="t1"/>
                                </LogicBinop>
                            </LogicBinop>
                        </Condition>
                    </Piece>
                    <!-- SKIP -->                    
                    <Piece>
                        <Binop op="plus">
                            <ct:SymbRef symbIdRef="k6"/>
                            <Binop op="times">
                                <Binop op="divide">
                                    <Binop op="minus">
                                        <ct:SymbRef symbIdRef="k7"/>
                                        <ct:SymbRef symbIdRef="k6"/>
                                    </Binop>
                                    <Binop op="minus">
                                        <ct:SymbRef symbIdRef="t7"/>
                                        <ct:SymbRef symbIdRef="t6"/>
                                    </Binop>
                                </Binop>
                                <Binop op="minus">
                                    <ct:SymbRef symbIdRef="time"/>
                                    <ct:SymbRef symbIdRef="t6"/>
                                </Binop>
                            </Binop>
                        </Binop>
                        <Condition>
                            <LogicBinop op="and">
                                <LogicBinop op="geq">
                                    <ct:SymbRef symbIdRef="time"/>
                                    <ct:SymbRef symbIdRef="t6"/>
                                </LogicBinop>
                                <LogicBinop op="lt">
                                    <ct:SymbRef symbIdRef="time"/>
                                    <ct:SymbRef symbIdRef="t7"/>
                                </LogicBinop>
                            </LogicBinop>
                        </Condition>
                    </Piece>                
                </Piecewise>
            </Equation>
        </Definition>
    </FunctionDefinition>
\end{lstlisting}
The following listing shows how the user-defined interpolation function \emph{MyInterpolationFunction} is then referenced in the model
\lstset{language=XML}
\begin{lstlisting}
        <StructuralModel blkId="sm1">

            <!-- TARGET FOR LOOKUP DATA REFERENCE -->
            <ct:Variable symbolType="real" symbId="Cc">
                <ct:Assign>
                    <math:Equation>
                        <math:FunctionCall>
                            <ct:SymbRef symbIdRef="MyInterpolationFunction"/>
                            <math:FunctionArgument symbId="time">
                                <ct:SymbRef symbIdRef="t"/>
                            </math:FunctionArgument>
                            <math:FunctionArgument symbId="k0">
                                <ct:SymbRef blkIdRef="pm1" symbIdRef="k0"/>
                            </math:FunctionArgument>
                            <!-- SKIP -->     
                            <math:FunctionArgument symbId="t7">
                                <ct:SymbRef blkIdRef="pm1" symbIdRef="t7"/>
                            </math:FunctionArgument>
                        </math:FunctionCall>
                    </math:Equation>
                </ct:Assign>
            </ct:Variable>
\end{lstlisting}

The last part is the \emph{Lookup table} definition and mapping, identical to the previous example with the exception that no interpolation type is define for the \emph{Target} variable:

\lstset{language=XML}
\begin{lstlisting}
            <Activity oid="d1">
               <LookupTable>
               
                    <!-- MAPPING -->
                    <!-- SKIP --> 
			<!-- this part is identical to the code in C1 case -->
                    <!-- SKIP --> 

                    <!-- TARGET mapping -->
                    <LookupTableTargetMapping inputTarget="variable">                       		
                        <ds:ColumnRef columnIdRef="Cc"/>
                        <ct:SymbRef blkIdRef="sm1" symbIdRef="Cc"/>
                    </LookupTableTargetMapping>

                    <!-- DATASET -->
                    <!-- SKIP --> 
			<!-- this part is identical to the code in C1 case -->
                    <!-- SKIP --> 

                </LookupTable>                
            </Activity>
\end{lstlisting}



\bibliography{pharmml-specification}
\end{document}






%TEMPLATES 
%% 1. Template for table with figures
%\begin{figure}[htbp]
%\centering
%\begin{tabular}{cc}
% \includegraphics[width=80mm]{pics/pic1} & 
% \includegraphics[width=80mm]{pics/pic2} \\
% \includegraphics[width=80mm]{pics/pic3} &
% \includegraphics[width=80mm]{pics/pic4}
%\end{tabular}
%\caption{about the figure}
%\label{figTable:labelText}
%\end{figure}

%\begin{table}[ht]
%\begin{center}
%\begin{tabular}{rrrrrrrrrrr}
%  \hline
% & 1 & 2 & 3 & 4 & 5 & 6 & 7 & 8 & 9 & 10 \\ 
%  \hline
%1 & 0.24 & -1.47 & -0.56 & 0.24 & 0.71 & 1.23 & 0.44 & 0.40 & 1.10 & 1.84 \\ 
%   \hline
%\end{tabular}
%\end{center}
%\end{table}
 

%\begin{figure}[htb!]
%\centering
%  \includegraphics[width=105mm]{}
% \caption{}
% \label{fig:myplot}
%\end{figure}

%PIECE-WISE
%f(z) =     \left\{ \begin{array}{rcl}
%         value1 & \mbox{for} & condition1 \\ 
%         value1 & \mbox{for} & condition1
%             \end{array}\right.

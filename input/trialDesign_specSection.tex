%%%%%%%%%%%%%%%%%%%%%%%%%%%%%%%%%%%%%%%%%%%%%%%%%%%%%%%%%%%%%%%%%
%%%%%%%%%%%%%%%%%%%%%%%%%%%%%%%%%%%%%%%%%%%%%%%%%%%%%%%%%%%%%%%%%
\chapter{Trial design model}
\label{sec:CTS}
\label{maths:epoch-defn}

%%%%%%%%%%%%%%%%%%%%%%%%%%%%%%%%%%%%%%%%%%%%%%%%%%%%%%%%%%%%%%%%%
\section{Introduction}

In tools such as NONMEM and MONOLIX it has been common practice to encode the trial design in a data file.
More specifically parts of the overall problem, e.g. the structural model and the parameters are explicitly
encoded in the model file, while other, design related, parts are encoded in the data file. This implies
that the software tool, when processing the data file, must associate the data items with model variables
in order to recognise all characteristics of a study, such as subject-specific measurement time points
and values, covariates, variability levels, etc. This has clear disadvantages for simulation purposes,
because it means that when wanting to change only the design, the data file has to be changed, an error
prone and time consuming work. This is clearly not an ideal situation and PharmML addresses this issue.

It is important to stress that we have based a major part of the trial design on one of the standards 
developed by CDISC "a global, open, multidisciplinary, non-profit organisation that has established 
standards to support the acquisition, exchange, submission and archive of clinical research data 
and metadata" \cite{CDICS:2013}. Over the recent years, this organisation has worked out a set of 
standards widely used in the medical and pharmaceutical research, both in academic and commercial centres. 
Using this standard gives us the reassurance that \pharmml will be able
to represent all trial structures that we are likely to encounter. 


\subsection{Sources of clinical data}

Clinical trials are carefully structured and can vary considerably in
their complexity. Typically, a trial will have one or more arms with
each arm containing one or more treatment regimens and observation
protocols. Individuals are then allocated to each arm from a population
of subjects who have been screened for their suitability to participate
in the trial. In \pharmml we describe the structure and population of a
trial explicitly in a dedicated section. This differs from some other
approaches, but we feel it makes the clinical trial much clearer to
document and easier to encode computationally.


%Typically, in tools such as NONMEM and MONOLIX, the trial design is
%encoded within a tabular data-file. This file contains dosing
%information, covariates and experimental observations for each
%individual in the trial. In addition the definition of the structure
%of the clinical trial is also embedded into this table, meaning that
%the data-file contains a lot of redundant information.

The clinical data comes usually from different sources (and formats) and 
in \pharmml we distinguish this information by separating it into the following three
classes of data dependent on their origin\footnote{It is interesting to note that the developers of
PharML had a similar insight and organised data in a similar
way \cite{NLMEcons:2008}.}:
%
\begin{description}
\item[Population] The attributes of the individuals in the study: the
  population in the population model. Each individual has a weight,
  an age, a gender and numerous other properties that may or may not
  be modelled as covariates in a given model. Importantly, the 'arm' 
  membership of every subject/patient is part of this information.
  In addition, these properties may change over time. 
\item[Dosing] When and how a drug or drugs are administered to the
  individuals in the trial.
\item[Measurements] These are the observations taken from each
  individual at specific times during the study. Such measurements
  provide the objective data used during parameter estimation and are
  typically the outputs calculated during a simulation.
\end{description}

%%%%%%%%%%%%%%%%%%%%%%%%%%%%%%%%%%%%%%%%%%%%%%%%%%%%%%%%%%%%%%%%%
\section{Trial Design}

By separating out these classes of information you can see that the
information we need to define for a clinical trial is as follows:
%
\begin{description}
\item[Structure] The organisation of the trial, how the subjects
  are grouped into different treatment groups and what the dosing
  regimen is within these treatment groups.
\item[Population] As above, the properties specific to the individuals,
  including those that vary over time.
\item[Individual Dosing] This is related to the treatment regimens
  described in the trial structure, but describes the dosing history
  for each individual in the study.
\end{description}
%
The measurement data is then used exclusively for estimation and
is encoded in the third major building block of \pharmml, in the 'Modelling 
Steps'.

%%%%%%%%%%%%%%%%%%%%%%%%%%%%%%%%%%%%%%%%%%%%%%%%%%%%%%%%%%%%%%%%%
\subsection{Structure}
\label{subsec:TrialStructure}

To define the Trial Structure we have reused, almost verbatim, the
CDISC Study Design Model \cite{CDISC:2011a}, which is an
XML representation of a clinical trial. Figure \ref{fig:CellSegmentEpochArmEvent_concept}
below shows how the CDISC trial structure is organised. It has six main components:
\begin{description}
\item[Epoch] The epoch defines a period of time during the study which
  has a purpose within the study. For example a washout or a treatment
  window. In CDISC Epochs can describe screening or follow-up periods,
  which are out of the scope of \pharmml. An epoch is usually defined
  by a time period.
\item[Arm] The arm represents a path through the study taken by a
  subject. An arm is composed of a study cell for each epoch in the study.
\item[Cell] The study cell describes what is carried out during an
  epoch in a particular arm. There is only one cell per epoch.
\item[Segment] The segment describes a set of planned observations and
  interventions, which may or may not involve treatment. Note that in
  \pharmml our definition is more limited and we only describe
  treatments. A segment can contain one or more activities.
\item[Activity] The activity is an action that is taken in the
  study. Here it is typically a treatment regimen or a washout.
\item[StudyEvent] A study event describes the collection of
  information about a particular individual. In CDISC this can be
  information captured during screening or other non-treatment phases
  of the clinical trial. But here we restrict it to capturing
  observations during the treatment. In \pharmml this is how we
  capture occasions.
\end{description}
\begin{figure}[htb]
\centering
\includegraphics[height=0.2\textheight]{pics/CellSegmentEpochArmEvent_concept}%
\caption{Overview of the basic concept of the Trial Structure: arm,
epoch, event and a cell with segment and activity/treatment as used in the CDISC Study
  Design Model. See the next figure for an example.}
\label{fig:CellSegmentEpochArmEvent_concept}
\end{figure}
\begin{figure}[htb]
\centering
\includegraphics[height=0.35\textheight]{pics/templateTrialDesign}%
\caption{An example of a study with two arms and three epochs: Screen,
Treatment and Follow Up. A segment can contain more then one Activity/Treatment
as can be seen in \var{Cell2} with one Pre-Treatment \var{P} and one Treatment \var{X}.
In this particular example no Event/Occasion is specified.}
\label{fig:templateTrialDesign}
\end{figure}
Figure \ref{fig:templateTrialDesign} shows one such example of a hypothetical trial
consisting of two arms and three epochs: \var{Screen}, \var{Treatment} and
\var{Follow Up}. There are accordingly six cells and segments, each consisting of one
or two activities. \var{Cell2} has the most complex structure carrying two
subsequent treatments, Pre-Treatment \var{P} and Treatment \var{X}. In this case
no Events/Occasions are specified which is an optional element in the design.

Examples of how a trial design is encoded in \pharmml can be found in
the examples (see chapter \ref{chap:worked-egs}).

\paragraph{Variability}
There is one aspect regarding the random variability worth mentioning
here. The \xelem{Population} block carries the information about subject 
level variability and those variability levels above the subject, see next section for more details.  
In the \xelem{Structure} element we encode the variability which is located below the
subject. This is typically known as \textit{inter-occasion variability} but deeper levels
are allowed in theory. The reader is referred to the section \ref{sec:variabilityModel} where the 
full nested hierarchy of the random variability discussed in detail.


%%%%%%%%%%%%%%%%%%%%%%%%%%%%%%%%%%%%%%%%%%%%%%%%%%%%%%%%%%%%%%%%%
\subsection{Population}
\label{subsec:TrialPopulation}

This is the second major element of the trial design description where we:
\begin{itemize}
\item
describe the individuals in the study
\item
describe their attributes (such as weight, gender, etc.)
\item
assign them to an arm of the study, but also
\item
indicate if variability at and/or below the subject level is to be defined, which is the case in the majority of models (if omitted
then this means that we explicitly consider a setup without any random variability,
which is the case for the na\"{\i}ve pooled data method), and
\item
indicate their country or centre membership to define higher levels of variability above the subject level.
\end{itemize}
We define the possible attributes of all individuals using the \var{IndividualTemplate} block and then map
each individual to this template using a \var{Dataset} block.

%\begin{description}
% \item[]
%
% \item[Epoch]
%
% \item[Epoch]
%
%\end{description}

%%%%%%%%%%%%%%%%%%%%%%%%%%%%%%%%%%%%%%%%%%%%%%%%%%%%%%%%%%%%%%%%%
\subsection{Individual dosing}
\label{subsec:TrialSIndivDosing}

The two previous sections on \var{Structure} and \var{Population} described information,
which is sufficient to encode e.g. a simple simulation task. Specifically, when the dosing is 
equal among the patients then this can be encoded in the \var{Structure} part of the schema with
one or more dose amounts and one or more dosing times for all. However, in most cases, especially
when we deal with real clinical data this is not so straighforward. Every patient will have its own specific 
amounts and dosing times.

For an estimation task we always need to provide experimental data for each dosing activity 
relevant to the particular case. With the current structure we can provide individual dosing 
information for every dosing activity defined in the \var{Structure} part (see \ref{subsec:TrialStructure}).

The structure of this part is similar to that used for \var{Population} in that first
a table template is defined with all relevant columns, i.e. \var{ID}, \var{TIME}, and \var{DOSE},
which is then populated with individual dosing data.


%%%%%%%%%%%%%%%%%%%%%%%%%%%%%%%%%%%%%%%%%%%%%%%%%%%%%%%%%%%%%%%%%%
%\subsection{PharmML implementation}
%\label{subsec:TrialSIndivDosingPharmML}
%
%Dependent on the task to be implemented, the information to be stored
%in the Trial Design block will differ. For example an estimation task requires all three
%elements discussed above, which in \pharmml have the intuitive names
%\xelem{Structure}, \xelem{Population} and \xelem{IndividualDosing}
%(see figure \ref{fig:simEstTasks_trialList}). For a simulation task only the first two are
%necessary.
%
%
%\begin{figure}[htb]
%\centering
%\includegraphics[width=0.7\linewidth]{pics/simEstTasks_trialList}%
%\caption{PharmML building blocks used in the definition of a trial design.}
%\label{fig:simEstTasks_trialList}
%\end{figure}


%%%%%%%%%%%%%%%%%%%%%%%%%%%%%%%%%%%%%%%%%%%%%%%%%%%%%%%%%%%%%
%\begin{figure}[htbp!]
%\centering
%\includegraphics[width=0.8\linewidth]{../pics/cellHierarchy2}
%\caption{General cell hierarchy (left); The root of the hierarchy is the cell which can contain one segment, one epoch and multiple arms. The segment element can have multiple child elements, the activities e.g. treatments or a washout. (right) An example of how it is applied in this example.}
%\label{fig:cellHierarchy}
%\end{figure}

%\section{Design elements and examples}
%\label{sec:CTS_exampleSection} % KEEP THIS LABEL!!!!!!
%The default working mode, as mentioned above, is when all the study characteristics are encoded in the data file. Alternatively, PharmML offers a very flexible structure for the setup of clinical trials. Using only a few basic elements the modeller can compose several types of designs, including parallel design and crossover designs with or without washout or run-in period\footnote{\textit{Run-in} -- do be continued. Definition after \cite{Iverson:2007fk}}, see Figure \ref{fig:CTSfigure1} for examples. The basic elements are:
%
%\begin{itemize}
%\item
%%\textit{Dosing}
%%\begin{itemize}
%%\item
%%one or multiple dosing events can be defined
%%\item
%%dosing type: single dose, sequence/repeated or steady-state dosing
%%\item
%%type of administration: bolus or infusion
%%\item
%%target compartment/variable
%%\item
%%amount and/or infusion rate
%%\item
%%dosing times
%%\end{itemize}
%%\item
%\textit{Treatment} -- describes dosing related data
%\begin{itemize}
%\item
%Administration/dosing type, \textit{Bolus} or \textit{Infusion}, as single value or a sequence
%\item
%Dosing Times -- as single value or a sequence
%\item
%Dose Amount -- as single value or a sequence
%\item
%Dosing target
%\begin{itemize}
%\item
%Target variable (indicates the target compartment) -- for ODE coded structural models, e.g. $Ad$ the drug amount in the depot compartment (see section \ref{sec:structuralModel})
%\item
%Dose variable -- for explicit algebraic equations with a dose amount variable, e.g. $D$ as in $C(t)= \frac{D}{V}e^{-k(t-t_D)}$
%\end{itemize}
%\end{itemize}
%\item
%\textit{Treatment Epoch\footnote{\textit{Epoch} -- Interval of time in the planned conduct of a study. An epoch is associated with a purpose (e.g., screening, randomization, treatment, follow-up), which applies across all arms of a study. NOTE: Epoch is intended as a standardised term to replace: period, cycle, phase, stage. Definition after \cite{CDICS:2011}}} -- basic time interval within a study
%\begin{itemize}
%\item
%Epoch name, e.g. \textit{Treatment\_A} or \textit{Washout}\footnote{\textit{Washout} -- A period in a clinical study during which subjects receive no treatment for the indication under study and the effects of a previous treatment are eliminated (or assumed to be eliminated).} or \textit{Run-in}
%\item
%\textit{Start time} and \textit{end time} of an epoch -- this sets reference time frame for any possible occasion within an epoch
%\item
%Occasion(s) -- defined by
%\begin{itemize}
%\item
%\textit{start time} and \textit{end time} of each occasion relative to epoch time frame
%\item
%level identifier
%\end{itemize}
%\end{itemize}
%\item
%\textit{Group} -- basic grouping structure for subjects usually containing one or more treatment epochs
%\begin{itemize}
%\item
%a sequence of epochs for a number of subjects, e.g. [\textit{Epoch\_A}, \textit{Washout}, \textit{Epoch\_B}]
%\item
%number of subjects
%\end{itemize}
%\end{itemize}
%
%\paragraph{Note 1} The name of a particular trial design, e.g. \textit{parallel} or \textit{crossover with washout} is not part of the language. It can be provided as annotation of the trial design model using an appropriate ontology.
%\paragraph{Note 2} The 'Washout' epoch means complete reset of all variables and is defined without start and end times, but this restrictions will be released in an upcoming specification.
%\paragraph{Note 3} The support for variability levels is limited in this specification of PharmML. More specifically, it can only encode occasions definable using start and end times for each occasion. Levels above the subject reference level, such as 'country' or 'centre' cannot be encoded.
%


%\begin{figure} % [htb!]
%\centering
%\begin{tabular}{c}
% \includegraphics[width=140mm]{ClinicalDesignPatterns_version3}
% \end{tabular}
%\caption{Examples for basic clinical trial design types and configurations. In general any combination of the basic structural elements, i.e. \textit{Dosing}, \textit{Treatment}, \textit{Treatment Epoch} and \textit{Group}, can be created. Based on \cite{Lavielle:2012} and \cite{Wang:2006}.}
%\label{fig:CTSfigure1}
%\end{figure}


%%%%%%%%%%%%%%%%%%%%%%%%%%%%%%%%%%%%%%%%%%%%%%%%%%%%%%%%%%%%%%%%%
%\subsection{Example 1 -- Basic crossover with washout}
%This example describes a crossover design\footnote{\textit{Crossover design} -- Each subject is allocated to a sequence of treatments across a number of treatment periods. Within crossover trials, sequences can include all possible treatments or a subset of these (incomplete block). Definition after \cite{CDICS:2011}} with washout, see example F in Figure \ref{fig:CTSfigure1}.
%
%\subsubsection{Trial design model}
%
%-- Treatment definition
%\begin{align*}
%Treatment\_A: & AdministrationType = \text{[OR bolus, IV bolus]}  \\
%& DoseTime = [6:24:72, 0:24:72]   \\
%& DoseSize = [50, \;\;100]   \\
%& DoseVariable = \text{[D, \;\;D]}  \\
%Treatment\_B: & AdministrationType = \text{OR bolus}  \\
%& DoseTime = 0:24:72   \\
%& DoseSize = 150   \\
%& DoseVariable = \text{D}
%\end{align*}
%
%-- Treatment Epoch definition
%\begin{align*}
%Epoch\_1: & Treatment\_A  \\
%& TreatmentStart = 0  \\
%& TreatmentEnd = 100  \\
%Epoch\_2: & Treatment\_B  \\
%& TreatmentStart = 0  \\
%& TreatmentEnd = 100  \\
%Epoch\_3: & Washout
%\end{align*}
%
%-- Group definition
%\begin{align*}
%Group 1: 	& TreatmentSeq = \text{[Epoch\_1, Epoch\_3, Epoch\_2]}	 \\
%			& GroupSize = 40		 \\
%Group 2: 	& TreatmentSeq = \text{[Epoch\_2, Epoch\_3, Epoch\_1]}  \\
%			& GroupSize = 60
%\end{align*}
%
%
%%%%%%%%%%%%%%%%%%%%%%%%%%%%%%%%%%%%%%%%%%%%%%%%%%%%%%%%%%%%%%%%%
%\subsection{Example 2 -- Complex crossover trial with washout, 4 groups}
%See example H in Figure \ref{fig:CTSfigure1}.
%
%\subsubsection{Trial design model}
%
%-- Treatment definition
%\begin{align*}
%T1: & AdministrationType = \text{[OR1, OR2, IV]}  \\
%& DoseSize = [50 \;\;\;100 \;\;\;100];   \\
%& DoseTime = [6:24:72, 0:24:72, 12:24:72];   \\
%& DoseVariable = \text{[D, \;\;D, \;\;D]}  \\
%T2: & AdministrationType = \text{OR1}  \\
%& DoseSize = 150;   \\
%& DoseTime = [0:24:72];    \\
%& DoseVariable = \text{D}  \\
%T3: & AdministrationType = \text{OR2}  \\
%& DoseSize = 150;   \\
%& DoseTime = [0:24:72];   \\
%& DoseVariable = \text{D}  \\
%T4: & AdministrationType = \text{IV}  \\
%& DoseSize = 150;   \\
%& DoseTime = [0:24:72];   \\
%& DoseVariable = \text{D}
%\end{align*}
%
%
%-- Treatment Epoch definition
%\begin{align*}
%Epoch\_1: & T1  \\
%& TreatmentStart = 0  \\
%& TreatmentEnd = 100  \\
%Epoch\_2: & T2  \\
%& TreatmentStart = 0  \\
%& TreatmentEnd = 100  \\
%Epoch\_3: & T3  \\
%& TreatmentStart = 0  \\
%& TreatmentEnd = 100  \\
%Epoch\_4: & T4  \\
%& TreatmentStart = 0  \\
%& TreatmentEnd = 100  \\
%Epoch\_5: & Washout
%\end{align*}
%
%
%-- Group definition
%\begin{align*}
%Group 1: 	& TreatmentSeq = \text{[Epoch\_1, Epoch\_5, Epoch\_2, Epoch\_5, Epoch\_3]}	 \\
%			& GroupSize = 40		 \\
%Group 2: 	& TreatmentSeq = \text{[Epoch\_2, Epoch\_5, Epoch\_3, Epoch\_5, Epoch\_4]}  \\
%			& GroupSize = 40		 \\
%Group 3: 	& TreatmentSeq = \text{[Epoch\_4, Epoch\_5, Epoch\_1, Epoch\_5, Epoch\_2]}	 \\
%			& GroupSize = 60		 \\
%Group 4: 	& TreatmentSeq = \text{[Epoch\_4, Epoch\_5, Epoch\_3, Epoch\_5, Epoch\_2]}  \\
%			& GroupSize = 60
%\end{align*}





%%%%%%%%%%%%%%%%%%%%%%%%%%%%%%%%%%%%%%%%%%%%%%%%%%%%%%%%%%%%%%%%%
%\subsubsection{To-Do list}
%
%\paragraph{Crossover - other aspects not covered yet}
%- 2x2 -- covered already \\
%- p x q -- see Table 2 in \cite{Wang:2006} \\
%- Latin square 3x3 \& 4x4
%
%
%%%%%%%%%%%%%%%%%%%%%%%%%%%%%%%%%%%%%%%%%%%%%%%%%%%%%%%%%%%%%%%%%
%\paragraph{Factorial design}
%In a factorial design two or more treatments are evaluated simultaneously through the use of varying combinations of the treatments. The simplest example is the 2x2 factorial design in which subjects are randomly allocated to one of the four possible combinations of two treatments, A and B say. These are: A alone; B alone; both A and B; neither A nor B. Source: \cite{EMA:1998}\\
%- no example available
%
%\paragraph{Other aspects of factorial design to be considered}
%- 2x2, Ix J x K, unbalanced, 'complicated'
%
%
%%%%%%%%%%%%%%%%%%%%%%%%%%%%%%%%%%%%%%%%%%%%%%%%%%%%%%%%%%%%%%%%%
%\paragraph{Cohort study}
%Study of a group of individuals, some of whom are exposed to a variable of interest, in which subjects are followed over time. Cohort studies can be prospective or retrospective. [AMA Manual of Style] See also prospective study.
%
%%%%%%%%%%%%%%%%%%%%%%%%%%%%%%%%%%%%%%%%%%%%%%%%%%%%%%%%%%%%%%%%%
%\paragraph{Prospective study}
%Investigation in which a group of subjects is recruited and monitored in accordance with criteria described in a protocol.
%
%
%%%%%%%%%%%%%%%%%%%%%%%%%%%%%%%%%%%%%%%%%%%%%%%%%%%%%%%%%%%%%%%%%
%\paragraph{Survival}
%Observe N subjects until at most time T (censored observations). Alternatively observe N subjects until at least R events occur.\\
%- no example available
%
%
%%%%%%%%%%%%%%%%%%%%%%%%%%%%%%%%%%%%%%%%%%%%%%%%%%%%%%%%%%%%%%%%%
%\paragraph{Observational}
%Subjects are not randomly allocated to treatment.\\
%- no examples available
%
%
%%%%%%%%%%%%%%%%%%%%%%%%%%%%%%%%%%%%%%%%%%%%%%%%%%%%%%%%%%%%%%%%%
%\paragraph{Dose escalation Methods in Phase I, \cite{Le-Tourneau:2009fk}}
%-- Rule-based designs: \\
%Traditional 3+3 design, Accelerated titration designs, Pharmacologically guided dose escalation\\
%Model-based designs: \\
%-- Modified continual reassessment method, Escalation with overdose control, Time-to-event continual reassessment method, EffTox -- efficacy and toxicity method, TriCRM -- an adaptive continual reassessment method that considers three potential trial outcomes: no efficacy and no toxicity, efficacy only, and toxicity only\\
%
%%%%%%%%%%%%%%%%%%%%%%%%%%%%%%%%%%%%%%%%%%%%%%%%%%%%%%%%%%%%%%%%%
%\paragraph{Additional option/criteria -- from Mike's list}
%Within each of these general types of study there are several possible flavours:\\
%Titration, Forced titration, Target concentration, Adaptive, With stopping rules, With interim analyses\\
% - no examples available
%
%%%%%%%%%%%%%%%%%%%%%%%%%%%%%%%%%%%%%%%%%%%%%%%%%%%%%%%%%%%%%%%%%
%\paragraph{Other classification criteria}
%
%\paragraph{Blinding:}
%open/unblinded, single/double/triple  blinded
%
%\paragraph{Order of study}
%- pxq, IxJxK







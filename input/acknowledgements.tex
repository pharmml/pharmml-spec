\chapter*{Acknowledgements}

 \begin{quote}
{\small
When you know a thing, to hold that you know it; and when you do not know a thing, to allow that you do not know it - this is knowledge.\\
\textit{Confucius, The Confucian Analects}}
\end{quote}

In developing this latest version of \pharmml we are building on work
we documented in the previous version of this specification, the
contributions to which we acknowledged (see
section~\ref{chap:acknow-v0_1}). This latest version has been
developed over a shorter timeframe and during the summer vacation
period so consequently we are acknowledging a smaller number of
people. As ever we would like to acknowledge the input of Marc
Lavielle: the changes to the parameter and error model in this version
we were considerably influenced by his input. During the early stages
of development we sent out a proposal document, asking for feedback
and we are grateful to Mike Smith, Emmanuelle Comets, Duncan Edwards
and Marc Lavielle for the useful feedback they gave to these ideas. We
are also indebted to Paolo Magni for his work encoding models in the
previous version of \pharmml. In doing so he highlighted some
limitations in that version and helped spark some useful discussion
that helped us improved this version. Duncan Edwards also provided
excellent review comments to version 0.1.0 and the changes to the
changes we define data and represent the variability model are all
attributable to him. Roberto Bizzotto helped us to understand the
different flavours of the residual error models and their encoding in
MLXTRAN and NMTRAN. Since the last release there have been a number of
discussions on the DDMoRe WP4 mailing list and we would like to thank
the contributors (most of whom we mentioned already) include Andy
Hooker and Nick Holford. We would like to thank Nick in particular for
asking us to encode the model described by Chan \emph{et al.}\xspace
\cite{Chan:2005fk}. Together with Phylinda Chan, Nick helped us
understand what was a challenging trial design that gave us a great
test case to work on. Vincent Buchheit and his EFPIA colleagues
provided valuable feedback on different trial design aspects.  We
would like to thank Mihai Glont and Raza Ali for
scrutinising the specification during their work on the \ddmore
repository. Raza in particular provided some useful feedback on the
maths section in the language overview chapter. Sarah Keating helped
us understand the complexity of unit consistency checking and although
this feature is not in this version, we believe her experience will
help us get it right in the future.

We'd also like to thank Lutz Harnisch for his leadership of the
\ddmore project and the administrative team at Interface Europe for
their help ensuring its smooth running. Finally we would like to thank
Wendy Aartsen for many things, but specifically for organising the
April consortium meeting, which provided such an excellent forum for
the language specification.

The \ddmore project and consequently this work, is funded by the
Innovative Medicines Initiative (IMI), a large-scale public-private
partnership between the European Union and the pharmaceutical industry
association EFPIA\@. We would like to gratefully acknowledge their
support.



%%% Local Variables: 
%%% mode: latex
%%% TeX-master: "../pharmml-specification"
%%% End: 

\chapter{This Document}
\label{chap:this-document}

% Overview of this spec. How to read it etc.

\section{Overview}

In this specification we describe the \pharmml standard. We define what the language is, how it is encoded,
how it should be understood, and how software can validate it. Initially, we define in chapter \ref{chap:scope}
what is in and out of scope of the current version of \pharmml. Underlying \pharmml is a mathematical model
that describes a pharmacometric model. Understanding this is important if you are to completely understand
the language. In chapter \ref{chap:mathsdefn} we go through this model in detail, providing a detailed
mathematical description of the model and its associated assumptions. Also important for the understanding of
\pharmml is the discussion of trial designs in chapter \ref{sec:CTS}, where the basic assumptions behind
the way trial designs are handled in \pharmml are discussed. Next we move on to
the description of the implementation of \pharmml. In chapter \ref{chap:lang-overview} we describe the
organisation of a \pharmml document and then go on to explain key features of the language, such as variable
scoping, that are important to understand before investigating the language further. Chapter \ref{chap:worked-egs}
explains the details of \pharmml using examples. These examples are designed to highlight specific features of
the language and by taking you through a series of worked examples we hope to help you to fully understand \pharmml.
The final chapter in Part \ref{part:primer} (Chapter \ref{chap:unresolved}) lists a number of unresolved issues.

At the end of this document we have the technical reference (Part \ref{part:techref} on page \pageref{part:techref}).
This provides the fine detail of the technical implementation of \pharmml, including the XML Schema design and
the rules to which a correct \pharmml document must conform. This is reference material and not intended to be
read like a novel  from start to finish.

To conclude, helping you to understand \pharmml is the main goal of this specification. We want you to use this
document to review and critique \pharmml. We want you to suggest improvements to the language. Most of all
we want you to use it.

\section{How to read this Specification}

\pharmml is a language of benefit to pharmacometric modellers, but ironically is not designed to be read by
them when in every day use. We expect software support for \pharmml to be developed by software engineers,
who do not have a deep understanding of pharmacometrics and modelling. Therefore, the challenge for us in
drafting this specification is to satisfy both readerships: modellers and software engineers. To help, we
have come up with the following advice on how each readership could read this specification.

If you are a pharmacometric modeller or mathematician then you will want to start with the mathematical
definition of \pharmml in chapter \ref{chap:mathsdefn} (page \pageref{chap:mathsdefn}). From there you
may want to skim the language overview (chapter \ref{chap:lang-overview}) before working through the
examples in chapter \ref{chap:worked-egs} (page \pageref{chap:worked-egs}). You may want to revisit
chapter \ref{chap:lang-overview} (page \pageref{chap:lang-overview}) after reading the examples.

If you are a software engineer then we recommend that you start with the language overview in chapter
\ref{chap:lang-overview} (page \pageref{chap:lang-overview}). After this, work through the examples
in chapter \ref{chap:worked-egs} (page \pageref{chap:worked-egs}) and try to understand the language
features in this way. Those of you with a strong mathematics background will find it helpful to read
through the mathematical definition as well (chapter \ref{chap:mathsdefn}). Finally, when it comes to
implementing \pharmml support, you will find the technical reference (Part \ref{part:techref} on
page \pageref{part:techref}) very important, but only after you have an understanding of \pharmml
from Part \ref{part:primer}.

%\section{Conventions used in this document}

%I'm sure there are some. We should put something here!!!!

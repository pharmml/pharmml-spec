\chapter{\pharmml Version 0.1.0}

This appendix includes information about the previous version of \pharmml. The language specification evolves over time and involves many individuals who's participation in the development of the language also changes between releases. In this chapter we record those who contributed to version 0.1.0 of \pharmml and provide information about what has changed.

\section{Acknowledgements}
\label{chap:acknow-v0_1}

 \begin{quote}
{\small
Nothing is original. Steal from anywhere that resonates with inspiration or fuels your imagination. Select only things to steal from that speak directly to your soul. If you do this, your work (and theft) will be authentic. Authenticity is invaluable; originality is non-existent. And don't bother concealing your thievery -- celebrate it if you feel like it. In any case, always remember what Jean-Luc Godard said: "It's not where you take things from -- it's where you take them to." \\
\textit{Jim Jarmusch}}
\end{quote}

In developing \pharmml we are indebted to the many individuals within
the \ddmore project and other colleagues who contributed to the
development of the standard and this specification document that
describes it. While we have had help and assistance from many people
we would like to highlight the contributions of some key
individuals. Marc Lavielle has helped us by the clarity with which he
has described the mathematical underpinnings of the type of
pharmacometric model described here. He has patiently answered our
many questions and has always been quick to review documents. Mats
Karlsson and Lutz Harnisch have provided us with support throughout
the project and Lutz in particular has been an active and challenging
contributor at our workshops. Andrea Mari has provided us with many
helpful suggestions for the design of \pharmml and his ideas about
modularity have influenced the design you see here. France Mentr\'{e}
has been a valuable presence during workshops and has provided a calm
reason which has helped make the \pharmml workshops very
productive. Emmanuelle Comets assisted us greatly in the development
if the variability model used in \pharmml. Her deep insight made
something that was opaque to us become very clear and simple. Nick
Holford provided advice and feedback on multiple occasions and
countless use cases indispensable in the process of the language
development.  Mike Smith has been an enthusiastic contributor to this
work and has always been keen to help us during workshops. His help
teaching us NONMEM and helping understand real-world modelling
scenarios has certainly informed this work. Paolo Magni provided
expertise in many aspects of population modelling and statistics and
Roberto Bizzotto consulted us on various occasions regarding NMTRAN\@.
Duncan Edwards provided a number of excellent ideas on the trial design model.
And finally Andy Hooker, who was very helpful in the early stages of
the development of \pharmml, but was unavailable over the past few
months. His sharp eye spotted some limitations in early prototypes and
so saved us from ourselves!

Besides these people there
are have many more people who have attended workshops and/or provided us
with input otherwise. They are: 
Chris Franklin,
Eric Blaudez,
Ivan Matthews, 
Jonathan Chard,
Joost N. Kok,
Kaelig Chatel, 
Mateusz Rogalski,
Mihai Glon\c{t}, 
Natallia Kokash,
Richard Kaye, 
Simon Thomas, 
Nadia Terranova,
Vijayalakshmi Chelliah.

Some of our colleagues at EMBL-EBI have been helpful with the
technical aspects of the standard. Sarah Keating's work on \sbml and
libSBML provided us with ``the voice of experience''. Camille Laibe and
Sarala Wimalaratne helped us develop our annotation strategy for
\pharmml. Pierre Grenon helped us
develop the structural model ontology and also helped us develop the
standard library approach described here. Finally SM and MS would like
to thank Henning Hermjackob for his support during recent
re-organisations at the EBI and in helping us attend important
workshops and meeting vital for our work.

We would like to thank the administrative team at Interface
Europe for their help in organising workshops and the smooth running
of the \ddmore project. And not least Wendy Aartsen of Leiden
University for her continued support and good humor in organising
meetings, preparing reports and generally shielding us from
administration and helping us do science!

The \ddmore project and consequently this work, is funded by the
Innovative Medicines Initiative (IMI), a large-scale public-private
partnership between the European Union and the pharmaceutical industry
association EFPIA\@. We would like to gratefully acknowledge their
support.


\section{Changes from version 0.1.0 to 0.2.0}

\tablefirsthead{\toprule Change & Description\\\midrule}
\tablehead{\multicolumn{2}{l}{continued from previous page}\\\toprule%
Change & Description \\\midrule}
\tabletail{\midrule\multicolumn{2}{r}{continues on next page}\\}
\tablelasttail{\bottomrule}
\begin{center}
\small
\begin{mpxtabular}{p{0.38\linewidth} p{0.55\linewidth}}
Refactored XML design & Replaced many attributes with elements and increased reused of global elements - in particular from the CommonTypes schema.\\
Introduced object identifiers (OIDs).\\
Renamed symbol referencing element to SymbRef \\
Refactored variable definitions & Derivatives now defined with the initial conditions.
Improved definition of initial conditions in specification. \\
Redesigned database. & Now defines data in XML. Has nested tables and cannot refer to external files.\\
Introduced integer and real types & Before there was only a scalar type, now we make a distinction.\\
Complete redesign of Trial Design definition. & Now define the trial design explicitly. Cannot do it through data. The trial structure borrows from CDISC.\\
Simplified the estimation step. & It now only maps objective data to the model and describes the estimation operations.\\
Simplified the simulation step& Simulations cannot now driven by a datafile and the definition of repetitions is removed. This feature may be reintroduced in the future.\\
Extended the definition of the Parameter Model & Implicit parameter models are now supported as is a Gaussian parameter model with a non-linear covariate model.\\
Extended the residual error model & Can support much more complex error models including those commonly defined in NONMEM.\\
Removed external inclusions. & To simplify this version of the language all imports of external resources have been removed. This simplifies validation and the aim is to reintroduce this later.\\
Added element identifier & To facilitate referencing by external resources all elements can have an optional id attribute.\\
Pre-release UncertML 3.0 incorporated. & Probability distributions are now defined using an externally maintained resource.\\
\end{mpxtabular}
\end{center}

\section{Changes from version 0.2.0 to 0.2.1}

\tablefirsthead{\toprule Change & Description\\\midrule}
\tablehead{\multicolumn{2}{l}{continued from previous page}\\\toprule%
Change & Description \\\midrule}
\tabletail{\midrule\multicolumn{2}{r}{continues on next page}\\}
\tablelasttail{\bottomrule}
\begin{center}
\small
\begin{mpxtabular}{p{0.33\linewidth} p{0.60\linewidth}}
  Fixes to specification & Fixed errors and typos in the text.\\
  Expanded examples & Added a steady state example.\\
  Additional schema documentation. & Expanded the schema documentation.\\
  Minor schema refactoring. & Refactored the trialDesign.xsd to
  replace
  anonymous types with complex types and so comply with our design
  guidelines.\\
Added the id type. & We realised when writing version 0.2.0 that we
needed a type for identifiers. This made it to the validation section
but was omitted in the XML Schema definition and the examples. This is
now rectified.\\
\end{mpxtabular}
\end{center}

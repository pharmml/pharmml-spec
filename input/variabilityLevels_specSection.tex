\section{Nested hierarchy as the random variability structure}
\label{sec:variabilityModel}
\label{math:variability}

This section describes the variability structure of the random effects and the related naming convention. It is largely based on the discussions and conclusions from the Copenhagen focus meeting \cite{Copenhagen:2013}. Accordingly, in the following we will distinguish: 
\begin{itemize}
\item
(related to the observations) -- \textit{residual variability}, also known as \textit{intra-individual variability} and
\item
(related to the parameters) -- \textit{inter-individual} and \textit{inter-occasion variabilities}
\end{itemize}
The former is described in the section \ref{sec:residualErrorModel}, while the latter is described in this section.

\begin{figure}[htb!]
\centering
  \includegraphics[width=70mm]{Subject-level}
 \caption{Inter-individual variability typically occurring in an experiment, here $\log(V_{i=1\cdots5})$ i.e. values for five subjects, varying around a typical value $\log(V_{pop})$, are shown.}
 \label{fig:subjectLevelVariability}
\end{figure}

\begin{figure}[htb!]
\centering
  \includegraphics[width=120mm]{Subject-occasion-level}
 \caption{Subject variability level, $(0)$, and within-subject (or occasion) variability level, $(-1)$, typically occurring in an experiment, with index $i$ for subjects and $k$ for occasion. Here two subjects only are visualised, each of them having four or three occasions, respectively.}
 \label{fig:subjectOccasionLevelVariability}
\end{figure}

\subsection{Motivation}
One way to look at variability is to consider the following simple experiment: in this experiment, we estimate the volume of distribution in five subjects. Following a drug administration we collect blood samples over a time interval and estimate each subject's PK parameters. The result will be a set of five individual estimates such as those in Figure \ref{fig:subjectLevelVariability}. The values vary around a certain typical/population value. It is apparent that the only variability source is the fact that these are different persons, i.e. we have a rough estimate of the so called \textit{inter-individual} variability.\\
As an extension of this setup, we can now consider different number of occasions, when the PK parameters are estimated for each subject. If we restrict the discussion to two subjects only, each of them having three or four occasions, respectively, we can illustrate the results such those in Figure \ref{fig:subjectOccasionLevelVariability}. Repeatedly performing the same experiment for each subject is equivalent to create an additional level of variability, the \textit{inter-occasion within individual} variability.\\ 
Similarly, one can add e.g. 'country' as a new variability level. If a clinical trial has been conducted in various countries, it is reasonable to ask if the geographic location influences the outcome of the study. 

\subsection{General case}
As a generalisation of the examples described above, one can derive the \textit{nested hierarchy} (also known as \textit{inclusion hierarchy}) of the variability structure of random effects. It can be visualised as a tree or alternatively using a Venn diagram, see Figures \ref{IOVgeneral_tree}, \ref{IOVgeneral_venn}. \\
The tree representation consists of \textit{nodes} and \textit{links} or \textit{edges}. It has the advantage that it visualises the whole structure explicitly from the top level, the \textit{root} node, down to lowest level of the variability. It provides immediate insights needed to understand or to verify the setup of a trial design. However, in case of a very complex structure, with high number of levels and/or subjects, it can become very large, making the tree difficult to represent in a typical document. It this case showing only partial branches will be more helpful, e.g. Figure \ref{IOVgeneral_tree}. On the contrary, the Venn diagram visualises the levels only, and it might be more suited for the complex cases. Usually, the variability structure consists of only one or two levels, e.g. \textit{individual} or  \{\textit{individual}, \textit{occasion}\}, see examples below. 


The \textit{root}, i.e. the top node in the tree structure, stands for the population/typical value of a parameter. Following the current nomenclature, every subsequent variability level is either 'positive' or 'negative' dependent on its position relative to the 'subject level', denoted as 0 -- the level 'zero'. Each level has a covariance matrix associated with it, i.e. 
\begin{itemize}
\item 
$\Omega^{+n}$ -- for levels above the 'zero' level -- their names will vary according to the nature of the levels. For example the variability on country level is called 'between-country variability'.
\item 
$\Omega^0$ -- also called BSV (between subject variability) or IIV (inter-individual variability).
\item 
$\Omega^{-n}$ -- for levels below the 'zero' level -- called WSV (within-subject variability) or IOV (inter-occasion variability).
\end{itemize}
The number of levels will vary dependent on the nature of the study. Cases without or with only positive/negative levels are possible. Please note that \pharmml doesn't require or use numbers to be assigned to the various levels of variability. Instead the user can define meaningful identifiers.

\begin{figure}[htb!]
\centering
  \includegraphics[width=120mm]{IOV-general-TREE}
 \caption{General nested hierarchy of the variability structure -- as tree. Note that \pharmml doesn't require or use numbers to be assigned to the various levels of variability. Instead the user can define meaningful identifiers.}
 \label{IOVgeneral_tree}
\end{figure}

\begin{figure}[htb!]
\centering
  \includegraphics[width=100mm]{IOV-general-VENN}
 \caption{General nested hierarchy of the variability structure -- as Venn diagram.}
 \label{IOVgeneral_venn}
\end{figure}


\paragraph{Example 1}
This example handles the simplest scenario, with only one level of variability: \textit{subject}--level, see Figure \ref{tree_IOV0}. The following symbols are used
\begin{itemize}
\item
$i$ -- subject index, $1\le i \le N$
\end{itemize} 
with $N_l$ -- number of subjects.\\
The typical parameter model, without covariate, reads as follows:
\begin{align*}
& \log(V_i) = \log(V_{pop}) + \eta_i^{(0)}  
\end{align*} 
or alternatively:
\begin{align*}
& V_i = V_{pop} \,e^{\eta_i^{(0)}}  
\end{align*} 
with $\eta_i^{(0)} \sim \mathcal{N}\big(0,\Omega^{(0)}\big)$.


\begin{figure}[htb!]
\centering
  \includegraphics[width=120mm]{tree_IOV0}
 \caption{Example 1 -- single level of variability: \textit{subject}-- level}
 \label{tree_IOV0}
\end{figure}

\paragraph{Example 2}
In this example there are three levels of variability: \{\textit{centre, subject, occasion}\}, see Figure \ref{tree_IOV1}. Following symbols are used:
\begin{itemize}
\item
$l$ -- centre index, $1\le l \le L$
\item
$i$ -- subject index, $1\le i \le N_l$
\item
$k$ -- occasion index, $1\le k \le N_{li}$
\end{itemize} 
with
\begin{itemize}
\item
$L$ -- number of centres
\item
$N_l$ -- number of subjects in centre \textit{l}
\item
$N_{li}$ -- number of occasions in subject \textit{i} in centre \textit{l}
\end{itemize} 
The parameter model, without covariate, reads as follows:
\begin{align*}
& \log(V_{lik}) = \log(V_{pop}) + \eta_l^{(1)} + \eta_{li}^{(0)} + \eta_{lik}^{(-1)}  
\end{align*} 
or alternatively:
\begin{align*}
& V_{lik} = V_{pop} \, e^{\eta_l^{(1)}} e^{\eta_{li}^{(0)}} e^{\eta_{lik}^{(-1)}}  
\end{align*} 
with
\begin{align*}
 & \eta_l^{(1)} \sim \mathcal{N}\big(0,\Omega^{(1)}\big), \quad \eta_{li}^{(0)} \sim \mathcal{N}\big(0,\Omega^{(0)}\big),
\quad \eta_{lik}^{(-1)} \sim \mathcal{N}\big(0,\Omega^{(-1)}\big) 
\end{align*}


\begin{figure}[htb!]
\centering
  \includegraphics[width=120mm]{tree_IOV1}
 \caption{Example 1 -- three levels of variability: \{\textit{centre, subject, occasion}\}}
 \label{tree_IOV1}
\end{figure}


\paragraph{Example 3}
In this example there are four levels of variability: \{\textit{country, centre, subject, occasion}\}, see Figure \ref{tree_IOV2}. The symbol list is extended by one for 'country' as follows:
\begin{itemize}
\item
$m$ -- country index, $1\le m \le M$
\item
$l$ -- centre index, $1\le l \le N_m$
\item
$i$ -- subject index, $1\le i \le N_{ml}$
\item
$k$ -- occasion index, $1\le k \le N_{mli}$
\end{itemize} 
with
\begin{itemize}
\item
$M$ -- number of countries
\item
$N_m$ -- number of centres in country \textit{m}
\item
$N_{ml}$ -- number of subjects in centre \textit{l} in country \textit{m}
\item
$N_{mli}$ -- number of occasions in subject \textit{i} in centre \textit{l} in country \textit{m}
\end{itemize} 
The parameter model reads as follows:
\begin{align*}
& \log(V_{mlik}) = \log(V_{pop}) + \eta_m^{(2)} + \eta_{ml}^{(1)} + \eta_{mli}^{(0)} + \eta_{mlik}^{(-1)}  
\end{align*} 
or alternatively:
\begin{align*}
& V_{mlik} = V_{pop} \, e^{\eta_m^{(2)}} e^{\eta_{ml}^{(1)}} \; e^{\eta_{mli}^{(0)}} \; e^{\eta_{mlik}^{(-1)}}  
\end{align*} 
with
\begin{align*}
 & \eta_m^{(2)} \sim \mathcal{N}\big(0,\Omega^{(2)}\big), \quad \eta_{ml}^{(1)} \sim \mathcal{N}\big(0,\Omega^{(1)}\big), \quad
 \eta_{mli}^{(0)} \sim \mathcal{N}\big(0,\Omega^{(0)}\big), \quad \eta_{mlik}^{(-1)} \sim \mathcal{N}\big(0,\Omega^{(-1)}\big) 
\end{align*} 



\begin{figure}[htb!]
\centering
  \includegraphics[width=120mm]{tree_IOV2}
 \caption{Example 2 -- four levels of variability:  \{\textit{country, centre, subject, occasion}\}}
 \label{tree_IOV2}
 \end{figure}


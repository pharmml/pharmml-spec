\chapter{Unresolved issues}
\label{chap:unresolved}

\begin{issues}
\iss{1}{Can symbol resolution rules be simplified?} Currently the rules for symbol resolution (see section~\ref{sec:ref-symbol-resolution}) define the class of symbol that can be referred to. For example in the GeneralCovariate parameter definition, we restrict any equation defined there to only reference parameters and covariates --- random variables are prohibited. This ensures that this part of the \pharmml document is used correctly, but it is it too restrictive?
\iss{2}{More work to define operations and algorithms in \pharmml.} There is not specification for what estimation operations and algorithms should be supported by \pharmml. Ideally algorithm definitions will be supported by external resources such as KiSAO (\url{http://biomodels.net/kisao/}), but there is no support there yet.
\iss{3}{The way we map a dataset column to an independent variable is not consistent.} In the Estimation Step we map to the independent variable symbol (t) using a \xelem{SymbRef} element and in the Trial Design we use the \xelem{IndependentVariableMapping} element. It will simplify the rules if w are consistent.
\iss{4}{Units} We had planned to introduce units into this release using the mechanism adopted by SBML. However, their approach does not enable the encoding of temperature in either Fahrenheit or Celsius (because these conversions require the addition of a constant). SBML only alow temperature in Kelvin as a result. Do we wish to follow their approach or try and find a different solution?
\iss{5}{Interpolation} When estimating from experimental data it is often necessary to use time-points between those for which we have experimental data. Software interpolate between the known data-points to obtain a value, but of course there is more than one way to do this. The approach taken is tool specific, so the question is do we wish to specify what interpolation method is used in the estimation step?
\iss{6}{Use of the \xatt{columnNum} in the dataset definition.} In version 0.1.0 of \pharmml the dataset read from a tabular ascii file, such as a tab delimited file. Because of this we used the \xatt{columnNum} to map the column in the data-file to that in the dataset. Now that the data is defined in XML this use of the \xatt{columnNum} is not used and the columnNum had been reinterpreted to define the order of the column in the dataset definition. This is superfluous and the column order could be easily defined by the order in the XML document - as it is for the contents of the \xelem{Row} element.
\end{issues}
